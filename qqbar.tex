
\documentclass{article}

\usepackage{nopageno}

\usepackage{vmargin}
\usepackage{times}
\usepackage{graphics}
\usepackage{amsmath, amscd, amsxtra, amsthm}
\usepackage{amssymb}
\usepackage[retainorgcmds]{IEEEtrantools}

\usepackage{comment}
\usepackage{listings}
\usepackage{minted}

% This package makes cut-and-paste of carets work right, so
% we can cut-and-paste from the PDF document into Maxima.

\usepackage[T1]{fontenc}

\newcommand{\Bold}[1]{\mathbf{#1}}
\newcommand{\CC}{\Bold{C}}
\newcommand{\QQ}{\Bold{Q}}
\newcommand{\Norm}{{\rm Norm}}
\newcommand{\QQbar}{\overline{\QQ}}

\parindent 0pt
\parskip 12pt

\begin{document}
\title{Factoring polynomials over $\QQbar$ using Singular {\tt absfact}}
\author{Brent Baccala}
\date{June 13, 2018}
\maketitle

%\begin{abstract}
%\end{abstract}

The Singular {\tt absfact} library provides routines for the {\it
  absolute factorization} of polynomials with integer coefficients.
Sage uses {\tt absfact} to implement factorization of polynomials with
coefficients in {\tt QQbar}, its implementation of the algebraic
number field.

Singular {\tt absfact} will factor a multivariate polynomial $p$ with
integer coefficients, returning a set
of irreducible polynomials, each with an associated number field
$\mathrm{NF} = \QQ(\beta)$, with an irreducible polynomial $n(\beta)=0$
such that the polynomial and all of its conjugates are factors of
$p$.  The conjugates correspond precisely with the complex
numbers $\beta$ that solve $n(\beta)=0$, and can be formed by looping
over all possible embeddings of $\mathrm{NF}$ into $\CC$, or
equivalently $\QQbar$.

The Sage implemention is based on \cite{geddes}'s observation (in
section 8.8) that all of a polynomial's factors are factors of its
norm, and the norm has only rational coefficients.  Thus, Singular's
{\tt absfact} can be used to factor the norm; all Sage has to do is
determine which factors belong to the original polynomial and which
were introduced when applying the norm.

More specifically, the {\it norm} of an element in a number field is
defined as the product of the element with all of its conjugates, i.e,
the product of the elements obtained by applying all possible
automorphisms, or equivalently the product of the elements obtained
by all possible embeddings of the number field into its algebraic
closure.  The norm can be extended to polynomial rings over the
number field, and can be computed in Sage as follows:

\begin{minted}[frame=single]{python}
norm_f = prod([f.map_coefficients(h)
               for h in numfield.embeddings(QQbar)])
\end{minted}

The utility of the norm lies in the facts that it commutes with
multiplication and that the norm of a number field element lies
in $\QQ$, and the norm of a polynomial over a number field is
a polynomial over $\QQ$.

% Consider a polynomial $p$ with coefficients in $K = Q(\alpha) \supset
% Q$, where $K$ is an algebraic extension of $Q$ and $\alpha$ is its
% primitive element with minimal polynomial $m(\alpha)=0$, equipped with
% a mapping into $\CC$ (the complex field).  We wish to determine its
% factorization over $\QQbar$.

% We are provided with a factorization of $\Norm(p)$, presented as a set
% of irreducible polynomials, each with an associated number field
% $\mathrm{NF} = Q(\beta)$, with an irreducible polynomial $n(\beta)=0$
% such that the polynomial and all of its conjugates are factors of
% $\Norm(p)$.  The conjugates correspond precisely with the complex
% numbers $\beta$ that solve $n(\beta)=0$, and can be formed by looping
% over all possible embeddings of $\mathrm{NF}$ into $\CC$, or
% equivalently $\QQbar$.

% We first check to see if the associated number field can be embedded
% in the original number field, i.e. is there a mapping $\beta \to
% b(\alpha)$ such that $n(b(\alpha)) = 0$ in $K$.

% We first check to see if our original number field can be embedded
% in the polynomial's associated number field, i.e. is there a mapping
% $\alpha \to a(\beta)$ such that $m(a(\beta)) = 0$ in $N$.  If so,
% then we use it to map our original polynomial to $N$ and compute
% the multiplicity of the factor.  Then we consider all possible
% mappings of $N$ into $\CC$, (i.e, consider $\beta$ to each root
% of $n(\beta)$.  For each compatible mapping...

% $p$ has factorization $p_1 \cdots p_n$.  Its norm has factorization
% $\Norm(p_1)\cdots \Norm(p_n)$.  Factor $p_1$ has an associated number
% field $N = Q(\beta)$ with an irreducible polynomial $n(\beta)=0$, and
% a mapping $\beta \to \CC$ such that $p_1$ is a polynomial in $N[x]$.

% If there is a mapping $\alpha \to a(\beta)$ such that $m(a(\beta)) =
% 0$ in $N$, then there is a unique mapping $\beta \to \CC$ such that $a(\beta)$ maps
% to the same complex number that $\alpha$ maps to,

Taking the norm produced a polynomial with integer coefficients,
so we can now use {\tt absfact}, and since the norm is a product
of the original polynomial, all of the original polynomial's
factors are present in the norm, although taking the norm
introduced extraneous factors, so now we'd
like to do essentially this:

\begin{minted}[frame=single]{python}
for hom in NF.embeddings(QQbar):
    factor_f = factor_NF.map_coefficients(hom)
    for i in itertools.count(1):
        if f % factor_f**i != 0:
            multiplicity = i-1
            break
    if multiplicity > 0:
        factorization.append((factor_f, multiplicity))
\end{minted}

This code works fine.  There are two problems with it, however.
First, it works for {\tt QQbar}, but how do we handle {\tt AA}?
Second, and more important, it relies on Sage's ability to divide
polynomials in {\tt QQbar}.  Sage does this by converting the polynomials
into a number field, using Singular to do the division, and then
converting the result back to {\tt QQbar}.  This is inefficient, and
at the time of this writing, hasn't yet been merged into the released
codebase.

Instead, we'd prefer to do this calculation directly in the number
fields, without converting back and forth to {\tt QQbar}.  We've
already got the polynomial we're trying to factor in a number field
called {\tt numfield}, equipped with an embedding into {\tt QQbar}
(a homomorphism named {\tt morphism}), plus a number field
called {\tt NF}, where the factor resides.
I was thinking that I could construct the union of the two number
fields, and do something like this:

% By the definition of absolute factorization (?), we have to multiply
% all of the conjugates together to form a polynomial in $Q[x]$.

% The first question we need to answer is whether the intersection of
% {\tt numfield} and {\tt NF} is $Q$, or a larger number field.  If
% their intersection is $Q$, then all of the conjugates have the same
% multiplicity.

% Assume the converse, that their intersection is a number field $K \supset Q$.

% If our original polynomial is written with coefficients as polynomials
% in alpha, where alpha is numfield's generator, can NF map into those
% polynomials even if it can't map to alpha?

%a^4 = 1

%x+(a^2)


\begin{minted}[frame=single]{python}
# Essentially, union_fld = numfield.union(NF), except that number
# fields don't have a union method

gen = qq_generator
gen = gen.union(morphism(numfield.gen(0))._exact_field())
union_fld = gen.union(NF.embeddings(QQbar)[0](NF.gen(0))._exact_field()).field()

for NF_embed in NF.embeddings(union_fld):
    for numfield_embed in numfield.embeddings(union_fld):

        factor_union = factor_NF.map_coefficients(NF_embed)
        f_union = numfield_f.map_coefficients(numfield_embed)

        for i in itertools.count(1):
            if f_union % factor_union**i != 0:
                multiplicity = i-1
                break

        if multiplicity > 0:
            for hom in union_fld.embeddings(QQbar):
                if hom * numfield_embed != morphism:
                    continue
                factor_QQbar = factor_union.map_coefficients(hom)
                factorization.add((factor_QQbar, multiplicity))
\end{minted}

However, this doesn't work.  Consider factoring

$$f = (x^2+\sqrt{2})(x+\sqrt[8]{2})$$

The presence of the eighth root causes {\tt numfield} to be
$\QQ(\alpha)$, $\alpha^8=2$, and $f$ becomes:

$$f = (x^2+\alpha^4)(x+\alpha)$$
$$\Norm(f) = (x^4-2)^4(x^8-2)$$

The first factor creates
{\tt NF} $=\QQ(\beta)$, $\beta^4=2$, and the union of {\tt NF}
with {\tt numfield} is {\tt numfield} itself.  The factor is $x-\beta$,
which embeds into {\tt numfield} as $x-\alpha^2$.

The problem is that $x-\alpha^2$ doesn't divide $x^2+\alpha^4$
in $\QQ(\alpha)$.

The reason is that {\tt NF} isn't a {\it splitting field}
for $\Norm(f)$.
In {\tt NF}, the norm's first factor factors as
$(x^4-2) = (x-\beta)(x+\beta)(x^2+\beta^2)$.  But $f$'s
factor is $(x^2+\alpha^4) = (x^2+\beta^2)$, which
isn't divisible by $x-\beta$.

At first, I thought that iterating over all possible embeddings
would deal with this problem, but in this example there
are only two possible embeddings of {\tt NF} into {\tt numfield}
and all the other one does is flip the sign of $\beta$.

\begin{comment}
\begin{minted}[frame=single]{python}
numfield_f = f.map_coefficients(elem_dict.__getitem__, new_base_ring=numfield)

def NF_elem_map(e):
    return NF(e.numerator()) / NF(e.denominator())

factor_NF = factor.map_coefficients(NF_elem_map, new_base_ring=NF)

homs = numfield.embeddings(NF)

if len(homs) > 0:

    for numfield_into_NF in homs:

        numfield_NF = numfield_f.map_coefficients(numfield_into_NF)

        for i in itertools.count(1):
            if numfield_NF % (factor_NF)**i != 0:
                multiplicity = i-1
                break

        if multiplicity > 0:
            for hom in NF.embeddings(QQbar):
                if hom * numfield_into_NF == morphism:
                    QQbar_factor = factor_NF.map_coefficients(hom)
                    factorization.append((QQbar_factor, multiplicity))

else:

    assert multiplicity % numfield.degree() == 0
    multiplicity = multiplicity // numfield.degree()

    for hom in NF.embeddings(QQbar):
        QQbar_factor = factor_NF.map_coefficients(hom)
        factorization.append((QQbar_factor, multiplicity))
\end{minted}
\end{comment}

So, I've gone back to the first version of the code and modified it to
handle {\tt AA} by discarding all complex embeddings that map to the
lower half plane (negative imaginary component) and doubling up
(via conjugation) on the complex embeddings that map to the upper
half plane.  Real embeddings (imaginary component zero) are left
alone.  This produces completely real factors if the original
polynomial was completely real.

\begin{minted}[frame=single]{python}
for hom in NF.embeddings(QQbar):
    factor_f = factor_NF.map_coefficients(hom)
    if f.base_ring() is AA:
        target = hom(NF.gen(0))
        if target.imag() < 0:
            continue
        elif target.imag() > 0:
            conjugate_hom = NF.hom([target.conjugate()], QQbar)
            factor_f *= factor_NF.map_coefficients(conjugate_hom)
        factor_f = factor_f.change_ring(AA)
    for i in itertools.count(1):
        if f % factor_f**i != 0:
            multiplicity = i-1
            break
    if multiplicity > 0:
        factorization.append((factor_f, multiplicity))
\end{minted}



\begin{thebibliography}{9}

\bibitem[Ge92]{geddes}
Geddes, Czapor, Labahn, {\it Algorithms for Computer Algebra}. Springer.
ISBN: 978-0-585-33247-5

\end{thebibliography}

\begin{comment}
\begin{minted}[frame=single]{python}
# We now have a number field and a factor in that number field such
# that the factor and all of its conjugates multiply together to
# form a factor of the original polynomial's norm.

homs = numfield.embeddings(NF)

# Can we embed our original numfield into NF?

if len(homs) > 0:

    for numfield_into_NF in homs:

        # We can't rely on the multiplicity computed by Singular,
        # since there might be multiple factors in the original
        # polynomial that give rise to the same factor in the
        # norm.  Therefore, we recompute the multiplicity here
        # for each possible embedding.

        numfield_NF = numfield_f.map_coefficients(numfield_into_NF)

        for i in itertools.count(1):
            if numfield_NF % (factor_NF)**i != 0:
                multiplicity = i-1
                break

        # Then, for all of the conjugates, check to see if composing
        # with numfield_into_NF produces the original morphism that
        # maps number_field into QQbar.  If so, this conjugate is
        # a factor of the original polynomial, and not just its norm.

        if multiplicity > 0:
            for hom in NF.embeddings(QQbar):
                if hom * numfield_into_NF == morphism:
                    QQbar_factor = factor_NF.map_coefficients(hom)
                    factorization.append((QQbar_factor, multiplicity))

else:

    # No.  This corresponds to a factor of the original polynomial
    # that doesn't require numfield, so add all of its conjugates
    # to factorization.

    assert multiplicity % numfield.degree() == 0
    multiplicity = multiplicity // numfield.degree()

    for hom in NF.embeddings(QQbar):
        QQbar_factor = factor_NF.map_coefficients(hom)
        factorization.append((QQbar_factor, multiplicity))
\end{minted}
\end{comment}

\end{document}
