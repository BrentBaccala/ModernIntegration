
At this point, there is only one major missing piece in our
integration theory for simple radicals --- how do we limit the
multiples of a divisor to a testable set?  We've seen how to
repeatedly raise a divisor to higher and higher powers, but how do we
know when to stop?  At what point can we declare that a divisor has no
multiple that is principle?

We'll attack this problem the same way we attacked polynomial
factorization in Chapter ?, by mapping into a finite field, solving a
corresponding problem there, then somehow lifting the result back to
the original field.  The details differ, but the basic idea is the
same.  Of course, divisors, like polynomials, behave somewhat
different in finite fields, so our first task is to study
some of their unique properties in this domain.

Let's start with a simple but crucial observation:

THEOREM

In an algebraic extension over a finite field, the evaluation field is
also finite.

PROOF

Consider a finite field of constants ${\cal F}$, over which we'll
extend first into a rational function field ${\cal F}(x)$ and then add
an algebraic extension ${\cal F}(x,y)$, where $y$ satisfies some
minimial polynomial $f(x,y)=0$.  Start with the constant field, which
gives us a finite number of values for $x$.  Plugging each of these
values into the minimal polynomial gives a finite set of polynomials
$f(y_i)=0$.  By Theorem ?, we can extend ${\cal F}$ into a finite
extension field ${\cal F}[\gamma]$ where all the roots of the
polynomial exist.  Since there a only a finite number of polynomials,
we need at worst a finite set of extensions ${\cal
F}[\gamma_1,...,\gamma_k]$ to construct a field in which all the roots
of all the polynomials exist.  Using the Theorem of the Primitive
Element, we can collapse all of these into a single finite extension
field ${\cal F}[\phi]$.  Since all values of $x$ exist in ${\cal F}$,
and all values of $y$ exist in ${\cal F}[\phi]$, an evaluation
homomorphism carries any rational function in $x$ and $y$ into
${\cal F}[\phi]^\infty$.

END THEOREM

This theorem leads directly to the single more important difference
(to us) between divisors in an infinite field versus those in a finite
field.  {\it In a finite field, some multiple of every divisor is
principle.}  The reason is that the multiplicative group of the
evaluation field has finite order.  The simplest way to demonstrate
this is to construct theorems analogous to Theorems ? and ?:

THEOREM

In an algebraic extension of a finite field with characteristic
greater than 2, a function can always be constructed with an $m^{\rm
th}$-order zero at a specified place $(\alpha, \beta)$ and zero order
at all other finite places, where $m$ is the multiplicative order of
the evaluation field.

PROOF

The desired function is

$$(x-\alpha)^m + (y-\beta)^m$$.

Clearly, this function is zero at $(\alpha, \beta)$ and of $m^{\rm
th}$ order there (PROOF THIS).  At all other places one of the two
terms will be non-zero, and both exist in the evaluation field.  By
Theorem ?, any non-zero number raised to the multiplicative order of
its field is one.  Thus the value of this function will be either
$1+0$, $0+1$, or $1+1=2$, all finite and non-zero, and thus of zero
order.

END THEOREM

THEOREM

In an algebraic extension of a finite field with characteristic
greater than 2, a function can always be constructed with an $m^{\rm
th}$-order pole at a specified place $(\alpha, \beta)$ and zero order
at all other finite places, where $m$ is the multiplicative order of
the evaluation field.

PROOF

The desired function is

$${f(\alpha,y)^m\over(x-\alpha)^m(y-\beta)^m} + 1$$

where the division by $(y-\beta)^m$ is exact.
Clearly, this function has a pole at $(\alpha, \beta)$ and of $m^{\rm
th}$ order there (PROOF THIS).  CONSIDER OTHER PLACES OVER $\alpha$.
At all other places the denominator
term will be non-zero, and thus one, and the numerator will be
either zero or one (by Theorem ?)
Thus the value of this function at these places will be either
$0+1$ or $1+1=2$, both finite and non-zero, and thus of zero
order.

END THEOREM


EXAMPLE

Show that some multiple of ${\mathrm Z}(1,1)$ is principle in
${\bf Z}_5(x,y); y^2=x$.

Let's first construct a multiplication table for ${\bf Z}_5$:

\begin{center}
\begin{tabular}{c|c c c c c}
  & 0 & 1 & 2 & 3 & 4 \cr
\hline
0 & 0 & 0 & 0 & 0 & 0 \cr
1 & 0 & 1 & 2 & 3 & 4 \cr
2 & 0 & 2 & 4 & 1 & 3 \cr
3 & 0 & 3 & 1 & 4 & 2 \cr
4 & 0 & 4 & 3 & 2 & 1 \cr
\end{tabular}
\end{center}

Now, let's list out the places on the Riemann surface for
${\bf Z}_5(x,y); y^2=x$.

\begin{center}
\begin{tabular}{c l}
$x$ & $(x,y)$ \cr
\hline
0 & (0,0) \cr
1 & (1,1) \quad (1,4) \cr
2 & $(2,\gamma) \quad (2,-\gamma); \quad \gamma^2 - 2 =0$ \cr
3 & $(3,\theta) \quad (3,-\theta); \quad \theta^2 - 3 =0$ \cr
4 & (4,2) \quad (4,3) \cr
\end{tabular}
\end{center}

It looks like we need ${\bf Z}_5[\gamma,\theta]$ to express these places.
It's simplest to collapse $\gamma$ and $\theta$ into a single algebraic
extension.  We could use the Theorem of the Primitive Element to
do this, but in this case just looking at the multiplication table
and noting that $3 = 2^3 = \gamma^6$ shows that $\theta = \pm \gamma^3$.
So, in fact, we only need ${\bf Z}_5[\gamma]$:

\begin{center}
\begin{tabular}{c l}
$x$ & $(x,y)$ \cr
\hline
0 & (0,0) \cr
1 & (1,1) \quad (1,4) \cr
2 & $(2,\gamma) \quad (2,-\gamma); \quad \gamma^2 - 2 =0$ \cr
3 & $(3,\gamma^3) \quad (3,-\gamma^3)$ \cr
4 & (4,2) \quad (4,3) \cr
\end{tabular}
\end{center}

Since ${\bf Z}_5[\gamma]$ has $5^2=25$ elements, its multiplicative
group has order one less than this.  We conclude that 24 is our
``magic'' multiple, and that ${\mathrm Z}^{24}(1,1)$ must be
principle in this field.  Its generator should be simply
$(x-1)^{24} + (y-1)^{24}$.  Clearly this function is zero for
$(x,y)=(1,1)$.  Let's verify that it's non-zero for some other
places on the Riemann surface:

\begin{eqnarray*}
(0,0) &:& (-1)^{24} + (-1)^{24} = 4^{24} + 4^{24} = 1+1 = 2 \cr
(1,4) &:& 3^{24} + 0^{24} = 1 + 0 = 1 \cr
(2,\gamma) &:& (\gamma-1)^{24} + (2-1)^{24} = 1+1 = 2 {\rm ,\quad since:} \cr
&&\cr
&& (\gamma-1)^2 = (\gamma^2-2\gamma+1) = 3-2\gamma \cr
&& (\gamma-1)^4 = (3-2\gamma)^2 = (9-12\gamma+4\gamma^2) = 2-2\gamma \cr
&& (\gamma-1)^8 = (2-2\gamma)^2 = (4-8\gamma+4\gamma^2) = 2-3\gamma \cr
&& (\gamma-1)^{12} = (2-2\gamma)(2-3\gamma) = (4-10\gamma+6\gamma^2) = 1 \cr
\end{eqnarray*}

In the final series of calculations, I used $\gamma^2=2$ and reduced
mod 5 repeatedly.  I think the pattern should be clear, and leave
further verification as an exercise.
