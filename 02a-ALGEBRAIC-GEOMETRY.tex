
\mychapter{Algebraic Geometry}

In pure algebra, such as we've studied in the last chapter, $x$ is just $x$.  It does not
``take a value'' and has no other interpretation.  Everything we did in the last chapter
is based solely on the axioms of a commutative ring or field.

We next want to consider what happens when we let $x$ take a value, say a real number,
and then conclude that the equation $x^2-1=0$ is true if $x$ is either $-1$ or $1$.

This is now {\it algebraic geometry}.

The roots of algebraic geometry lie in studying the zeros of
polynomial equations.  We began with a single polynomial in a single
variable, and have learnt a great deal about it.  We know how to solve
it (at least in terms of radicals) if its degree is less than 5.
Galois proved that no such solution (in radicals) exists (in the
general case) for larger degree, though abstract algebra provides us
with a suitable general theory to handle this case.  Simple long
division tells us that it can have no more roots than its degree, and
Gauss showed that all of the roots exist as complex numbers --- the
Fundamental Theorem of Algebra.

The next logical step is to consider zeros of a single polynomial in
two variables, and this equation has also received a great deal of
attention from mathematicians.  Like the univariate case, we have
theories devoted to low-order special cases --- {\it linear equations}
(all terms first degree or constant), the {\it conic sections} (all
terms second degree or less), and the {\it elliptic curves} (one term
third degree; all others second degree or less).  In the general case,
$\sum a_{ij} x^i y^j = 0$ is called an {\it algebraic curve}.

\mysection{Solving systems of equations}

Probably the most basic application of primary decomposition is solving systems of equations.

An ideal in a polynomial ring has an associated {\it variety}, which is the set of points that
zero all of the polynomials in the ideal.  A variety can be expressed as the intersection of
its irreducible components, each of which corresponds precisely with a primary ideal in
the primary decomposition.

If the underlying field of constants has no zero divisors (as is the case with ordinary real
or complex numbers), then we are justified in taking the radical of the ideal, and then
the primary decomposition is just the prime decomposition because our primary ideals
are all prime ideals.

Returning to the $(xy,xz)$ example, we found that this ideal's primary decomposition
is $(x) \cap (y,z)$.  In other words, the solution variety of the system
of equations $xy=0$, $xz=0$ is formed by the $z$-axis $y=0$ $z=0$ and the $x-y$ plane $x=0$.
