
\mychapter{The Risch Theorem}

{\bf THIS CHAPTER IS VERY VAGUE AND INCOMPLETE.}

\section{Jacobian Varieties}

An algebraic extension is a simple example of what algebraic geometers
term a {\it variety}, which is the zero locus of a set of polynomials
defined over some field.  Thus, for example, the unit circle is a
variety (defined over the real numbers), because it is the zero locus
of $x^2+y^2=1$.  But the points $(1,0)$ and $(-1,0)$ are also a
variety, because they are the zero locus of the {\it set} of
polynomials $\{x^2=1; y=0\}$.

An {\it abelian variety} is a variety accompanied by a commutative
group structure on its elements, which typically includes picking an
arbitrary zero point as the identity element.  The circle is an
abelian variety, if we identify its points with their angles from the
x-axis and make $(1,0)$ our identity element.  Now any two points can
be ``added'' or ``subtracted'' (by adding or subtracting their
respective angles) to obtain a third point, and each point has an
inverse associated with it (its mirror image across the x-axis).  It
should be obvious that the choice of a zero point was totally
arbitrary.  Likewise, the points ${(1,0), (-1,0)}$ also form an
abelian variety; their group structure is isomorphic to ${\bf Z}_2$ and
the choice of one of them as the identity is, again, arbitrary.

Is every variety abelian?  No, but any complete, non-singular variety
can be homomorphicly mapped into an associated abelian variety
(typically of higher dimension), called its {\it Jacobian variety}.
This fact, combined with the extensive body of literature on abelian
varieties ([Mumford], [Birkenhake and Lange], [Lang], to mention a
few), makes the Jacobian variety an important object of study (though
David Mumford, in the preface to [Mumford], described it as a
``crutch'').

We will be needing only a tiny bit of this theory here, so my goal in
this section is only to demonstrate how the Riemann-Roch Theorem
allows us to set up an abelian group structures on an algebraic
extension.

\section{The Riemann-Roch Theorem}

The Riemann-Roch Theorem is one of the most celebrated theorems in
mathematics.  Not only does it provide a crucial tool in understanding
the structure of algebraic extensions, but it does so by tying
together algebra, analysis, and geometry in one equation.

First, let's review that equation:

\theorem {\rm (Riemann-Roch)}

% For any divisor $\mathfrak{b}$,
% 
% $$l(\mathfrak{b}) = \deg \mathfrak{b} + 1 - g + l(\mathfrak{c}-\mathfrak{b}) $$
% 
% where $l(\mathfrak{b})$ is the dimension of the vector space
% $L(\mathfrak{b})$ of multiples of $-\mathfrak{b}$, $g$ is the genus of
% the extension, and $\mathfrak{c}$ is any divisor of the canonical class of
% differentials.

For any divisor $\mathfrak{b}$,

% $$l(\mathfrak{b}) = \deg \mathfrak{b} + 1 - g + l(\mathfrak{c}-\mathfrak{b}) $$
% $$l(-\mathfrak{b}) = \deg \mathfrak{b} + 1 - g + l(\mathfrak{b}-\mathfrak{c}) $$
% $$l(\mathfrak{b}) = \deg -\mathfrak{b} + 1 - g + l(-\mathfrak{b}-\mathfrak{c}) $$
% $$l(\mathfrak{b}) = - \deg \mathfrak{b} + 1 - g + l(-\mathfrak{b}-\mathfrak{c}) $$

$$l(\mathfrak{b}) = - \deg \mathfrak{b} + 1 - g + l(-\mathfrak{b}-\mathfrak{c}) $$

where $l(\mathfrak{b})$ is the dimension of the vector space
$L(\mathfrak{b})$ of multiples of $\mathfrak{b}$, $g$ is the genus of
the extension, and $\mathfrak{c}$ is any divisor of the canonical class of
differentials.

\endtheorem

Interrelated by this theorem is the purely algebraic concept of the
dimension of the vector space of multiples of a divisor, the geometric
concept of the genus, and the analytic concept of a differential.

However, this sophistication comes with a price.  Specifically, we
need a topology to define the genus, and we need a limit to define the
differential.

Andr\'e Weil showed how the Riemann-Roch theorem can be stripped of
the analysis and the geometry, and proved as purely a result in
algebra.  The genus, instead of a topological invariant, now appears
as merely a least upper bound on a divisor's degree of specialization,
and a differential becomes an object in a dual space that maps a
function into the field of constants.  The advantage of this
formulation is that does not require any topological structure, and is
therefore well suited to use with finite fields.  It is this
formulation I will now adopt.

First, at any place in the function field, there is a local valuation
ring with a maximal prime ideal.  We can normalize the valuation (it
is discrete) and pick a element of unit valuation to use as a
uniformizing variable.  By multiplying as necessary by some power of
this element, we can adjust any field element to be a unit of the
valuation ring and thus associate an order ${\rm ord}_{\mathfrak{p}}$
with that element.  The valuation ring's units are a finite extension
of the constant subfield; they are the constant subfield if it is
algebraically closed.  By subtracting out the remainder mod
$\mathfrak{p}$, we get an element of higher order, which we can again
subtract out, and so on, building a power series in the uniformizing
variable.  Each element of the function field thus has a power series
associated with it at each place $\mathfrak{p}$.

A collection of such power series, one at each place, with arbitrary
coefficients except that there are only a finite number of
coefficients with negative powers, is called a {\it vector}.  Each
individual power series is called a {\it component} of the vector.
Clearly, every function has a vector associated with it; but the
converse is not necessarily true.  The mapping from functions to
vectors is injective, though.  Any two different functions will have a
non-zero difference that must therefore have a finite value, of finite
order, at some place $\mathfrak{p}$, and their vectors will differ at
that point.

We also have a dual space of {\it covectors}.  The coefficients of a
covector component at a place are dual to the constant field at that
place; if the constant field is algebraically closed, then the
covector coefficients are in the constant field.  Like vectors,
covectors can only have a finite number of negative power
coefficients.

We define a dot product between a vector $v$ and a covector $\lambda$:

$$ v \cdot \lambda = \sum_{\mathfrak{p}} \sum_{i+j=-1} v_{\mathfrak{p},i}
  \lambda_{\mathfrak{p},j} $$

where $v_{\mathfrak{p},i}$ is the coefficient of the $i^{\rm th}$
power in $v$'s component at $\mathfrak{p}$, and likewise for
$\lambda_{\mathfrak{p},j}$.  Notice that the second summation requires
at least one of $i$ or $j$ to be negative, so there will only be a
finite number of places for the first sum at which the second sum
contributes anything at all.

Weil also requires the {\it Theorem of Independence}, which states
that, although an arbitrary (full) vector may not have a function
associated with it, a function can always be found which matches a set
of finite prefixes at a finite number of places.  This can be
demonstrated using Theorem \ref{finite orders construction},
repeatedly applied a finite number of times.  We also need to know
that a function without a pole is constant.

With this setup, we can now prove a series of theorems that lead up
the the Riemann-Roch Theorem.

\theorem

$$l(\mathfrak{p}) \leq \deg \mathfrak{p} +1$$

i.e, $l(\mathfrak{p})$, the dimension (over the constants) of
$L(\mathfrak{p})$, the multiples of $-\mathfrak{p}$, is no more than
the degree of the divisor $\mathfrak{p}$, plus one.

\proof

Since $\deg -\mathfrak{p} = - \deg \mathfrak{p}$, there are at least
$\deg \mathfrak{p}$ poles (counting multiplicities) in
$-\mathfrak{p}$, and at least $\deg \mathfrak{p}$
coefficients with negative powers in the vectors corresponding
to the elements in $L(\mathfrak{p})$

.  We can impose

\endtheorem

\theorem

If $\mathfrak{A}$ is divisible by $\mathfrak{B}$, i.e, if
$\mathfrak{A}\mathfrak{B}^{-1} \subseteq {\cal I}$, then

$$n(\mathfrak{A}) - l(\mathfrak{A}) \leq n(\mathfrak{B}) - l(\mathfrak{B}) $$

$$n(\mathfrak{p}) \equiv {\rm deg} \mathfrak{p}$$

\proof

Consider $\mathfrak{C} = \mathfrak{A}\mathfrak{B}^{-1}$.
Now $n(\mathfrak{C}) = n(\mathfrak{A}) - n(\mathfrak{B})$ and since
$\deg -\mathfrak{C} = - \deg \mathfrak{C}$, and
$\mathfrak{C}$ is integral (by supposition),
there are exactly
$n(\mathfrak{C})$ poles (counting multiplicities) in $-\mathfrak{C}$.
, and at least $\deg \mathfrak{p}$
coefficients with negative powers in the vectors corresponding
to the elements in $L(\mathfrak{p})$

.  We can impose

\endtheorem

Back to the Riemann-Roch Theorem...

It immediately follows (from $\mathfrak{b}={\bf 0}$) that
$l(\mathfrak{c})=g$, which can be taken as the definition of the
genus.

We can now pick $g$ independent differentials from $\mathfrak{c}$ and
use them (along with an arbitrary origin) to map into the torus
${\bf C}/\Lambda^g$.

Now, Abel's Theorem and the Jacobi inversion theorem ([Griffiths and
Harris], p. 235) shows that ${\rm Pic}^0$, the group of divisors of
degree zero modulo linear equivalence is isomorphic to ${\bf
C}/\Lambda^g$.

Alternately, ([Lang], II, \S1, Theorem 3), we can factor a mapping
of a product into an abelian variety into mappings on each factor.

Lang also characterizes Abel's theorem as follows:

\begin{quote}

Let $\omega_1, ..., \omega_g$ be a basis for the differential forms
on the first kind of V.  If $\mathfrak{a} = \sum n_i P_i$ is a
[divisor] of degree 0 on V, and P is a fixed point of V, then
the map into ${\bf C}/\Lambda^g$ given by:

$$\mathfrak{a} \to \sum n_i (\int_P^{P_i}\omega_1, ..., \int_P^{P_i}\omega_g)$$

is well defined modulo the periods... the kernel consists of those
divisors that are linearly equivalent to 0 (i.e, principle); this is
Abel's theorem.

\end{quote}


\section{Endomorphism Rings}

Any commutative group $G$ induces a (non-commutative) ring structure
on its endmorphisms, defined as follows (remember that an
endomorphism is a homomorphism from an object to itself):

Two endmorphisms $\phi(g): G \to G$ and $\gamma(g): G \to G$ are added
using $G$'s group operation on the images: $(\phi+\gamma)(g) =
\phi(g)\cdot\gamma(g)$, where $\cdot$ denotes the group operation.
The additive identify is the endmorphism that maps the entire group
onto its identity element.

Two endmorphisms $\phi(g): G \to G$ and $\gamma(g): G \to G$ are
multiplied using composition of mappings: $(\phi\gamma)(g) =
\phi(\gamma(g))$.  The multiplicative identity is the endmorphism that
maps every element in the group onto itself.

Let us now verify that these operations define a ring, the {\it endomorphism
ring} of G, which we shall denote ${\rm End}(G)$.  The properties
of the identity elements are fairly obvious, I think.  Almost as
obvious is that the associative and commutative properties of the
underlying group translate directly into additive associative and
commutative properties in the endmorphism ring.  The multiplicative
properties follow from composition of mappings being associative, but
not necessarily commutative.  The distributive law follows from the
easily verified identity $\phi(\gamma(g)\cdot\mu(g)) = \phi(\gamma(g)) \cdot
\phi(\mu(g))$, using the fact that $\phi$ is an endomorphism, and thus
a homomorphism, and therefore maps the group operator through.

The ring of integers ${\bf Z}$ can be mapped homomorphicaly\footnote{An easy
consequence of ${\bf Z}$'s repelling universal property in the category of
rings, see [Lang], p. ?} into any ring, and an endomorphism ring is no
exception.  We'll denote by $[m]$ the endmorphism mapped to by the
integer $m$. $[0]$ is clearly the additive identity mapping all
elements to the group identity.  $[1]$ is, of course, the
multiplicative identity mapping all elements to themselves.  $[2]$ is
$[1]+[1]$, the endmorphism that composes each element with itself
(using the group operator): $[2]: g \to g\cdot g$.  $[3]$ composes
each element with itself thrice: $[3]: g \to g\cdot g\cdot g$, etc.

Because ${\bf Z}$ is commutative, the subring $[m]$ it maps to is also
commutative, even though ${\rm End}(G)$ may not be.



\section{Good Reduction}

My notes from [Sh61].

{\small\begin{verbatim}

A function (Y -> k) is regular at a point on a variety if there exists
an open neighborhood of the point where the function is given by a
rational function of polynomials, the denominator never zero.
(relation to coordinate ring?)

A map (Y -> k) is regular on Y if it is regular at every point of Y.

A morphism (X -> Y) between varieties is a continuous map (in the
Zariski topology) such that pullbacks of regular functions are
regular.

A rational map is a morphism defined only on an open subset.

Given a rational map from affine varieties X to Y, if [x,y,z] is Y's
coordinate system, then we pullback to a regular function, which is a
rational function in X's function field.  So Y's coordinates are given
by rational functions in X's coordinates.

A group variety is an algebraic variety equipped with a group
structure, where the group operation and group inversion are rational
maps.  If the group operation is commutative, it's an abelian variety.

A homomorphism between abelian varieties is a rational map that
commutes with the group operation.

An endomorphism is a homomorphism from the variety to itself.

Endomorphisms of a group form a ring.  Addition is performed by
mapping through both endomorphisms, then applying the group operation,
which is commutative, so endomorphism addition is commutative.
Multiplication is performed by composition, and need not be
commutative.

Using just the addition structure, we get an abelian group that can be
structued as a Z-module.  Multiplication by an integer is just
repeated application of the endomorphism.

We can promote the endomorphism ring into an algebra by tensoring with
Q, call this EndQ(A).  Elements of EndQ(A) are basically endomorphisms
with an associated 1/n denominator (numerators can be sucked into the
endomorphism).

A lattice is a free Z-module of the same rank as the algebra over Q.

An order is a lattice that is also a subring and contains the identity.

The order of A, written t, is the image of the endomorphism ring in
the endomorphism algebra.

a is a lattice in the endomorphism algebra contained in t, so it's a
collection of actual endomorphisms.

g(a,A) is the set of points on A mapped to 0 by every element of a.

Prop 16. Let a be an integral ideal (p. 49) of F.  Reduction mod p
defines a homomorphism of g(a,A) onto g(a,A^).  If a is prime to the
characteristic of k^, this homomorphism is an isomorphism.


Div(C) is group of divisors
Div0(C) is group of degree 0 divisors
P(C) is group of principle divisors

P(C) in Div0(C) in Div(C)

define Pic(C) = Div(C)/P(C)
define Pic0(C) = Div0(C)/P(C)

Pic0(C) is isomorphic to the Jacobian variety with a point O.

Given a degree zero divisor in Div0(C), we can use the group law on
the Jacobian to construct a single point corresponding to the divisor.
Asking if the divisor is principle is asking if this point is O.
Asking if any multiple of the divisor is principle is asking if any
multiple of this point is O.

We can define an endomorphism to be addition by a point, using the
group law.  NO - doesn't take identity to identity.

Let's consider [n], the endomorphism defined by applying the group
operation n times (n is an integer).  This should generate a lattice
contained in t, so it's an integral ideal.  g([n], A) is the set of
points whose n-multiples are principle.

Given a divisor whose (k p^q)-multiple is principle, let's multiply by
p^q and get a divisor whose k-multiple is principle.  Endomorphism
ideal [k] is prime to p.  Then this divisor point will be in g(a,A)
and g(a,A^), where a is [k].

We can determine that the divisor point is in g(a,A^) for a=[k] by
determining that the divisor's k-multiple is principle on the module
curve.  Since this is an isomorphism, the divisor point is also in
g(a,A) for a=[k].




Given a non-singular algebraic curve C, we reduce mod p to get Cp.
Good reduction implies that Cp is non-singular with the same genus.  C
has an associated Jacobian J.  Cp also has an associated Jacobian Jp.

PROBLEM: (hopefully) Show that J mod p is Jp.

Construct J as follows.  Pick r > 2g-2.  Symmetric group J^(r).  Pick
r-g extra points and find an open covering of J^(r).  Within each
open set, construct J locally.

\end{verbatim}
}
