
% Slideshow, written by Brent Baccala, for a lecture at Catholic University
%
% A screencast of the original lecture is available on youtube:
%
% https://www.youtube.com/watch?v=JwsuAEF2FYE

\documentclass[aspectratio=169,dvipsnames]{beamer}
\usetheme{Madrid}

\title{Techniques of Modern Integration}
\author{Brent Baccala}
\institute{\tt cosine@freesoft.org}
\date{April 3, 2019}

\setbeamertemplate{footline}{}
\beamertemplatenavigationsymbolsempty

%\usepackage{vmargin}
\usepackage{times}
\usepackage{graphics}
\usepackage{amsmath, amscd, amsxtra, amsthm}
\usepackage{amssymb}
\usepackage[retainorgcmds]{IEEEtrantools}
\usepackage{framed}
\usepackage{mdframed}
\usepackage{float}

\usepackage{comment}
\usepackage{graphicx}

\usepackage{tabularx}

\usepackage{ragged2e}

%% to scale '<' for Bronstein's notation
\usepackage{scalerel}

%% Suggested by http://tex.stackexchange.com/questions/349580
\usepackage{array}

%% Define a new column declaration
\newcolumntype{L}[1]{>{\raggedright\arraybackslash}p{#1}}

\newcommand{\ud}{\,\mathrm{d}}

\newcommand{\N}{{\rm N\,}}
\newcommand{\lc}{{\rm lc\,}}

\usepackage{clipboard}
\openclipboard{ModernIntegration}

\usepackage[usefamily=sage,keeptemps=all,pygments=false]{pythontex}


% These packages are used when embedding Maxima code

\usepackage{breqn}      % auto-break long equations - has to come after pythontex
\usepackage{listings}
\usepackage{fancyvrb}

\usepackage{tikz}
\usetikzlibrary{positioning, fit, backgrounds, arrows, trees, shapes, shadows, calc}

\usepackage{adjustbox}

\begin{document}

\newtheorem{idea}{Idea}

% Do similar formatting with Sage

\newcommand{\sageinputcode}{}
\newcommand{\sageoutputmath}[1]{#1}

%% Sage preamble - should get this by calling latex_extra_preamble() in sage.misc.latex

\newcommand{\ZZ}{\Bold{Z}}
\newcommand{\NN}{\Bold{N}}
\newcommand{\RR}{\Bold{R}}
\newcommand{\CC}{\Bold{C}}
\newcommand{\QQ}{\Bold{Q}}
\newcommand{\QQbar}{\overline{\QQ}}
\newcommand{\GF}[1]{\Bold{F}_{#1}}
\newcommand{\Zp}[1]{\ZZ_{#1}}
\newcommand{\Qp}[1]{\QQ_{#1}}
\newcommand{\Zmod}[1]{\ZZ/#1\ZZ}
\newcommand{\CDF}{\Bold{C}}
\newcommand{\CIF}{\Bold{C}}
\newcommand{\CLF}{\Bold{C}}
\newcommand{\RDF}{\Bold{R}}
\newcommand{\RIF}{\Bold{I} \Bold{R}}
\newcommand{\RLF}{\Bold{R}}
\newcommand{\CFF}{\Bold{CFF}}
\newcommand{\Bold}[1]{\mathbf{#1}}

%% XXX - check if we can turn autoprint off and leave it turned off

\renewenvironment{sageblock}{\renewcommand{\sageoutputmath}[1]{}\setpythontexautoprint{false}\sagecode}{\setpythontexautoprint{true}\endsagecode

% At end of sageblock, redefine Sage processing commands,
% then call \printpythontex to include the Sage output,
% which will call the Sage processing commands.

% Command called by python code that processes maximacode input
% Command number is in argument 1
% Command text is saved in verbatim block MaximaCode
%
% We need a \break between two adjacent \sageinputcode's,
% but we leave it out if there's an intervening \sageoutputmath.

\definecolor{mytextcolor}{rgb}{0,0,0}

\def\sageinputcodebreak{}
\renewcommand{\sageinputcode}{
\sageinputcodebreak
\bgroup
\def\arraystretch{1.5}
\begin{tabular}{L{1cm} L{10cm}}
\textcolor{blue}{\ttfamily sage:} &
\textcolor{mytextcolor}{\BUseVerbatim[baseline=t]{SageCode}}
\end{tabular}
\egroup
\def\sageinputcodebreak{\break}
}

\renewcommand{\sageoutputmath}[1]{
\def\sageinputcodebreak{}
\setlength\abovedisplayskip{0pt}
\setlength\belowdisplayskip{0pt}
\setlength\abovedisplayshortskip{0pt}
\setlength\belowdisplayshortskip{0pt}
\begin{dmath*}
##1
\end{dmath*}
}

% Wrap everything in a nice colored frame and print it
\begin{mdframed}[backgroundcolor=yellow!20]
\tiny
\printpythontex
\end{mdframed}
}

\newenvironment{sagespacedblock}{\renewcommand{\sageoutputmath}[1]{}\setpythontexautoprint{false}\sagecode}{\setpythontexautoprint{true}\endsagecode

% At end of sageblock, redefine Sage processing commands,
% then call \printpythontex to include the Sage output,
% which will call the Sage processing commands.

% Command called by python code that processes maximacode input
% Command number is in argument 1
% Command text is saved in verbatim block MaximaCode
%
% We need a \break between two adjacent \sageinputcode's,
% but we leave it out if there's an intervening \sageoutputmath.

\def\sageinputcodebreak{}
\renewcommand{\sageinputcode}{
\sageinputcodebreak
\bgroup
\def\arraystretch{1.5}
\begin{tabular}{L{1cm} L{10cm}}
\textcolor{blue}{\ttfamily sage:} &
\textcolor{Brown}{\BUseVerbatim[baseline=t]{SageCode}}
\end{tabular}
\egroup
\def\sageinputcodebreak{\break}
}

\renewcommand{\sageoutputmath}[1]{
\def\sageinputcodebreak{}
\begin{dmath*}
##1
\end{dmath*}
}

% Wrap everything in a nice colored frame and print it
\begin{mdframed}[backgroundcolor=yellow!20]
\printpythontex
\end{mdframed}
}

% sagecommon - don't put "sage:" prompts before the input lines,
% and don't print output at all

\newcounter{sagecommoncounter}
\setcounter{sagecommoncounter}{0}

\newenvironment{sagecommon}
  {\stepcounter{sagecommoncounter}\VerbatimOut{sagecommon-\thesagecommoncounter}}
  {\endVerbatimOut
\begin{mdframed}[backgroundcolor=yellow!20,fontcolor=brown]
\VerbatimInput{sagecommon-\thesagecommoncounter}
\end{mdframed}
}

\newenvironment{sagecommonsmall}{\small\sagecommon}{\endsagecommon}
\newenvironment{sagecommontiny}{\tiny\sagecommon}{\endsagecommon}

% sagecommon.sage was generated by the Makefile after a LaTeX run.
% Now we insert a load command at beginning of every sage session
% to load it in.

\begin{pythontexcustomcode}{sage}
load("sagecommon.sage");
\end{pythontexcustomcode}

\setbeamertemplate{enumerate item}[default]
\setbeamercolor{enumerate item}{fg=black}

\begin{frame}
\titlepage
\begin{block}{}
\centerline{\tt https://www.freesoft.org/ModernIntegration}
\end{block}
%%\begin{block}{Abstract}
%%An introduction to Risch integration.
%%\end{block}
\end{frame}

\begin{frame}
\frametitle{Who Wants to be a Mathematician?}
\def\QuestionFont{\Huge\bf}
\def\AnswerFont{}
\def\huge{}
\def\LARGE{}
\Paste{WWTBAM}
\end{frame}

\begin{frame}
\frametitle{Elementary (Liouvillian) Forms}
\tiny
\def\sech{{\rm sech}}
\def\csch{{\rm csch}}

\begin{columns}[T]
\begin{column}{.48\textwidth}

\begin{tabular}{c c c c}
Expression & \multicolumn{1}{c}{Liouvillian Form} &
Expression & \multicolumn{1}{c}{Liouvillian Form} \\
\hline
& \\
$f^g$ & $\displaystyle e^{\,g \ln f}$ &
 & \\
$\sin x$ & $\displaystyle {-i \,{{e^{ix} - e^{-ix}}\over 2}}$ \vbox to20pt{}&
 $\sinh x$ & $\displaystyle {{e^{x} - e^{-x}}\over 2}$ \vbox to20pt{} \\
$\cos x$ & $\displaystyle {{e^{ix} + e^{-ix}}\over 2}$ &
 $\cosh x$ & $\displaystyle {{e^{x} + e^{-x}}\over 2}$ \vbox to20pt{} \\
$\tan x$ & $\displaystyle {-i \,{{e^{ix}-e^{-ix}}\over {e^{ix}+e^{-ix}}}}$ &
 $\tanh x$ & $\displaystyle {{e^{x}-e^{-x}}\over {e^{x}+e^{-x}}}$ \\

$\sec x$ & $\displaystyle {2\over{e^{ix} + e^{-ix}}}$ &
 $\sech\, x$ & $\displaystyle {2\over{e^{x} + e^{-x}}}$ \vbox to20pt{} \\
$\csc x$ & $\displaystyle {{2i}\over{e^{ix} - e^{-ix}}}$ \vbox to20pt{}&
 $\csch\, x$ & $\displaystyle {2\over{e^{x} - e^{-x}}}$ \vbox to20pt{} \\
$\cot x$ & $\displaystyle {i \,{{e^{ix}+e^{-ix}}\over {e^{ix}-e^{-ix}}}}$ &
 $\coth x$ & $\displaystyle {{e^{x}+e^{-x}}\over {e^{x}-e^{-x}}}$ \\

\end{tabular}
\end{column}
\begin{column}{.48\textwidth}

\begin{tabular}{c c c c}
Expression & \multicolumn{1}{c}{Liouvillian Form} &
Expression & \multicolumn{1}{c}{Liouvillian Form} \\
\hline
& \\

$\arcsin x$ & $\displaystyle -i \,\ln (ix + \sqrt{1-x^2})$ &
 $\sinh^{-1} x$ & $\displaystyle \ln (x + \sqrt{x^2+1})$ \\
$\arccos x$ & $\displaystyle -i \,\ln (x + i\sqrt{1-x^2})$ &
 $\cosh^{-1} x$ & $\displaystyle \ln (x + \sqrt{x^2-1})$ \\
$\arctan x$ & $\displaystyle {1\over2}\,i\,\ln {{ix-1}\over{ix+1}}$ &
 $\tanh^{-1} x$ & $\displaystyle {1\over2} \ln {{1+x}\over{1-x}}$ \\
$

\sec^{-1} x$ & $\displaystyle -i \,\ln {{1 + i\sqrt{x^2-1}}\over{x}}$ &
 $\sech^{-1} x$ & $\displaystyle {1\over2} \ln {{1+\sqrt{1-x^2}}\over{1-\sqrt{1-x^2}}}$ \\
$\csc^{-1} x$ & $\displaystyle -i \,\ln {{i + \sqrt{x^2-1}}\over{x}}$ &
 $\csch^{-1} x$ & $\displaystyle {1\over2} \ln {{\sqrt{1+x^2}+1}\over{\sqrt{1+x^2}-1}}$ \\
$\cot^{-1} x$ & $\displaystyle {1\over2}\,i\,\ln {{i+x}\over{i-x}}$ &
 $\coth^{-1} x$ & $\displaystyle {1\over2} \ln {{x+1}\over{x-1}}$ \\

\end{tabular}
\end{column}
\end{columns}

\end{frame}

\begin{frame}
\frametitle{Testing $\ln$ for transcendence}
\begin{theorem}
$\ln u$ is transcendental over $k$ and
$\qquad\Longleftrightarrow\qquad$
$\frac{u'}{u}$ is not a derivative of any element of $k$

the constant subfield remains fixed
\end{theorem}
\begin{proof}
$\Rightarrow$
Assume the contrary, that $v'=\frac{u'}{u}$ for some $v \in k$.  Then $\ln u = v$.

\bigskip
$\Leftarrow$
Assume the contrary, that $a=\ln u$ exists in $K$, an algebraic extension of $k$.

Let $a$'s minimal polynomial be $p(a) = a^m + \cdots + c_0$.  Differentiating,
we get $p'(a) = m \frac{u'}{u} a^{m-1} + \cdots + c_0' = 0$, contradicting the
minimality of $p(a)$ if $m>1$.

If $m=1$, then $p'(a) = \frac{u'}{u} + c_0' = 0$,
so $\frac{u'}{u}$ is the derivative of $-c_0$.

\end{proof}
\end{frame}

\begin{frame}
\frametitle{Logarithmic Derivatives}
\begin{definition}
In a differential field $k$,
$u$ is a {\bf logarithmic derivative} of $v \in k$ if there exists an
integer $n$ such that
\[ nu=\frac{v'}{v} \]
\end{definition}
\[ v = (\exp f)^n \]
\[ v' = n f' (\exp f)^n \]
\[ \frac{v'}{v} = n f' \]
\end{frame}

\begin{frame}
\frametitle{Logarithmic Derivatives (cont)}
\begin{theorem}[Bronstein]
If $u$ is not a logarithmic derivative in a differential field $k$,
then it is not a logarithmic derivative in any algebraic extension of $k$.
\end{theorem}
\begin{proof}
Assume the contrary: that $K$ is an algebraic extension of $k$ in which $u$
is a logarithmic derivative.  Let $\alpha$ be the element of $K$ for which
$nu=\frac{\alpha'}{\alpha}$ and let $p(\alpha)$ be $\alpha$'s minimal polynomial in $K$
\[ p(\alpha) = \alpha^m + \cdots + k_0 = 0 \]
\[ p(\alpha)' = m n u \alpha^m + \cdots + k_0' = m n u p(\alpha) = m n u (\alpha^m + \cdots + k_0)\]
\[ \frac{k_0'}{k_0} = m n u \]
\end{proof}
\end{frame}

\begin{frame}
\frametitle{Testing $\exp$ for transcendence}
\begin{theorem}
$\exp u$ is transcendental over $k$
$\qquad\Longleftrightarrow\qquad$
$u'$ is not a logarithmic derivative of any $v \in k$

and the constant subfield remains fixed
\end{theorem}
\begin{proof}
$\Rightarrow$
Assume the contrary, that $u'$ is the logarithmic derivative of $v \in k$,
so $nu' = \frac{v'}{v}$.  $v' = n u' v$, so $v = (\exp u)^n$.
\bigskip

$\Leftarrow$
Assume the contrary, that $a=\exp u$ exists in $K$, an algebraic extension of $k$.

Then $a' = u' \exp u = u' a$, and $\frac{a'}{a} = u'$, so $u'$ is a logarithmic
derivative in $K$, contradicting (Bronstein's theorem)
that it is not a logarithmic derivative in $k$.
\end{proof}
\end{frame}

\begin{frame}
\frametitle{$\int \frac{4^x+1}{2^x+1} {\rm d} x$}
Three ways to represent $\frac{4^x+1}{2^x+1}$ in Liouvillian form
\begin{itemize}
\item The Easy Way

$${\mathbb C}(x,\Psi = \exp(x \ln 2))$$

$$ \frac{\Psi^2+1}{\Psi+1}$$

\item The Hard Way

$${\mathbb C}(x,\Psi = \exp(x \ln 4), \xi ) \qquad \xi^2=\Psi$$

$$ \frac{\Psi+1}{\xi+1}$$

\end{itemize}
\end{frame}

\begin{frame}
\frametitle{$\int \frac{4^x+1}{2^x+1} {\rm d} x$}
%% Three ways to represent $\frac{4^x+1}{2^x+1}$ in Liouvillian form
\begin{itemize}
\item The Wrong Way

$${\mathbb C}(x,\Psi,\Phi) \qquad \Psi = \exp(x \ln 4) \qquad \Phi = \exp(x \ln 2)$$

$$ \frac{\Psi+1}{\Phi+1}$$

$$ \frac{\Psi'}{\Psi} = \frac{(\ln 4) \Psi}{\Psi} = 2 \ln 2 = 2 (x \ln 2)'$$

i.e, $(x \ln 2)$ is a logarithmic derivative in ${\mathbb C}(x,\Psi)$

\bigskip
Implies an algebraic relationship between $\Psi$ and $\Phi$

\end{itemize}
\end{frame}

\begin{frame}
\Huge
\centerline{The Logarithmic Extension}
\end{frame}

\begin{frame}
\frametitle{The Logarithmic Integration Theorem}
\small
\Paste{LIT1}
\end{frame}

\begin{frame}
\frametitle{The Logarithmic Integration Theorem (cont)}
\tiny
\Paste{LIT2}
\end{frame}

\begin{frame}[fragile]
\frametitle{The \$50,000 integral}
$$\int{{x\{(x^2e^{2x^2}-\ln^2(x+1))^2+2xe^{3x^2}(x-(2x^3+2x^2+x+1)\ln(x+1))\}}\over{(x+1)(\ln^2(x+1) - x^2e^{2x^2})^2}} dx$$
\tiny
\begin{sageblock}
integrand = x                                           \
   *((x^2*exp(2*x^2)-log(x+1)^2)^2                      \
      +2*x*exp(3*x^2)*(x-(2*x^3+2*x^2+x+1)*log(x+1)))   \
   / ((x+1)*(log(x+1)^2 - x^2*exp(2*x^2))^2)
\end{sageblock}

\end{frame}

\tikzstyle{field} = [draw,drop shadow={opacity=.4,shadow xshift=0.04, shadow yshift=-0.04},rounded corners=3]

\begin{frame}[fragile]
\frametitle{The \$50,000 integral (cont)}

\small
\begin{tabular}{ m{.5\textwidth} m{.5\textwidth} }
We put our integral into Liouvillian form, \break assigning $\psi = \exp(x^2)$ and $\theta = \ln (x+1)$.
&
  \begin{tikzpicture}
    \small
    
    \node (log field) [field, minimum height=150, minimum width=150, fill=blue!45] {};
    \node (log label) [below=5pt] at (log field.north) {$\CC(x,\psi,\theta) \qquad \theta = \log (x+1)$};

    \node (exp field) [field, fill=blue!30, minimum height=100, minimum width=130] {};
    \node (exp label) [below=5pt] at (exp field.north) {$\CC(x,\psi) \qquad \psi = \exp x^2$};

    \node (rational field) [field, minimum height=40, minimum width=50, fill=white] {$\CC(x)$};

  \end{tikzpicture}
\\
\end{tabular}

\begin{sageblock}
var('theta, psi');
lintegrand = integrand.subs( {log(x+1) : theta, exp(x^2) : psi, exp(2*x^2) : psi^2, exp(3*x^2) : psi^3})
\end{sageblock}

\end{frame}

\begin{frame}[fragile]
\frametitle{The \$50,000 integral (cont)}

\begin{sageblock}
F.<x,psi> = FractionField(ZZ['x', 'psi']);
R.<theta> = F['theta']

D = Derivation(R, {x: 1, theta: 1/(x+1), psi: 2*x*psi})

num = R(lintegrand.numerator(False))
den = R(lintegrand.denominator(False))
\end{sageblock}

\end{frame}

\begin{frame}[fragile]
\frametitle{The \$50,000 integral (cont)}

\begin{sageblock}
(a,N) = num.quo_rem(den)
A = integrate(SR(a), x)
n = [f[0] for f in factor(den)]

b = partfrac(N, den);
displayarray(b);
\end{sageblock}

\end{frame}

\begin{frame}[fragile]
\frametitle{The \$50,000 integral (cont)}

\begin{sagecode}
R = {};
B = {};
Q = {};
C = {};
\end{sagecode}

\begin{sageblock}
R[0,1] = b[n[0],2]
B[0,1] = - R[0,1] / D(n[0])
Q[0,1] = -(R[0,1] + B[0,1] * D(n[0])) / n[0]
C[0] = (b[n[0],1] - D(B[0,1]) - Q[0,1]) / D(n[0])
\end{sageblock}

\begin{sageblock}
R[1,1] = b[n[1],2]
B[1,1] = - R[1,1] / D(n[1])
Q[1,1] = - (R[1,1] + B[1,1] * D(n[1])) / n[1]
C[1] = (b[n[1],1] - D(B[1,1]) - Q[1,1]) / D(n[1])
\end{sageblock}

\end{frame}

\begin{frame}[fragile]
\frametitle{The \$50,000 integral (cont)}

\begin{sageblock}
lans = A + sum([B[i,1]/n[i] for i in range(2)]) \
  + sum([2 * C[i] * log(n[i]) for i in range(2)]).simplify_log()/2
\end{sageblock}

\begin{sageblock}
ans = lans.subs({theta : log(x+1), psi : exp(x^2)})
\end{sageblock}

\begin{sageblock}
bool(diff(ans,x) == integrand)
\end{sageblock}

\end{frame}

\begin{frame}
\Huge
\centerline{The Exponential Extension}
\end{frame}

\begin{frame}
\frametitle{The Exponential Integration Theorem}
\small
\Paste{EIT1}
\end{frame}

\begin{frame}
\frametitle{The Exponential Integration Theorem (cont)}
\tiny
\Paste{EIT2}
\end{frame}

\begin{frame}[fragile]
\small
\frametitle{Risch Equations over $\CC(x)$}
\Paste{C(x) Risch Equation}

Let's consider a single pole at
$\gamma$ and expand $S$, $T$, and $r$ using partial fractions:

$$r = \frac{a}{(x-\gamma)^j} + \cdots  \qquad  r' = \frac{-ja}{(x-\gamma)^{j+1}} + \cdots$$

$$S = \frac{b}{(x-\gamma)^k} + \cdots \qquad T = \frac{c}{(x-\gamma)^l} + \cdots$$

Combining everything into the Risch equation \eqref{eq: C(x) Risch}, we find:

$$\frac{-ja}{(x-\gamma)^{j+1}} + \cdots + \frac{ba}{(x-\gamma)^{j+k}} + \cdots = \frac{c}{(x-\gamma)^l} + \cdots$$

There are three cases to consider:

\begin{enumerate}

\item $k<1$.  The $\frac{-ja}{(x-\gamma)^{j+1}}$ term dominates the left hand side,
and $j = l-1$

\item $k=1$.  Either $j=l-1$ or $j=b$.

\item $k>1$.  The $\frac{ba}{(x-\gamma)^{j+k}}$ term dominates the left hand side, so $j=l-k$

\end{enumerate}

By checking all of $S$'s and $T$'s poles, we can
identify all the poles in $r$'s denominator and determine the
multiplicity with which they appear.

\bigskip

This determines $r$'s denominator.

\end{frame}

\begin{frame}[fragile]
\small
\frametitle{Risch Equations over $\CC(x)$ (cont)}

We now have a polynomial Risch equation
that must be satisfied by the numerator $p$:

\Paste{C[x] Risch Equation}

Our next aim is to upper bound the degree of $p$, and there are again three cases.

\begin{enumerate}
\item $p$ is a constant and $pB = C$
\item $\deg p = \deg C - \max(\deg A - 1, \deg B)$
\item $\deg A = \deg B + 1$ and $\deg p =-\frac{\lc B}{\lc A} $

$$A = a_j x^j + \cdots \qquad B = b_{j-1} x^{j-1} + \cdots$$
$$p = p_k x^k + \cdots \qquad p' = k p_k x^{k-1} + \cdots$$

$$A p' + B p = (k a_j p_k + b_{j-1} p_k) x^{j+k-1} \cdots$$

\end{enumerate}

\end{frame}

\begin{frame}[fragile]
\tiny
\frametitle{The Error Function}

\begin{tabular}{ m{.5\textwidth} m{.5\textwidth} }
$$\int e^{-x^2}\, {\rm d}x$$

We'll use ${\mathbb C}(x, \psi = \exp\, -x^2)$, so $\psi' = -2x$ and
study

$$\int \psi \, {\rm d}x$$

&
\tikzstyle{field} = [draw,drop shadow={opacity=.4,shadow xshift=0.04, shadow yshift=-0.04},rounded corners=3]
  \begin{tikzpicture}
    
    \node (log field) [field, minimum height=50, minimum width=100, fill=blue!45] {};
    \node (log label) [below=5pt] at (log field.north) {$\CC(x,\psi) \qquad \psi = \exp\, (-x^2)$};

    \node (rational field) [field, minimum height=20, minimum width=30, fill=white, below=1pt] {$\CC(x)$};

  \end{tikzpicture}
\\
\end{tabular}

\begin{comment}

$$\int e^{-x^2}\, {\rm d}x$$

We'll use ${\mathbb C}(x, \psi = \exp\, -x^2)$, so $\psi' = -2x$ and
study

$$\int \psi \, {\rm d}x$$

\end{comment}

We know from the Exponential Integration Theorem that our
solution, if it exists, must have the form $A_1\psi$, where $A_1 \in
{\mathbb C}(x)$, and $A_1$ must satisfy equation \eqref{eq: exponential An}:

$$A_1' - 2x A_1 = 1$$

This is already a polynomial Risch equation, and $A_1'$ has only a
constant coefficient, so $A_1$ can not have a non-trivial denominator.

\bigskip

Identifying $A$ as $1$, $B$ as $-2x$, and $C$ as $1$, we see that
$C$ is not a constant
multiple of $B$, so a constant $A_1$ can't solve our equation (case 1).

\bigskip

We compute:

$$\deg C - \max(\deg A - 1, \deg B) = 0 - 1 = -1$$

so case 2 doesn't work.

\bigskip

Finally (case 3), we see that $\deg A \ne \deg B + 1$.

\bigskip

We conclude that no solution to this Risch equation exists in ${\mathbb C}(x)$,
so the integral can not be expressed in elementary form.

\end{frame}

\begin{frame}
\frametitle{Normal and Special Polynomials}

Irreducible polynomials in a differential ring can be characterized as follows:

\begin{itemize}
\item {\bf Normal} irreducible polynomials do not divide their own derivatives
\item {\bf Special} irreducible polynomials divide their own derivatives
\end{itemize}

In $\CC[x]$, or any $K[x]$ where all $k\in K$ are constant, all irreducible polynomials are {\bf normal}.

\bigskip

In a logarithmic transcendental extension, all irreducible polynomials are {\bf normal}.

\begin{proof}
In all of these cases,
differentiation lowers the degree of the polynomial, and the original polynomial is {\it irreducible}
\end{proof}

\end{frame}

\begin{frame}
\frametitle{Normal and Special Polynomials (cont)}

In an exponential transcendental extension $K[\psi]$, where $\psi = \exp \Psi$, $\psi$ is the
only {\bf special} irreducible polynomial; all other irreducible polynomials are {\bf normal}.

\begin{proof}
\begin{enumerate}
\item $\psi$ is special
$$\psi' = \Psi' \psi \qquad \left(\Psi'\psi\right) | \psi$$
\item All other irreducible polynomials are normal.  W.l.o.g. let $p$ be monic and special

$$p = \psi^n + \cdots + p_0$$
$$p' = n \Psi' \psi^n + \cdots + p_0' = n \Psi' p = (n \Psi') \psi^n + \cdots + n \Psi' p_0$$
%%$$p_0' = n \Psi' p_0$$
$$\frac{p_0'}{p_0} = n \Psi'$$
...contradicting the assumption that $\Psi'$ is not a logarithmic derivative
\end{enumerate}
\end{proof}

\end{frame}

\begin{frame}
\frametitle{Risch Equations over Fields with Normal Polynomials}
\tiny
Let $K$ be a differential field, and $K(\theta)$ be a
non-constant transcendental extension of $K$.

\begin{equation}
\label{eq: transcendental polynomial Risch}
r' + S r = T \qquad S,T,r \in K(\theta)
\end{equation}

What happens when our partial
fractions decomposition yields normal polynomials
in the denominators of $S$ or $T$?

$$S = \frac{b(\theta)}{n(\theta)^k} + \cdots
\qquad T = \frac{c(\theta)}{n(\theta)^l} + \cdots$$

$$r = \frac{a(\theta)}{n(\theta)^j} + \cdots  \qquad
r' = \frac{a'(\theta)n(\theta)-ja(\theta)n'(\theta)}{n(\theta)^{j+1}} + \cdots = \frac{-ja(\theta)n'(\theta)}{n(\theta)^{j+1}} + \cdots$$

Equation \ref{eq: transcendental polynomial Risch} becomes:

$$\frac{-ja(\theta)n'(\theta)}{n(\theta)^{j+1}} + \cdots + \frac{a(\theta) b(\theta)}{n(\theta)^{j+k}} + \cdots = \frac{c(\theta)}{n(\theta)^l} + \cdots$$

Both of the numerators on the left hand side could have $\theta$-degree greater than $\deg_\theta n(\theta)$,
so we divide them by $n(\theta)$:

%% $$R_1(\theta) = -ja(\theta)n'(\theta) \mod n(\theta) \qquad R_2(\theta) = a(\theta)b(\theta) \mod n(\theta)$$

%% $$\frac{R_1(\theta)}{n(\theta)^{j+1}} + \cdots + \frac{R_2(\theta)}{n(\theta)^{j+k}} + \cdots = \frac{c(\theta)}{n(\theta)^l} + \cdots$$

$$\frac{-ja(\theta)n'(\theta) \mod n(\theta)}{n(\theta)^{j+1}} + \cdots + \frac{a(\theta)b(\theta) \mod n(\theta)}{n(\theta)^{j+k}} + \cdots = \frac{c(\theta)}{n(\theta)^l} + \cdots$$

\begin{enumerate}

\item $k=0$ and $j = l-1$.

\item $k=1$ and either $j=l-1$ or $j = \frac{b(\theta)}{n'(\theta)} \mod n(\theta)$.

\item $k>1$ and $j=l-k$.

\end{enumerate}

\end{frame}

\begin{comment}
\begin{frame}[fragile]
\tiny
\frametitle{Risch Equations over Fields with Special Polynomials}
Let $K$ be a differential field, and $\psi=\exp\Psi$ be an
exponential extension of $K$.

\begin{equation}
\label{eq: general polynomial Risch}
r' + S r = T \qquad S,T,r \in K(\psi)
\end{equation}

What happens when our partial
fractions decomposition yields special polynomials
in the denominators of $S$ or $T$?

$$S = \frac{b}{\psi^k} + \cdots \qquad T = \frac{c}{\psi^l} + \cdots \qquad b,c \in K$$
$$r = \frac{a}{\psi^j} + \cdots  \qquad  r' = \frac{-j \Psi' a + a'}{\psi^{j}} + \cdots \qquad a \in K$$

Equation \ref{eq: general polynomial Risch} becomes:

$$\frac{-j \Psi' a + a'}{\psi^{j}} + \cdots + \frac{a b }{\psi^{k+j}} + \cdots
= \frac{c}{\psi^l} + \cdots$$

We get two cases:

\begin{enumerate}
\item $k=0$ and either $j=l$ or
$j = \frac{a' + a b}{a \Psi'}$.

\smallskip
Published algorithms exist to solve this equation (see Bronstein's {\it Symbolic Integration I}).
%% (assuming that this fraction is an integer).
\item $k>0$ and $j=l-k$.
\end{enumerate}

\end{frame}
\end{comment}

\begin{frame}[fragile]
\frametitle{Risch Equations over Fields with Special Polynomials}
\tiny
We now have a polynomial Risch equation, though there can still be
special factors in the denominator, i.e, negative powers of $\psi$:

\begin{equation}
\label{eq: special polynomial Risch}
A r' + B r = C \qquad A,B,C \in K[\psi] \quad r \in K(\psi) \qquad \psi = \exp \Psi
\end{equation}

$$r = \frac{a}{\psi^j} + \cdots  \qquad  r' = \frac{-j \Psi' a + a'}{\psi^{j}} + \cdots$$

If $A$ and $B$ have no $\psi$ factors, then their zeroth order coefficients will produce $j$-th order fractions:

$$ A(0) \frac{-j \Psi' a + a'}{\psi^{j}} + \cdots
+ B(0) \frac{a}{\psi^j} + \cdots  = C$$

Since $C$ is a polynomial, the fractions on the left must cancel, and we obtain:

$$\left[ -ja \Psi' + a' \right] A(0) + a B(0) = 0$$

$$ j\Psi' - \frac{a'}{a }  = \frac{B(0)}{A(0)}$$

Integrating, we obtain:

\begin{equation}
\label{special risch equation - denominator bound}
j \Psi - \ln a = \int \frac{B(0)}{A(0)} dx
\end{equation}

\end{frame}

\begin{frame}[fragile]
\frametitle{Risch Equations over Fields with Special Polynomials (cont)}
\tiny
\begin{equation}
\label{special risch polynomial equation}
A r' + B r = C \qquad A,B,C \in K[\psi] \quad r \in K\hstretch{0.5}{<}\psi\hstretch{0.5}{>}
\end{equation}

Now we want to upper bound the $\psi$-degree of $r$.

$$r = r_n \psi^n + \cdots \qquad r' = (r_n' + n r_n \Psi') \psi^n + \cdots$$

If there is no cancellation between the $Ar'$ and $Br$ terms in equation \ref{special risch polynomial equation}, then

$$\deg_\psi r = \deg_\psi C - \max(\deg_\psi A, \deg_\psi B)$$

However, if $A$ and $B$ have equal $\psi$-degree, then the leading term on the left hand side is:

$$\left(\lc A\, (r_n' + n r_n \Psi') + \lc B\, r_n\right)\psi^{n+\deg_\psi A} + \cdots$$

Cancellation is achieved if the following condition is met:

\begin{equation}
\label{special risch equation - numerator bound}
n \Psi' + \frac{r_n'}{r_n} = - \frac{\lc B}{\lc A}
\end{equation}

This equation has the same form as \ref{special risch equation - denominator bound}, so again, we integrate:

\begin{equation}
\label{special risch equation - numerator bound}
n \Psi + \ln r_n = - \int \frac{\lc B}{\lc A} dx
\end{equation}

If $A$ and $B$ have equal $\psi$-degree, and this integral has this desired form, then cancellation
is possible between the terms on the left hand side of
\ref{special risch polynomial equation}, and

$$\deg_\psi r = \max(n, \deg_\psi C - \max(\deg_\psi A, \deg_\psi B))$$



This degree can be negative, so long as it is no lower than
the lower degree bound determined earlier.
\end{frame}

\begin{frame}[fragile]
\frametitle{Bronstein's Integral}

\begin{tabular}{ m{.4\textwidth} m{.6\textwidth} }
$$\int \frac{e^x - x^2 + 2x}{(e^x + x)^2 x^2}e^{(x^2-1)/x+1/(e^x+x)} dx$$
&
  \begin{tikzpicture}
    \small
    
    \node (log field) [field, minimum height=60, minimum width=220, fill=blue!45] {};
%%    \node (log label) [below=5pt] at (log field.north) {$\CC(x,\psi,\theta) \qquad \theta = \log (x+1)$};
    \node (log label) [below=5pt, align=left, right=3pt of log field.west] {$\CC(x,\psi,\theta)$ \\ $\theta = \exp \left(\frac{x^2-1}{x}+\frac{1}{\theta+x}\right)$};

%%    \node (exp field) [field, fill=blue!30, minimum height=40, minimum width=100, anchor=east] at (log field.east) {};
    \node (exp field) [field, fill=blue!30, minimum height=40, minimum width=100, anchor=east, left=3pt of log field.east] {};
    \node (exp label) [below=5pt, align=left, right=3pt of exp field.west] {$\CC(x,\psi)$ \\ $\psi = \exp x^2$};

    \node (rational field) [field, minimum height=20, minimum width=30, fill=white, anchor=east, left=3pt of exp field.east] {$\CC(x)$};

  \end{tikzpicture}
\\
\end{tabular}

\begin{sageblock}[bronstein]
# First, we switch back to using the Symbolic Ring,
# instead of the ring variables from the last example.

var('x', 'theta', 'psi');

integrand = (exp(x) - x^2 + 2*x) / ((exp(x) + x)^2 * x^2) * exp((x^2-1)/x + 1/(exp(x)+x))

exponent = (x^2-1)/x + 1/(psi+x)

lintegrand = integrand.subs( {exp(x) : psi, exp(exponent) : theta})
\end{sageblock}

\end{frame}

\begin{sagecode}[bronstein]
# Now create a proper ring and convert to using its variables.

F.<x,psi> = FractionField(ZZ['x', 'psi']);
R.<theta> = F[];

D1 = Derivation(R, {x: 1, psi: psi});
D = Derivation(R, {x: 1, psi: psi, theta: D1(R(exponent))*theta});

A1f = function('A1', nargs=1);
A1 = A1f(x);
\end{sagecode}

\begin{frame}[fragile]
\frametitle{Bronstein's Integral (cont)}

\begin{sageblock}[bronstein]
a1 = lintegrand/theta
eq = diff(A1,x) + D(theta)/theta * A1 - a1
[BC, _], [A, _] = eq.numerator().coefficients(diff(A1,x));
[C, _], [B, _] = BC.numerator().coefficients(A1);
A
B
C
\end{sageblock}

\end{frame}

\begin{frame}[fragile]
\frametitle{Bronstein's Integral: $\deg_\psi A_1$ lower bound \hfill $\psi = \exp \Psi = \exp x$}
\tiny

\begin{sageblock}[bronstein]
R2 = ZZ['x']['psi'];
R2(A)
R2(B)
R2(C)
\end{sageblock}

\bigskip
$$r = \frac{a}{\psi^j} + \cdots$$

\begin{equation}
\tag{\ref{special risch equation - denominator bound}}
j \Psi - \ln a = \int \frac{B(0)}{A(0)} dx
\end{equation}

\begin{sageblock}[bronstein]
(A/B).subs(psi=0)
\end{sageblock}

\bigskip
Equation \ref{special risch equation - denominator bound}
does have the desired form with $j=1$ and $a$ an arbitrary constant.

\end{frame}

\begin{frame}[fragile]
\frametitle{Bronstein's Integral: $\deg_\psi A_1$ upper bound \hfill $\psi = \exp \Psi = \exp x$}
\tiny

\begin{sageblock}[bronstein]
R2(A)
R2(B)
R2(C)
\end{sageblock}

$$r = r_n \psi^n + \cdots$$

\begin{equation}
\tag{\ref{special risch equation - numerator bound}}
n \Psi + \ln r_n = - \int \frac{\lc B}{\lc A} = - \int \frac{x^2+1}{x^2}
\end{equation}

\begin{sageblock}[bronstein]
integrate(R2(B).lc() / R2(A).lc(), x)
\end{sageblock}

\bigskip
Equation \ref{special risch equation - numerator bound}
does {\it not} have the desired form, so
$$\deg_\psi r = \deg_\psi C - \max(\deg_\psi A, \deg_\psi B) = -1$$

\end{frame}

\begin{frame}[fragile]
\frametitle{Bronstein's Integral (cont)}
\tiny

$$A_1 = A_{1,-1} \psi^{-1} \qquad A_{1,-1} \in {\mathbb C}(x)$$

\begin{sagecode}[bronstein]
A1m1f = function('A1m1', latex_name='A_{1,-1}', nargs=1);
A1m1 = A1m1f(x);
\end{sagecode}

\begin{sageblock}[bronstein]
eq2 = eq.subs({A1 : A1m1/psi, diff(A1,x) : (diff(A1m1,x)*psi - A1m1*D(psi))/psi^2})

eq2.numerator().degree(psi)
eq2.numerator().coefficient(psi, 2)
eq2.numerator().coefficient(psi, 1)
eq2.numerator().coefficient(psi, 0)
\end{sageblock}

$$A_{1,-1} = 1 \qquad A_1 = \psi^{-1}$$

\end{frame}

\begin{sagecode}[bronstein]

# BUG FIX - the polynomial subs() routine doesn't call subs
# recursively, so only the top level polynomial var gets
# replaced.  The difference is the nested call to subs()

def bwbsubs(self, *x, **kwds):
    if len(x) == 1 and isinstance(x[0], dict):
        g = self.parent().gen()
        if g in x[0]:
            return self(x[0][g]).subs(x[0])
        elif len(x[0]) > 0:
            raise TypeError("keys do not match self's parent")
        return self
    return self(*x, **kwds)

# Python insanity!  This changes the class's method to use my patched
# version.

type(theta).subs = bwbsubs;

\end{sagecode}

\begin{frame}[fragile]
\frametitle{Bronstein's Integral (cont)}
\tiny

$$A_{1,-1} = 1 \qquad A_1 = \psi^{-1} \qquad F = A_1 \theta = \frac{\theta}{\psi}$$

\begin{sageblock}[bronstein]
lans = theta/psi
ans = lans.subs({theta : exp(exponent), psi : exp(x)})
bool(diff(ans,x) == integrand)
\end{sageblock}

\bigskip
$$\int \sage[bronstein]{integrand} dx = \sage[bronstein]{ans}$$

\end{frame}

\begin{frame}
\Huge
\centerline{The Algebraic Extension}
\end{frame}

\begin{frame}[fragile]
\frametitle{The Algebraic Toolkit}
\tiny
There are four basic operations we perform on a rational
function in order to integrate it:

\begin{enumerate}
\item We {\bf factor} its numerator and denominator
\begin{sagecode}[ch7-intro]
load("sagecommon.sage");
import re
R.<x> = QQ[];
n = (x-1)^3*(x+3)*(x+1);
d = (x-2)^2*(x-3);
p = n/d;
def partfrac2(n,d):
    r = partfrac(n%d,d)
    a = latex(n//d)
    a += '&'
    for k,v in sorted(r.items(), reverse=True):
        if k[1] > 1:
            denom = "({})^{}".format(latex(k[0]), k[1])
        else:
            denom = latex(k[0])
        if v < 0:
            a += "-\\frac{"+latex(-v)+"}{"+denom+"}"
        else:
            a += "+\\frac{"+latex(v)+"}{"+denom+"}"
    return a
def partfrac3(n,d):
    r = partfrac(n%d,d)
    #a = latex(n//d).replace("x", "\\frac{{1}}{{1/x}}")
    #a = re.sub("x", "\\frac{{1}}{{1/x}}", latex(n//d))
    a = re.sub("x(\^{[0-9]}|)", "\\\\frac{1}{1/x\\1}", latex(n//d))
    a += '&'
    for k,v in sorted(r.items(), reverse=True):
        if k[1] > 1:
            denom = "({})^{}".format(latex(k[0]), k[1])
        else:
            denom = latex(k[0])
        if v < 0:
            a += "-\\frac{"+latex(-v)+"}{"+denom+"}"
        else:
            a += "+\\frac{"+latex(v)+"}{"+denom+"}"
    return a
\end{sagecode}
$$\sage[ch7-intro]{p} = \frac{\sage[ch7-intro]{n.factor()}}{\sage[ch7-intro]{d.factor()}}$$

In an algebraic extension, we compute a {\bf divisor}.
\item We construct a {\bf partial fractions expansion}
\begin{IEEEeqnarray*}{rCCL}
\sage[ch7-intro]{p} & = & \sage[ch7-intro]{partfrac2(n, d)} \\
                    & = & \sage[ch7-intro]{partfrac3(n, d)}
\end{IEEEeqnarray*}
In an algebraic extension, we construct a {\bf principal parts expansion}.
\item We {\bf reconstruct} a function given a {\bf factorization} of its numerator and denominator.

\bigskip
In an algebraic extension, we use a {\bf Riemann-Roch algorithm} to reconstruct a function from its divisor.
\bigskip
\item We {\bf reconstruct} a function given a {\bf partial fractions expansion}.

\bigskip
In an algebraic extension, we solve a {\bf Mittag-Leffler problem} to reconstruct a function from its principal parts expansion.
\end{enumerate}

\end{frame}

\begin{frame}[fragile]
\frametitle{The Algebraic Toolkit (cont)}

\begin{sageblock}[ex1]
R.<x> = FunctionField(QQ)
f = (x^5 + x^4 - 6*x^3 + 2*x^2 + 5*x - 3)/(x^3 - 7*x^2 + 16*x - 12)
f.divisor()

\end{sageblock}

\end{frame}

\begin{frame}[fragile]
\frametitle{The Algebraic Toolkit (cont)}

Analyze the principal parts of $\frac{x}{y} \ud x$ on the curve $y^2 = 1 - x^2$

\begin{sagecode}[ex1]
QQbar.options.display_format = 'radical';
\end{sagecode}

\begin{sageblock}[ex1]
R.<x> = FunctionField(QQbar, implementation='kash')
L.<y> = R[]; F.<y> = R.extension(y^2 + x^2 - 1)

v = x/y*x.differential()
D = v.divisor()
table([[p, F.completion(p, prec=0)(v)] for p,m in D.list() if m < 0])
\end{sageblock}

\bigskip
There are two places at infinity because $lim_{x\to\infty} y = \pm i x$.

\end{frame}

\begin{frame}[fragile]
\frametitle{The Algebraic Toolkit (cont)}

On the curve $y^2 = x^3 - x$, find a function with a third-order pole at the origin, and first-order zeros at $(1,0)$, $(-1,0)$, and infinity.

\begin{sageblock}[ex1]
R.<x> = FunctionField(QQbar, implementation='kash'); L.<y> = R[]; F.<y> = R.extension(y^2 - x^3 + x);

F.places_infinite()

O = F.maximal_order();

zeros = [F.places_infinite()[0], O.ideal(x-1, y).place(), O.ideal(x+1, y).place()]

pole = O.ideal(x,y).place()

divisor = sum(map(lambda a: a.divisor(), zeros)) - 3 * pole

divisor.basis_function_space()

\end{sageblock}

\end{frame}

\begin{frame}[fragile]
\frametitle{Chebyshev's Integral}

%%\centerline{$\int {{2x^6+4x^5+7x^4-3x^3-x^2-8x-8}\over{(2x^2-1)^2\sqrt{x^4+4 x^3+2 x^2+1}}} \ud x$}

\begin{tabular}{ p{.5\textwidth} p{.5\textwidth} }
$$\int {{2x^6+4x^5+7x^4-3x^3-x^2-8x-8}\over{(2x^2-1)^2\sqrt{x^4+4 x^3+2 x^2+1}}} dx$$

&
  \begin{tikzpicture}[baseline = (log field.north)]
    \small    
    \node (log field) [field, minimum height=50, minimum width=175, fill=blue!45] {};
    \node (log label) [below=5pt] at (log field.north) {$\CC(x,y) \qquad y^2 = x^4+4 x^3+2 x^2+1$};

    \node (rational field) [field, minimum height=20, minimum width=30, fill=white, below=1pt] {$\CC(x)$};

  \end{tikzpicture}
\\
\end{tabular}

\begin{sagecode}[chebyshev]
QQbar.options.display_format = 'radical';
\end{sagecode}

\begin{sageblock}[chebyshev]
R.<x> = FunctionField(QQbar, implementation='kash'); L.<y> = R[];
root = x^4+4*x^3+2*x^2+1; F.<y> = R.extension(y^2 - root)
num = 2*x^6 + 4*x^5 + 7*x^4-3*x^3-x^2-8*x-8; den = (2*x^2-1)^2;
integrand = num/(den*y) * x.differential()
\end{sageblock}

\end{frame}

\begin{frame}[fragile]
\frametitle{Chebyshev's Integral (cont)}
\centerline{$\int {{2x^6+4x^5+7x^4-3x^3-x^2-8x-8}\over{(2x^2-1)^2\sqrt{x^4+4 x^3+2 x^2+1}}} \ud x$}

\begin{sageblock}[chebyshev]

den.factor()
table([[pl, F.completion(pl, prec=0)(integrand)] for pl in integrand.divisor_of_poles().support()])
\end{sageblock}

\end{frame}

\begin{frame}[fragile]
\frametitle{Chebyshev's Integral (cont)}

\begin{sageblock}[chebyshev]
from sage.rings.function_field.divisor import FunctionFieldDivisor
Dpoles = integrand.divisor_of_poles();
D = FunctionFieldDivisor(Dpoles.parent().function_field(), {p:m-1 for p,m in Dpoles.list()})
basis = D.basis_function_space()
\end{sageblock}

\end{frame}

\begin{frame}[fragile]
\frametitle{Chebyshev's Integral (cont)}

We now want to solve a matrix equation:

$$M \cdot v = c$$

\begin{itemize}
\item $M$ are the principal parts coefficients of the basis elements
\item $v$ is a vector of weights on the basis elements
\item $c$ are the principal parts coefficients of the desired function
\end{itemize}

If $b$ is the vector of basis functions, then $v\cdot b$ will be the desired function.

\end{frame}

\begin{frame}[fragile]
\frametitle{Chebyshev's Integral (cont)}

\begin{sageblock}[chebyshev]
def principal_parts_matrix(div, basis):
    F = div.parent().function_field()
    coeffs = [(F.completion(p, prec=0), i) for p,m in div.list() for i in range(-m,0)]
    return matrix([[c[0](b)[c[1]] for c in coeffs] for b in basis]).transpose()

M = principal_parts_matrix(D, basis);

def solution_vector(div, differential):
    F = differential.parent().function_field()
    coeffs = [(F.completion(p, prec=0), i) for p,m in div.list() for i in range(-m,0)]
    return matrix([[c[0](differential)[c[1]-1]/c[1] for c in coeffs]]).transpose()

c = solution_vector(D, integrand);
\end{sageblock}

\bigskip
\tiny
\[ \sage[chebyshev]{M} \cdot v = \sage[chebyshev]{c} \]

\end{frame}

\begin{frame}[fragile]
\frametitle{Chebyshev's Integral (cont)}

\begin{sageblock}[chebyshev]
M.is_invertible()
pi = M.pseudoinverse()
v = pi * c
M * v == c
sol1 = (matrix(basis) * v)[0,0]
\end{sageblock}

\end{frame}

\begin{frame}[fragile]
\frametitle{Chebyshev's Integral (cont)}

\begin{sageblock}[chebyshev]
fns = [sol1, integrand];
table([[""] + fns] + [[pl] + [F.completion(pl, prec=0)(f) for f in fns] for pl in Dpoles.support()])
\end{sageblock}

\end{frame}

\begin{frame}[fragile]
\frametitle{Chebyshev's Integral (cont)}

\begin{sageblock}[chebyshev]
D2 = add([QQ(integrand.residue(pl)).sign() * pl for pl in Dpoles.support()])
D2.basis_function_space()
(2*D2).basis_function_space()
(3*D2).basis_function_space()
(4*D2).basis_function_space()
sol2 = (5*D2).basis_function_space()
\end{sageblock}

\end{frame}

\begin{frame}[fragile]
\frametitle{Chebyshev's Integral (cont)}

\begin{sageblock}[chebyshev]
sol1.differential() - (1/2)*(1/sol2[0])*sol2[0].differential()
integrand
\end{sageblock}

\end{frame}

\begin{frame}[fragile]
\frametitle{Chebyshev's Integral (cont)}

\begin{sagecode}[chebyshev]
num2 = R._to_bivariate_polynomial(sol2[0])[0];
den2 = R._to_bivariate_polynomial(sol2[0])[1];

ld = lcm([QQ(c).denominator() for c in den2.coefficients()]);
ln = lcm([QQ(c).denominator() for c in num2.coefficients()]);

R2.<x> = QQ[];
F2 = Frac(R2);
R3.<y> = F2[];
f = y^2 - (x^4 + 4*x^3 + 2*x^2 + 1);
Q = R3.quo(ideal(f));

R2.<x> = ZZ[];
F2 = Frac(R2);
R3.<y> = F2[];
f = y^2 - (x^4 + 4*x^3 + 2*x^2 + 1);
Q2 = R3.quo(ideal(f));

sol2a = Q2(Q(sol2[0].element()*(ln/ld)));
\end{sagecode}

\begin{sageblock}[chebyshev]
sol1a = R._to_bivariate_polynomial(sol1)
sol1b =sol1a[0].subs({L.0:sqrt(root)})/sol1a[1]

sol2a = R._to_bivariate_polynomial(sol2[0])
sol2b = sol2a[0].subs({L.0:sqrt(root)})/sol2a[1]

sol3 = sol1b - 1/2 * log(sol2b)

\end{sageblock}

\end{frame}

\end{document}
