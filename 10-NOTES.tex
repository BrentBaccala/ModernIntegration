
\mychapter{Notes}

\vfill\eject

\begin{maximablock}
diff(sqrt(x^4+1),x);
diff(%,x);
diff(%,x);
diff(%,x);

diff(sqrt(x^3+1),x);
diff(%,x);
diff(%,x);
\end{maximablock}

\vfill\eject


\mysection{Valuations}
\qquad [van der Waerden], \S18.1

A {\it valuation} is a generalization of the absolute value.  A {\it
valuation} is a mapping $\phi$ from a field ${\bf K}$ to an ordered
field ${\cal R}$ (typically the reals) obeying the following axioms:

\begin{center}
\begin{supertabular}{l l l r}
   positivity	& $\forall a \in {\bf K},$ & $\phi(a) \ge 0$ &(V1)\cr
   definiteness & $\forall a \in {\bf K},$ & $\phi(a) > 0 \Longleftrightarrow a \ne 0$ &(V2)\cr
   homomorphism (on the multiplicative group) & $\forall a,b \in {\bf K},$ & $\phi(ab) = \phi(a)\phi(b)$ &(V3)\cr
   subadditivity (or triangle inequality) & $\forall a,b \in {\bf K},$ & $\phi(a+b) \le \phi(a) + \phi(b)$ &(V4)\cr
\end{supertabular}
\end{center}

A moment's thought will show that the standard absolute value on the
reals obeys these axioms, as does the modulus on the complex field.
Valuations are similar to norms, except that norms are defined on
vector spaces, while valuations are defined on fields.

A valuation is said to be {\it non-Archimedian} if it also satisfies
the following axiom, stronger than V4:

\begin{center}
\begin{supertabular}{l l l r}
   non-Archimedian axiom & $\forall a,b \in {\bf K},$ & $\phi(a+b) \le \max(\phi(a), \phi(b))$ &(V4')\cr
\end{supertabular}
\end{center}

In this case, we can switch from a multiplicative to an additive
notation and obtain {\it exponential valuation} by replacing $\phi(a)$
with $w(a) = -\ln \phi(a)$:

\begin{center}
\begin{supertabular}{l l l r}
   & $\forall a \in {\bf K},$ & $w(a) \in (-\infty, \infty]$ &(E1)\cr
   & $\forall a \in {\bf K},$ & $w(a) = \infty \Longleftrightarrow a = 0$ &(E2)\cr
   & $\forall a,b \in {\bf K},$ & $w(ab) = w(a) + w(b)$ &(E3)\cr
   & $\forall a,b \in {\bf K},$ & $w(a+b) \ge \min(w(a), w(b))$ &(E4)\cr
\end{supertabular}
\end{center}

\vfill\eject
\mysection{Notes on [Fu08]}

[Fu08] is a good, freely available introduction to algebraic geometry.

{\small\begin{verbatim}


Riemann-Roch Theorem

Let C be an algebraic curve, let X be its non-singular model, and let
K be its function field.

Proposition 8.4.  Let x \in K, x \notin k. Let (x)_0 be the divisor
of zeros of x and let n=[K:k(x)].  Then

  1) (x)_0 is an effective divisor of degree n,
  2) There is a constant \tau such that l(r(x)_0) \ge rn-\tau \forall r.

Proof

Prop 6.9. K is an algebraic function field in one variable over k.  By
definition, this means that exists some t such that K is algebraic
over k(t).  So x \in K is algebraic over k(t), and \exists F \ in
k[X,T] such that F[x,t] = 0.  x is not algebraic over k (1-48), so t
must appear in F, so t is algebraic over k(x), and therefore k(x,t) is
algebraic over k(x) (1-50), so K is algebraic over k(x) (1-46).



Problem 1-54: If R is a domain with quotient field K, and L is a
finite algebraic extension of K, then there exists a basis for L over
K such that each basis element is integral over R.

Proof 1-54: Let {w_1, ..., w_n} be any basis for L over K.  Since each
basis element is algebraic over K, by clearing denominators we can
write:

a_{i0} w_i^{n_i} + a_{i1} w_{n_i-1} + \cdots = 0       a_{ij} \in R

We can pull a_{i0} into w_i, and thus adjust the w_i's to be integral
over R by multiplying each one by something in R.  Since anything
in L can be written

   l = \sum c_i w_i        c_i \in K

it can also be written

   l = \sum (c_i / r_i) w'_i

where the r_i adjust the w_i to be integral and c_i/r_i is still in K.



INTEGRAL ELEMENTS: w integral over k[x] means that w is finite
everywhere x is, and has poles only where x does.  w integral over
k[x^{-1}] means that w is finite everywhere x^{-1} is, and has poles
only where x has zeros.

So, k[x^{-1}] is a domain with quotient field k(x), and K is a finite
algebraic extension of k(x), so there exists a basis for K over k(x)
such that each basis element is integral over k[x^{-1}].

Let {w_1, ..., w_n} be such a basis for K over k(x).  We will show
that the poles of these functions must lie over the roots of x.

w_i^{n_i} + a_{i1} w_{n_i-1} + \cdots = 0       a_{ij} \in k[x^{-1}]

So, ord_P(a_ij) \ge 0 if P \ne S (zero set of x), since x^{-1} and
thus anything in k[x^{-1}] is finite away from S.



Problem 2-29: if for some i, ord(a_i) < ord(a_j) \forall j \ne i,
then a_1 + \cdots + a_n \ne 0

Proof of 2-29: Assume the contrary.  Then we can write a_i =
\sum_{i\ne j} a_j.  Taking ord of both sides, and using ord(a+b) \ge
min(ord(a), ord(b)), we see this is impossible.



Therefore, ord_P(w_i) \ge 0 if P \ne S, since otherwise
ord_P(w_i^{n_i}) < ord_P(a_ij w^{n_i-j}) \forall j.

Therefore, the poles of w_i are isolated at the zeros of x, and since
there are only a finite number of w_i and each has a finite number of
poles, then for some t, div(w_i) + tZ > 0 \forall i.

So w_i \in L(tZ), and if j \le r, then w_i x^{-j} \in L((r+t)Z)

Now w_i are independent over k(x) and 1, x^{-1}, ..., x^{-r} are
independent over k, so l((r+t)Z) \ge n(r+1).

Now, l((r+t)Z) = l(rZ) + dim(L((r+t)Z) / L(rZ))

dim(L((r+t)Z) / L(rZ)) \le tm (Prop 3-1), where m is the degree of Z.

So, l(rZ) \ge n(r+1) - tm, so pick \tau = tm-n, and

l(rZ) \ge nr - \tau, \forall r




Riemman-Roch

l(D) = deg(D) + 1 - g + l(W-D)

deg(W) = 2g-2    l(W) = g

l(0) = 1

l(D) = 0 if deg(D) < 0

If deg(D) > 2g-2, then l(D) = deg(D) + 1 - g

If deg(D) = 2g-2, then deg(W-D) = 0, and l(W-D) = 1 iff D-W is principal, otherwise l(W-D) = 0

Let D=W+X, where deg(X)=0, then l(D) = 2g-2 + 1 - g + (1/0) depending on whether X is principal
   l(D) = g (X is principal) or l(D) = g - 1 (X is not principal)


Goal: an g-dimensional algebraic variety that represents Pic0

Consider (2g-2)-dimensional symmetric space.  Each point corresponds to an effective
divisor of degree (2g-2).

Fix a (g-2)-tuple.  We're left with g free points.


Milne's construction

Use r-dimensional symmetric space, with r > 2g-2.  Pick an (r-g) tuple.

\end{verbatim}
}
