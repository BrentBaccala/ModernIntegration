
\setcounter{chapter}{8}
\chapter{Simple Algebraic Extensions}


\section{Integral Elements}

\definition

A field's {\bf local ring} at a place $p$ is the set of
all functions in the field with no pole at $p$; that
is, $ \lim_{x\to p} f(x) \ne \infty $.

A function is {\bf locally integral} at a place $p$
if it belongs to its field's local ring at $p$.

\enddefinition

Remember the distinction I drew between a {\it pole} and a {\it
removable singularity} is Chapter ?.  That's the reason for the limit
in the definition; removable singularities are specifically included
in a local ring.  Thus, $x\over\sqrt{x}$ is locally integral at $x=0$,
despite the fact that it has a zero denominator, because
$\lim_{x\to0}{x\over\sqrt{x}}=\sqrt{x}=0$, a finite value.

It is often important, however, to specify which field we are
refering to when we discuss a local ring.

\example \quad

In the field ${\cal C}(x)$, the function $x$ is locally integral at $x=0$.

In the field $F = {{\cal C}(x,y); y^2=x+1}$, the function $x$ is locally integral
at both $(x=0, y=1)$ and $(x=0, y=-1)$.  There is no place $x=0$ in
this field; to speak of it so is ambiguous, as there are two places
in $F$ where $x=0$.

Likewise, $y$ is locally integral in $F$ at both $(x=0, y=1)$ and
$(x=0, y=-1)$.  It is not locally integral in ${\cal C}(x)$ at $x=0$ simply
because $y$ does not exist in the field ${\cal C}(x)$.

\endexample

Thus, a function like $x$ can be locally integral in both ${\cal C}(x)$ and
${\cal C}(x,y)$; it is important to specify which we are talking about.

Since the fields ${\cal C}(x)$ and ${\cal C}(x,y)$ are distinct, and have different
places, they also have different local rings.  There is an important
connection between them, however.

\theorem
\label{local integral polynomial}

A function $f$ is locally integral at a place $p$ in ${\cal C}(x,y)$ if it
satisfies a monic polynomial with coefficients in the local ring of
${\cal C}(x)$ at the corresponding place in that field.

\proof

Define $z = {1 \over f}$ and consider what would happen if $f$ had a
pole at $p$.  Then $\lim_{(x,y)\to p} f(x) = \infty$, so $z(x,y) = 0$
(we are justified in discarding the limit by the principle of isolated
singularities).  Yet $f$ satisfies a monic polynomial with coefficients
in ${\cal C}(x)$'s local ring under $p$:

	$$f^n + a_{n-1} f^{n-1} + \cdots + a_1 f + a_0 = 0$$

	$$z^{-n} + a_{n-1} z^{-n+1} + \cdots + a_1 z^{-1} + a_0 = 0$$

	$$1 + a_{n-1} z + \cdots + a_1 z^{n-1} + a_0 z^n = 0$$

Now, since $z$ is zero at a place over $p$, at least one of this
polynomial's roots must be zero.  Since all of the $a_i$ are finite at
$p$ (they are in ${\cal C}(x)$'s local ring there), multiplying
any of them by zero yields zero, so substituting in
$z=0$ yields $1=0$. We conclude that $z$ can not be zero, and thus
$f$ can not have a pole over $p$.

\endtheorem

The converse is not true; just because a function does not satisfy
such a monic polynomial does not mean that it is not locally integral.
A slightly weaker converse does hold, however:

\theorem

A function $f$ in ${\cal C}(x,y)$ which is locally integral at all
places in ${\cal C}(x,y)$ over a place $p$ in ${\cal C}(x)$ satisfies
a monic polynomial with coefficients in the local ring of ${\cal C}(x)$
under $p$.

\proof

Construct the polynomial

$$(z - f(x,y))(z - f(x,y_2)) \cdots (z - f(x,y_n)) = 0$$

$f$ clearly satisfies this polynomial at every place over $p$.
Furthermore, multiplying out the terms yields:

$$\matrix{z^n & - (f(x,y)+f(x,y_2)+\cdots+f(x,y_n)) z^{n-1} \hfill\cr
   & + \quad (f(x,y)f(x,y_2)+f(x,y)f(x,y_3)+\cdots+f(x,y_{n-1})f(x,y_n)) z^{n-2} \hfill\cr
   & + \cdots + f(x,y)f(x,y_2)\cdots f(x,y_n) = 0 \hfill\cr}$$

All of these coefficients are symmetric functions in $y_i$, and thus
exist in ${\cal C}(x)$.  Furthermore, since $f$ is finite (by
assumption) at all conjugate values over $p$, each of these
coefficients must also be finite at $p$ (since they are constructed by
addition and multiplication from the conjugate values of $f$), and are
thus in the local ring of ${\cal C}(x)$ at $p$.

\endtheorem

In short, a function in ${\cal C}(x,y)$ which satisfies a monic
polynomial whose coefficients are in a local ring of ${\cal C}(x)$ at
a point $p$ is locally integral in ${\cal C}(x,y)$ at all places over
$p$.  A function which does not satisfy such a polynomial has
a pole at at least one place over $p$.

A more advanced, more purely algebraic approach to this subject would
{\it define} local rings using Theorem \ref{local integral
polynomial}, then use valuations and completions to {\it prove} the
connection between local rings and poles (or the lack thereof).  See
[van der Waerden], Chapter 18.  I will forego this in favor of this
simpler, more analytic (note the use of the limit) approach.


Having established the existence and basic properties of {\it locally}
integral elements, we can now discuss {\it globally} integral
elements.  Obviously, elements in a field will be locally integral at
more than one place.  In fact, field elements will be locally integral
at almost all places (because they only have a finite number of
poles).  So, intuitively, a {\it globally} integral element would be
one that is integral at every place in a field.  However, Theorem ?
immediately tell us that the only such functions (those which have no
poles anywhere) are the constants, so this turns out to be too
restrictive to be useful.  Instead, we exclude infinity and define:

\definition

A function is {\bf globally integral} in a field if it is locally
integral at all {\it finite} places in the field.

\enddefinition

What's special about infinity?  Why not exclude some other place?
Well, nothing's all that special about infinity.  We've already seen
how a birational transformation can be used to swap infinity with any
finite point.  We use infinity because it's convenient.  For example,

\theorem

The globally integral functions in ${\cal C}(x)$ are the polynomials.

\proof

The elements of ${\cal C}(x)$ are fractions of polynomials with no
common factors between numerator and denominator.  Since {\cal C} is
algebraically closed, both numerator and denominator factor into
linear terms with no cancelation between them.  To avoid any finite
poles, the denominator must be a constant.  Since polynomials
have no finite poles, there is no restriction on the numerator.

\endtheorem

Pretty straightforward, huh?

While it's obvious from inspection which functions are globally
integral in ${\cal C}(x)$, the situation in ${\cal C}(x,y)$ is more
difficult.  Something like $y \over x$, which appears to have a pole
at $x=0$, is actually globally integral if, for example, $y^2=x^3$.
Then we can consider squaring $y \over x$ to obtain ${y^2 \over x^2} =
{x^3 \over x^2} = x$.  If the square is finite, then the original
function had to be finite (you can't square infinity and get a
finite value), so we conclude that $y \over x$ is, in fact,
globally integral in ${\cal C}(x,y); y^2=x^3$.

We could use Theorem \ref{local integral polynomial} and the fact
that, in ${\cal C}(x,y); y^2=x^3$,

	$$ \left({y \over x}\right)^2 - x = 0 $$

to conclude that, since this is a monic polynomial with coefficients
globally integral in ${\cal C}(x)$, $y \over x$ must be globally
integral in ${\cal C}(x,y); y^2=x^3$.  Theorem \ref{local integral
polynomial} can be easily generalized:

\theorem

A function $f$ is globally integral in ${\cal C}(x,y)$ if it
satisfies a monic polynomial with globally integral coefficients in
${\cal C}(x)$.

\proof

\endtheorem

Proof that globally integral elements form a finite extension:
[Eichler], p. 54

Proof that field elements are determined by their divisors, up to a
constant multiple: [Eicher], p. 84

[Eicher], p. 86: Prime divisors are isomorphic with special local rings.

The problem is that we have no straightforward means to construct
such a polynomial, or prove that one doesn't exist, for any
particular function $f$.  To test a function to determine
if it is integral, we'll use a different approach.

\section{Modules}

We'll resort now to {\it modules}, a fairly important algebra concept
backed by a substantial body of theory, upon which I shall only draw
as needed.  General references included [Atiyah+McDonald], [Lang], and
about a thousand others.

\definition

An {\it R-module} over a ring R is an additive group M acted on by R
(i.e, there is a mapping $R \times M \to M$) in a distributive manner:

$$(r_1 + r_2)m = r_1 m + r_2 m \qquad r_1, r_2 \in {\rm R};\, m \in {\rm M}$$

where we have adopted the usual convention of writing R's action on M
as a multiplication.

\enddefinition

\definition

A {\it free R-module} is an R-module spanned by a linearly independent basis
$\{b_1, b_2, ... b_n\}$.  It consists of all elements formed as follows:
\footnote{I'll also note that a multiplication rule needs to be specified
between the basis elements and the elements of the ring, and an
addition rule between the elements of the module.  Also,
the expression has to be {\it unique} --- you can't be
able to write an element two different ways.  In our
case, these rules are obvious, but that's not always the case.}

	$$ a_1 b_1 + a_2 b_2 + ... + a_n b_n; \qquad a_i \in R $$


\enddefinition

Not all modules have a finite set of generators, and not all those
have a linearly independent set of generators.  Elements formed from a
basis can be added by using the module's distributive property (!!) to
factor out the coefficients from of each basis element and then
performing the addition in the ring R:

$$ (a_1 b_1 + a_2 b_2 + ... + a_n b_n) + (c_1 b_1 + c_2 b_2 + ... + c_n b_n) $$
$$  = (a_1 + c_1) b_1 + (a_2 + c_2) b_2 + ... + (a_n + c_n) b_n $$

So the elements generated from a basis clearly form a module.  R operates
on them by multiplication by every coefficient.

\example \quad
\label{sample modules}

An ideal I in a ring R is a R-module, but a subring S of R, in
general, is not, because multiplication by an element of R might not
produce a result in the subring.  R, however, can always be viewed as
an S-module.

% XXX
% If R is a principal ideal ring, then every ideal is also a
% free R-module, admitting a basis consisting of a single
% element, a generator of the ideal.
%
% Is this true?  Any zero divisor counterexamples?

\endexample


Note that it is vitally important to specify the ring used for the
coefficients.  For example, consider the basis $\{1, y\}$.  Treating
this as a ${\cal C}(x)$-module, I can form ${y \over x} = {1 \over x}
y$, since ${1 \over x} \in {\cal C}(x)$.  However, $y \over x$ does {\it
not} belong to the ${\cal C}[x]$-module generated by $\{1, y\}$.  I
would need to use polynomial coefficients to form a ${\cal
C}[x]$-module, not the rational functions coefficients allowed in a
${\cal C}(x)$-module.  We'll be primarily interested in ${\cal C}[x]$-modules,
${\cal C}(x)$-modules, and ${\cal I}$-modules, where ${\cal I}$ is the ring
of globally integral elements in ${\cal C}(x,y)$.



\section{Basis for all Rational Functions}

The first kind of basis we're interested in, a {\it basis for all
rational functions}, is one than spans the entire ${\cal C}(x,y)$ field
as a ${\cal C}(x)$-module.
In other words, we're looking for a basis $\{b_1, b_2,
... b_n\}$ so that everything in ${\cal C}(x,y)$ can be expressed
in the form:

	$$ a_1 b_1 + a_2 b_2 + ... + a_n b_n; a_i \in {\cal C}(x) $$

Such a basis will always have $n$ elements, where $n$ is the degree of
the ${\cal C}(x,y)$ extension over ${\cal C}(x)$, and can be most
conveniently characterized using its {\it conjugate matrix}:

\definition

The {\bf conjugates} of a rational function $\eta(x,y)$ in ${\bf
C}(x,y)$ are the functions formed by replacing $y$ with its conjugate
values.

The {\bf trace} of a rational function $\eta(x,y)$ is the sum of
its conjugates:

$${\rm T}(\eta(x,y)) = \sum_i \eta(x,y_i)$$

The {\bf norm} of a rational function $\eta(x,y)$ is the product of
its conjugates:

$${\rm N}(\eta(x,y)) = \prod_i \eta(x,y_i)$$

Both the trace and norm, as symmetric functions in $y_1,...,y_n$, are
functions in ${\bf C}(x)$.

The {\bf conjugate matrix} ${\bf M}_\omega$ of $n$ elements $\omega_i$
in ${\cal C}(x,y)$, where $n$ is the degree of ${\cal C}(x,y)$ over
${\cal C}(x)$, is the matrix whose each row consists of the $n$
conjugate values of a single element, and whose $n$ rows are formed in
this way from the $n$ elements.

A set of $n$ elements $\omega_i \in {\bf C}(x,y)$ form a {\bf rational
function basis} for ${\bf C}(x,y)$ if the determinant of their
conjugate matrix is non-zero: $|{\bf M}_\omega| \ne 0$

\enddefinition

\definition

For any function $\eta \in {\bf C}(x,y)$ and any rational function
basis $\omega_i$, the {\bf trace vector}
${\bf T}_{\eta/\omega} = \Big({\rm T}(\eta \omega_i)\Big)$ 
of $\eta$ relative to $\omega$
is formed from the
traces of the $n$ products of $\eta$ with the $n$ functions
$\omega_i$.

The {\bf conjugate vector} ${\bf C}_\eta = (\eta(x,y_i))$ is formed from the
$n$ conjugates of $\eta$.

\enddefinition

\theorem
\label{function is zero if trace vector is zero}

For any function $\eta \in {\bf C}(x,y)$ and any rational function
basis $\omega_i$, if ${\bf T}_{\eta/\omega}$ is the zero vector,
then $\eta$ is zero.

\proof

${\bf T}_{\eta/\omega}$, ${\bf M}_\omega$ and ${\bf C}_\eta$
satisfy the matrix equation

$${\bf T}_{\eta/\omega} = {\bf M}_\omega {\bf C}_\eta$$

since each row of this matrix equation has the form

$$ {\rm T}(\eta \omega_i) = \sum_j \omega_i(x,y_j)\eta(x,y_j) $$

Since ${\bf M}_\omega$ is invertible (since its determinant is
non-zero), if ${\bf T}_{\eta/\omega}$ is identically zero, then so must be
${\bf C}_\eta$, and $\eta$ is the first element in ${\bf C}_\eta$.

\endtheorem

\theorem
\label{|M| != 0 implies C(x) basis}

A rational function basis $\omega_i$ spans ${\bf C}(x,y)$ as
a ${\bf C}(x)$-module. ([Bliss], Theorem 19.1)

\proof

Note that when we multiply ${\bf M}_\omega$ by its transpose ${\bf
M}_\omega^T$, the $ij^{\rm th}$ element of ${\bf M}_\omega{\bf
M}_\omega^T$ is:

$$ \sum_k \omega_i(x, y_k)\omega_j(x, y_k) = {\rm T}(\omega_i \omega_j)$$

Since $|{\bf M}_\omega|$ is non-zero, $|{\bf M}_\omega^T|$ is
non-zero, and $|{\bf M}_\omega{\bf M}_\omega^T|$ is non-zero, so given
any function $\eta \in {\bf C}(x,y)$, we can solve the following
equation for ${\bf R}$:

$${\bf T}_\eta = {\bf M}_\omega {\bf M}_\omega^T {\bf R}$$

each of row of which reads:

$$ {\rm T}(\eta \omega_i) = \sum_j {\rm T}(\omega_i \omega_j) r_j $$

Since both ${\bf T}_\eta$ and ${\bf M}_\omega{\bf M}_\omega^T$ are composed of
nothing but traces, they exist in ${\bf C}(x)$, so ${\bf R}$ must also
exist in ${\bf C}(x)$ and its elements therefore commute with the
trace:

$$ {\rm T}(\eta \omega_i) = \sum_j {\rm T}(r_j \omega_j \omega_i) $$

Since the trace of a sum is the sum of the traces:

$$ {\rm T}(\eta \omega_i) = {\rm T}(\sum_j r_j \omega_j \omega_i) $$
$$ {\rm T}((\eta - \sum_j r_j \omega_j) \omega_i) = 0 $$

which implies that $\eta = \sum_j r_j \omega_j$, by Theorem
\ref{function is zero if trace vector is zero}, and since we've
already shown that the $r_j$ are rational functions in ${\bf C}(x)$,
this proves the theorem.

\endtheorem

Let me illustrate with a simple example.

\example

Consider the basis $\{1, y\}$ over the field ${\cal C}(x,y); y^2=x$.
The conjugate value of $y$ is $-y$ (PROVE THIS), so the conjugate
matrix is:

$$C=\left(\begin{matrix}1&1\cr y&-y\cr\end{matrix}\right)$$

and its determinant:

$$\det C=\left|\begin{matrix}1&1\cr y&-y\cr\end{matrix}\right| = -2y$$

Since $-2y$ is not zero, we conclude that $\{1, y\}$ is a basis
for all rational functions over ${\cal C}(x,y); y^2=x$.

\endexample

Notice that I didn't ask whether $-2y$ was zero at some place in the
field.  The determinant of the conjugate matrix can be zero at certain
places; in fact, often is.  It just can't be {\it identically} zero;
i.e, it can't be zero {\it everywhere}.  If this isn't clear, reread
Theorems \ref{function is zero if trace vector is zero} and \ref{|M|
!= 0 implies C(x) basis}, noting that all the matrices are defined
over the {\it fields} ${\bf C}(x)$ and ${\bf C}(x,y)$, where the only
zero element is 0.



\vfil\eject

Since polynomials have no finite poles, they are integral elements,
and thus ${\rm K}[x] \subseteq {\cal I}$.  Thus, ${\cal I}$ (the ring of
integral elements) is trivially a ${\rm K}[x]$-module (see Example
\ref{sample modules}), but what is not nearly so obvious is that it is
also a free module, a fact which underlies a great deal of our theory.
I'll prove this first by showing that ${\cal I}$ is finitely generated
as a ${\rm K}[x]$-module, then showing the existance of a linearly
independent set of generators.

Let's start with a preliminary theorem.

\theorem
\label{construction of dual basis}

If $\{w_1,\ldots,w_n\}$ is a basis for a finite separable field
extension $E/K$, then a dual basis $\{u_1,\ldots,u_n\}$ can be
constructed such that ${\rm Tr}(w_i u_j) = \delta_{ij}$.
([Lang] Corollary VI.5.3)

\proof

Consider the following matrix:

$$M = \pmatrix{{\rm Tr}(w_1 w_1) & & \cr \vdots & \ddots & \cr {\rm Tr}(w_1 w_n) & \cdots & {\rm Tr}(w_n w_n)}$$

Now take an element $x \in E$, and represent it relative to the basis
$\{w_1,\ldots,w_n\}$ as a row vector $X = (x_i)$.  Multiplying $X M$
produces a row vector whose $j^{\rm th}$ element can be written:

$$\sum_i x_i {\rm Tr}(w_j w_i) = {\rm Tr}(w_j \sum_i x_i w_i) = {\rm Tr}(w_j x) = {\rm Tr}_x(w_j)$$

where I used first the $K$-linearity and additive distributive
properties of ${\rm Tr}$, then wrote ${\rm Tr}_x: f(a) = {\rm Tr}(ax)$
to emphasize that I'm regarding ${\rm Tr}_x$ as a linear form in ${\rm
Hom}_E(E,K)$.  So, if $M$ is singular, then there exists some non-zero
element $x$ such that ${\rm Tr}_x$ is zero for all of $w_i$, which
form a basis set, so ${\rm Tr}_x$ must therefore be the zero map.
This can only happen if ${\rm Tr}$ is identically zero, which would be
the case for an inseparable extension.  For the separable case,
therefore, $M$ must be invertible, and we can write:

$$M^{-1} M = \pmatrix{1 & & \cr & \ddots & \cr & & 1}$$

A moment's thought now shows that the rows of $M^{-1}$ are the desired
dual basis elements, written with respect to $\{w_1,\ldots,w_n\}$.

\endtheorem

\vfil\eject

\theorem
\label{I is finitely generated}

${\cal I}$ is a finitely generated ${\rm K}$-module.
([A+MacD] Proposition 5.17; [Lang] Exercise VII.3)

\proof

Regarding ${\rm K}(x,y)$ as a vector space over ${\rm K}(x)$, we can
easily construct a basis of integral elements by starting with $\{1,
y, \ldots, y^{n-1}\}$ and multiplying each element (if needed) by a
polynomial in $x$ which cancels all of its poles:

$${\rm K}(x,y) = {\rm K}(x)\{w_1,\ldots,w_n\} \qquad \forall i(w_i \in {\cal I})$$

Using Theorem \ref{construction of dual basis}, construct a dual basis
$\{u_1,\ldots,u_n\}$ so that ${\rm Tr}(w_i u_j) = \delta_{ij}$.  Take
any $x \in {\cal I}$ and write it using the dual basis:

$$x = \sum_i a_i u_i \qquad a_i \in {\rm K}(x)$$

Now consider ${\rm Tr}(x w_j)$.  Now, $x$ and $w_j$ are both in ${\cal
I}$, so $x w_j$ is in ${\cal I}$, and therefore has a monic minimum
polynomial with coefficients in ${\rm K}[x]$.  Since ${\rm Tr}$ equals
some integer multiple of the negative of the second coefficient in a
monic minimum polynomial, ${\rm Tr}(x w_j) \in {\rm K}[x]$.  But also,

$${\rm Tr}(x w_j) = {\rm Tr}(\sum_i a_i u_i w_j)
 = \sum_i {\rm Tr}(a_i u_i w_j) %\hfil (Tr an additive homomorphism)
 = \sum_i a_i {\rm Tr}(u_i w_j) %\hfil (linearity of Tr)
 = a_i$$

which establishes that $\forall i (a_i \in {\rm K}[x])$, so

$${\cal I} \subseteq {\rm K}[x]\{u_1,\ldots,u_n\} $$

${\rm K}[x]$ is a Noetherian ring (Theorem ??), so ${\rm
K}[x]\{u_1,\ldots,u_n\}$ is a Noetherian module (Theorem ??),
which means that ${\cal I}$, as a submodule, is Noetherian
and thus finitely generated (Theorem ??).

\endtheorem

\vfil\eject

\theorem
\label{submodules of free modules over PIRs are free}

Any submodule of a finite free module over a principal ideal ring is free.
([Lang] Theorem III.7.1)

\proof

Let $F = R\{w_1,\ldots,w_n\}$ be a free $R$-module ($R$ a principal
ideal ring) with a submodule $M$.  Consider $F_i =
R\{w_1,\ldots,w_i\}$, the free $R$-module generated by the first $i$
basis elements, and $M_i = F_i \cap M$.  We will show inductively that
all of the $M_i$ are free $R$-modules, and since $F_i = F$ and $M_i =
M$, this will prove the theorem.

First, consider $M_1 = R\{w_1\} \cap M$.  If $M_1$ is not empty (and
thus free), then any $m \in M_1$ can be written $r w_1$.  Since a
module forms an additive group, and we can operate on the module using
all the elements of $R$, it follows that all the $r$'s must form an
ideal, and since $R$ is principal, that ideal can be written with a
single generator, say $(r_1)$, and $M_1 = R\{r_1 w_1\}$ (or is empty).

Now, assume that $M_j$ is free for all $j<i$.  Consider all $x \in
M_i$, which can be written $r_1 w_1 + \cdots + r_i w_i$.  Either $M_i
= M_{i-1}$ (and is therefore free), or at least some of the $r_i$ are
non-zero.  By the same rationale as the last paragraph, these $r_i$
form an ideal, which can be written $(r_i)$.  Take any element $x \in
M_i$ with its $i^{\rm th}$ coefficient $r_i$ and add it to
$M_{i-1}$'s basis set to form a basis set for $M_i$, since some
multiple of this element can be used to cancel any $i^{\rm th}$
coefficient from an element in $M_i$ and leave an element in $M_{i-1}$
which can be formed using the remaining basis elements.

\endtheorem

A {\it torsion-free} module has no ``zero divisors'', in the sense
that no non-zero element of its associated ring can operate on a
non-zero element of the module and produce zero.  Since fields are
torsion-free, and all of our modules are subsets of the field
$K(x,y)$, they are all torsion-free.

\theorem
\label{finitely generated torsion-free modules over a PIR are free}

Any finitely generated, torsion-free module $M$ over a principal ideal
ring $R$ is free. ([Lang] Theorem III.7.3)

\proof

Take a maximal set of $M$'s linearly independent generators $\{w_1,
\ldots, w_n\}$ and any remaining generators $\{y_{n+1}, \ldots,
y_m\}$.  Every $y_i$ is therefore linearly dependant on $\{w_1,
\ldots, w_n\}$:

$$a_i y_i + r_1 w_1 + \cdots + r_n w_n = 0 \qquad a_i \ne 0$$

Take the product of all the $a_i$'s: $a = a_{n+1} a_{n+2} \cdots a_m$
and consider the mapping $x \mapsto ax$ which is injective, since $a
\ne 0$ and the module is torsion-free, so therefore maps $M$ to $aM$,
an isomorphic image which is a submodule of the free module
$R\{w_1,\ldots,w_n\}$.  By Theorem \ref{submodules of free modules
over PIRs are free}, $aM$ is therefore free, and since it is
isomorphic to $M$, we can take a basis of $aM$, divide all of its
basis elements by $a$ (they are all multiples of $a$), and obtain
a basis for $M$.

\endtheorem

Now, since $K[x]$ is a principal ideal ring (Theorem ??), Theorems
\ref{I is finitely generated} and \ref{finitely generated torsion-free
modules over a PIR are free} demonstrate that ${\cal I}$
is a free $K[x]$-module.

\definition

A basis for ${\cal I}$ will be called an {\bf integral basis}.

\enddefinition

\theorem

Any integral basis is also a basis for the $K(x,y)$ field as a
$K(x)$-module.

\endtheorem

While the preceding theorems offer an existance proof for an integral
basis, it is not immediately clear how to obtain one for any
particular field, and in fact the calculation of an integral basis
ultimately becomes one of the biggest computational barriers in this
theory.  Therefore, I will defer a more detailed discussion until a
later chapter, and instead present a simple construction for the
special case of a simple radical extension.

\definition

A basis is said to be a {\bf locally integral basis} at a place $p$ in
${\cal C}(x)$ if the basis elements are locally integral at all places
in ${\cal C}(x,y)$ over $p$ and the determinant of its conjugate
matrix assumes a finite, non-zero value at that place.

\enddefinition

\theorem

A function $f$ in ${\cal C}(x,y)$ has no poles over a place $p$ in
${\cal C}(x)$ iff $f$ can be formed from a basis locally integral at
$p$, using elements from ${\cal C}(x)$ locally integral at $p$.

\endtheorem


With a local integral basis (no pole at a place), there are 1-to-1
correspondances:

	use coeffs in K(x) w/ no pole at place <=> elem in K(x,y) w/ no pole

	use coeff w/ pole at place <=> elem in K(x,y) w/ pole at place


\section{Integral Modules}

In ${\bf C}(x)$, we were working with the quotient field of a
principal ideal ring, so we could always find a single function to
generate any ideal / ${\bf C}[x]$-module.

In ${\bf C}(x,y)$, we are no longer working with a principal ideal
ring, so we can't guarantee that any particular ideal can be generated
by a single function, but it turns out that every ideal can be
generated by a {\it pair} of functions.  Our course of attack is first
to construct that pair of functions, then use them to determine if in
fact the ideal is principal.

DEFINITION

An {\it integral module} (or ${\cal I}$-module) is a module formed
over ${\cal I}$, the ring of integral elements in ${\bf C}(x,y)$.

Since ${\cal I}$ itself can be expressed as a ${\bf C}[x]$-module
using an integral basis, any ${\cal I}$-module is also a ${\bf
C}[x]$-module.  Not all ${\bf C}[x]$-modules are ${\cal I}$-modules,
however, since ${\cal I}$ is typically larger than ${\bf C}[x]$.

Some authors use the term {\it fractional ideal} to refer to an ${\cal
I}$-module.  I have avoided use of this term for two reasons.  First,
I wish to emphasize the concept of a module.  Second, ${\cal
I}$-modules are not ideals, either in the ring ${\cal I}$ (since they
may contain elements not in ${\cal I}$), nor in the field ${\bf
C}(x,y)$, since, as a field, ${\bf C}(x,y)$ has only the trivial
ideals.  The term {\it fractional ideal} is used because an ${\cal
I}$-module can be regarded as a fraction of ideals in ${\cal I}$.

THEOREM C

A function can always be constructed with a simple zero at a specified
place $(\alpha, \beta)$, $\alpha$ and $\beta$ both finite, zero order
at an additional finite set of places $\Sigma$, and non-negative order
at all other finite places.

PROOF

Begin with the function $(y-\beta)$, which has a simple zero at
$(\alpha, \beta)$ and non-negative order at all finite places.  If
none of the other places in $\Sigma$ have y-value $\beta$, then we are
done, since this function's zeros lie at exactly those places where $y
= \beta$.

Otherwise, compute $(y-\beta)\over(x-\alpha)$ at all places in $\Sigma$
that do {\it not} lie over $x = \alpha$.  Select a number
$\gamma$ different from all of these values.  The function
$(y-\beta) - \gamma (x-\alpha)$ has the desired properties, since
it has non-negative order at all finite places and zero order at
all places in $\Sigma$.

END THEOREM

THEOREM D

A function can always be constructed with a simple pole at a specified
place $(\alpha, \beta)$, $\alpha$ and $\beta$ both finite, zero order
at an additional finite set of places $\Sigma$, and non-negative order
at all other places.

PROOF

If $(\alpha, \beta)$ is non-singular, begin with the function:

$$f(\alpha,y)\over(x-\alpha)(y-\beta)$$

where $f(x,y)=0$ is the defining polynomial of the algebraic extension.
Note that the division by $(y-\beta)$ will always be exact, since
$f(\alpha, \beta)=0$.  We now have a rational function
$P(y)\over Q(x)$, where $P(y)$ is a polynomial in $y$ and $Q(x)$ is a
polynomial in $x$ ($x-\alpha$, to be precise).  It has a simple pole at
$(\alpha, \beta)$ and non-negative order at all other places, which
is obvious except for places over $x=\alpha$.

If $x=\alpha$ and $y\ne\beta$, $f$ takes the form $0\over0$, so we can
expand it using L'H\^opital's rule:

$$\lim_{(x,y)\to p} {{P(y)}\over{Q(x)}}
  = \lim_{{(x,y)\to p}} {{{dP(y)}\over{dx}}\over{{dQ(x)}\over{dx}}}
  = {{P'(y)}\over{Q'(x)}} \, {{dy}\over{dx}} $$

where $'$ denotes differentiation with respect to the polynomial's
variable.  Now, $Q(x)=(x-\alpha)$, so $Q'(x)=1$, and the denominator
is thus finite.  $P'(y)$ is a polynomial, and is thus also finite.
$(\alpha, \beta)$ is a finite place, so ${dy}\over{dx}$ is also
finite.  Since all the numerator terms are finite and the denominator
is non-zero, it follows that $f$ is at least finite, and thus has
non-negative order, at places over $x=\alpha$ other than $(\alpha,
\beta)$.

If $x=\alpha$ and $y=\beta$, then $f$'s numerator is non-zero and its
denominator is zero, so this is clearly a pole, but of what order?

Now, compute the value of the function at all other places in
$\Sigma$, using either L'H\^opital's rule or Puiseux expansion if some
of these are other places over $\alpha$.  If the value of the function
is non-zero at all of these places, then we are done.  Otherwise,
select a number $\gamma$ different from all of these values.  The
function:

$${f(\alpha,y)\over(x-\alpha)(y-\beta)} - \gamma$$

has the desired properties, since it still has a simple pole at
$(\alpha,\beta)$, and is now non-zero at all places in $\Sigma$.

SINGULARITIES?

END THEOREM

THEOREM E

A function can always be constructed with specified integer orders at
a finite set of places $\Sigma$ and non-negative order at all other finite
places.

PROOF

For each pole or zero, use Theorems C or D to construct a function
with a simple pole or a simple zero at that place, zero order at all
other places in $\Sigma$ and non-negative order at all other finite
places.  Raise each of these function to the integer power that is the
order of the corresponding pole or zero, then multiply them all
together.

END THEOREM

THEOREM A

There is a one-to-one relationship between integral modules and
divisors; divisors and integral modules are said to be {\it
associated}.  An integral module consists of all multiples except at
infinity of its associated divisor.

PROOF

Given an integral module, construct its associated divisor by taking
at every place the minimum of the orders of the module's generators at
that place.  This is a Notherian ring, so there is always a finite set
of generators, and each function has only a finite number of zeros and
poles.  Since integral elements have non-negative order at all places
other than infinity, multiplying a generator by an integral element
can only raise its orders at finite places.  Likewise, adding two
elements produces orders at each place greater than or equal to the
minimum of the orders of the summands.  Since all elements of an
integral module have the form:

	$$\sum a_i g_i, \qquad a_i \in {\cal I}, g_i \in {\bf C}(x,y)$$

everything constructed from the module's generators must be a
multiple of the associated divisor except at infinity.

Conversely, I will show that the set of all multiples of a given
divisor $\cal D$ form an integral module.  Use the preceding theorems
to construct a function $f$ with poles and zeros exactly as required
by $\cal D$.  Since ${\bf C}(x,y)$ is a field, the inverse of $f$ exists,
and any multiple $m$ of $\cal D$, when multiplied by $f^{-1}$, will be
integral, and can thus be expanded using an integral basis:

	$$mf^{-1} = \sum a_i b_i, \qquad a_i \in {\bf C}[x], b_i \in {\cal I}$$

Multiplying the integral basis $b_i$ though by $f$ produces a ${\bf
C}[x]$-module, which clearly contains all multiples $m$ of $\cal D$:

	$$m = \sum a_i (b_i f), \qquad a_i \in {\bf C}[x], b_i \in {\bf C}(x,y)$$

This ${\bf C}[x]$-module basis can now be used as the basis for an
${\cal I}$-module, which clearly contains the original ${\bf
C}[x]$-module, and thus all multiples of ${\cal D}$.  Now, since the
integral basis $b_i$ has non-negative order at all finite places, $b_i
f$ can have order no lower than that specified by ${\cal D}$, thus by
the first part of this theorem contains nothing but multiples of
${\cal D}$.  It is thus the integral module desired.

END THEOREM

Theorem A gives a constructive proof for forming an integral module
basis for a given divisor, but we can tighten the result and show that
only two generators are required, no matter how many generators form
the integral basis.

THEOREM B

For any integral module, the order of its associated divisor at every
place is the minimum of the orders of the module's generators at that
place.

END THEOREM

THEOREM F

Given a divisor ${\cal D}$, a pair of functions can always be
constructed that generate the divisor's associated integral module.

PROOF

Use Theorem E to construct a function $f$ with the divisor's required
poles and zeros, zero order at all other places conjugate to those
poles and zeros, and non-negative order elsewhere.  Construct $g$, a
polynomial in $x$ with n-th order roots at all points under n-th order
zeros.  $(f,g)$ is the required basis.  The only finite poles are
those of $f$ and $g$ has zero order everywhere except at $f$'s zeros
and their conjugates, so by Theorem B, $(f,g)$ forms a basis for
${\cal D}$'s associated ${\cal I}$-module.

END THEOREM

Once we have constructed a basis for the integral module, we need to
determine if the module is principal.  Since the total order of a
field element is always zero, this only makes sense for divisors of
zero order.  Futhermore, since an integral module corresponds to
multiples of a divisor {\it except at infinity}, we can't use this
technique to find functions with poles or zeros at infinity, but that
isn't a serious limitation since we can just transform into a field
with an ordinary point at infinity.

A zero order ${\cal I}$-module is principle iff it contains a function
with no poles at infinity.  We can determine this by expressing the
${\cal I}$-module as a ${\bf C}[x]$-module, simply by multiplying the
${\cal I}$-module basis through by an integral basis (remember that an
integral basis is simply a basis for ${\cal I}$ as a ${\bf
C}[x]$-module).

We then check these basis elements to see if one of them has no poles
at infinity.  The most straightforward way to do this is to invert the
field using $z={1\over x}$, which swaps zero with infinity.  We can
then construct an integral basis for the inverse field, and express
each of the module's basis elements (after inverting them) using this
inverse basis.  If there are any poles at infinity in the original
field, they will appear as poles at zero in the inverse field, and can
easily be detected by checking if $z=0$ is a zero of the denominators.

\example Compute $\int {1\over{\sqrt{1-x^2}}} \,dx$

The obvious attempt is to use the algebraic extension $y^2=1-x^2$ and
integrate ${1\over y}\,dx$.

But we first need to determine if this differential has any poles at
infinity, by inverting the field and looking for poles at zero.
Setting $z={1\over x}$, we convert our minimal polynomial into
$z^2y^2=z^2-1$ (after multiplying through by $z^2$), and using
$\hat{y}=zy$ we obtain our inverse field ${\bf C}(z,\hat{y}); \hat{y}^2=z^2-1$.

Since $x={1\over z}$ and $y={\hat{y}\over z}$, we convert our differential as follows:

 $${1\over y}\,dx ={z\over\hat{y}} (-{1\over{z^2}} dz) = -{1\over{z\hat{y}}} dz$$

Now, $\{1, \hat{y}\}$ is an integral basis for the inverse field, so we
multiply through by $\hat{y}\over\hat{y}$ to obtain:

 $${1\over y}\,dx = -{\hat{y}\over{z\hat{y}^2}} dz = -{1\over{z(z^2-1)}}\hat{y} \, dz $$

which is now in normal form and clearly has a pole at $z=0$, or $x=\infty$.  Note that

 $${1\over y} = {z\over\hat{y}} = {{z\hat{y}}\over{\hat{y}^2}}
 = {z\over{z^2-1}} \hat{y}$$

has no pole at $z=0$, a clear example of a differential having a pole
at a place where its constituent function has none.

In any event, we clearly can not use the original field to conduct the
integration, since it would require constructing a function with a
pole at infinity, and our algorithm can't handle this.  So we need to
transform into a field where the differential has no pole at infinity.

We've actually already done this!  Note that the integrand had no pole
at zero in the original field:

 $${1\over y}\,dx = {y\over y^2}\,dx = {1\over{1-x^2}}y \,dx $$

Since the inverse field swapped zero with infinity, it follows that
there is no pole at infinity in the inverse field, so we can proceed
to integrate $-{1\over{z(z^2-1)}}\hat{y} \,dz$ in ${\bf C}(z,\hat{y})$;
$\hat{y}^2=z^2-1$.

Simple inspection of the integrand (already in normal form) shows that
its poles are at $(0, i)$, $(0, -i)$, $(1, 0)$, and $(-1, 0)$.
Remember that we're now working on the Riemann surface of an algebraic
extension, so we need to specify $\it both$ $z$ and $\hat{y}$ to
specify a place.

The next step is to compute the residues at each of these places,
using the Theorem on p. 56 of Trager's thesis:

\begin{center}
\begin{supertabular}{l l l}
  $(0, i)$  &  $\displaystyle -{1\over{(z^2-1)}}\hat{y}$ @ $(0, i)$     & = $i$    \cr
  $(0, -i)$  &  $\displaystyle -{1\over{(z^2-1)}}\hat{y}$ @ $(0, -i)$   & = $-i$    \cr
  $(1, 0)$  &  $\displaystyle -2{1\over{z(z+1)}}\hat{y}$ @ $(1, 0)$      & = $0$    \cr
  $(-1, 0)$  &  $\displaystyle -2{1\over{z(z-1)}}\hat{y}$ @ $(-1, 0)$    & = $0$    \cr
\end{supertabular}
\end{center}

The poles with zero residues can be ignored.  We're interested in the
other two, which exist in ${\bf Q}[i]$, which can be regarded as a
vector field over ${\bf Q}$ with basis $\{1, i\}$, and we want to
construct a function whose poles and zeros match the $i$-component of
the residues (the 1-component is uniformly zero).

We start by constructing an ${\cal I}$-module basis for the divisor
with a simple zero at $(0,i)$ and a simple pole at $(0,-i)$.  Theorem
? shows that:

$$f = {{\hat{y}^2+1}\over{z(\hat{y}+i)}} = {{\hat{y}-i}\over{z}} $$

has a simple pole at $(0,-i)$.  At $(0,i)$, L'H\^opital's rule gives:

$$ \lim_{(z,\hat{y})\to (0,i)} {{\hat{y}-i}\over{z}}
   = {{(\hat{y}-i)'}\over{z'}} {{d\hat{y}}\over{dz}} = {{d\hat{y}}\over{dz}} = {z\over\hat{y}} = 0 $$

where the last transformation was accomplished by differentiating the
mimimal polynomial.  So $f$ has a zero at $(0,i)$, and I'll
note that we've just stumbled into the solution.  Theorem ? already
assures us that $f$ has only a single finite simple pole,
and we can see that its only zeros occur when $\hat{y}-i=0$, which,
according to the minimum polynomial, can only occur at $z=0$, thus
$(0,i)$ is its only finite zero, and it is simple, as we can
verify by showing that the corresponding pole in its inverse is simple:

$$ {1\over f} = {z\over{\hat{y}-i}} = {{z(\hat{y}+i)}\over{\hat{y}^2+1}}
  = {{z(\hat{y}+i)}\over{z^2}} = {1\over z}\hat{y} + {i\over z} $$


So we've found the function we're looking for by accident.  Let's save the
general case for the next example, and convert back to
our original field:

$${{\hat{y}-i}\over{z}} = x({y\over x}-i) = y - ix $$

Remembering that our residues came multiplied by a factor of $i$, we
conclude that our solution is $i\,\ln(y-ix)$, or:

\begin{eqnarray*}
\int {1\over{\sqrt{1-x^2}}} \,dx &=& i\,\ln(\sqrt{1-x^2}-ix) \\
                                 &=& -i\,\ln({1\over{\sqrt{1-x^2}-ix}}) \\
                                 &=& -i\,\ln({{\sqrt{1-x^2}+ix}\over{1-x^2+x^2}}) \\
                                 &=& -i\,\ln({\sqrt{1-x^2}+ix}) \\
                                 &=& \arcsin x \\
\end{eqnarray*}

where I used the negative of a logarithm being the logarithm of
the inverse, and the last transformation came from section ?.


\endexample


\section{Se\~nor Gonzalez, otra vez}

The Rothstein-Trager resultant allows us to compute all the residues
at once.  Trager, in his Ph.D. thesis, then showed how to construct a
function that is zero at all poles with a given residue, and non-zero
at all other poles, as well as at all places conjugate to a pole.
