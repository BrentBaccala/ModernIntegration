
\setcounter{chapter}{8}
\chapter{Simple Algebraic Extensions}

\section{Integral Modules}

In ${\bf C}(x)$, we were working with the quotient field of a
principal ideal ring, so we could always find a single function to
generate any ideal / ${\bf C}[x]$-module.

In ${\bf C}(x,y)$, we are no longer working with a principal ideal
ring, so we can't guarantee that any particular ideal can be generated
by a single function, but it turns out that every ideal can be
generated by a {\it pair} of functions.  Our course of attack is first
to construct that pair of functions, then use them to determine if in
fact the ideal is principal.

DEFINITION

An {\it integral module} (or ${\cal I}$-module) is a module formed
over ${\cal I}$, the ring of integral elements in ${\bf C}(x,y)$.

Since ${\cal I}$ itself can be expressed as a ${\bf C}[x]$-module
using an integral basis, any ${\cal I}$-module is also a ${\bf
C}[x]$-module.  Not all ${\bf C}[x]$-modules are ${\cal I}$-modules,
however, since ${\cal I}$ is typically larger than ${\bf C}[x]$.

THEOREM C

A function can always be constructed with a simple zero at a specified
place $(\alpha, \beta)$, zero order at an additional finite set of
places $\Sigma$, and non-negative order at all other finite places.

PROOF

Begin with the function $(y-\beta)$, which has a simple zero at
$(\alpha, \beta)$ and non-negative order at all finite places.  If
none of the other places in $\Sigma$ have y-value $\beta$, then we are
done, since this function's zeros lie at exactly those places where $y
= \beta$.

Otherwise, compute $(y-\beta)\over(x-\alpha)$ at all places in $\Sigma$
that do {\it not} lie over $x = \alpha$.  Select a number
$\gamma$ different from all of these values.  The function
$(y-\beta) - \gamma (x-\alpha)$ has the desired properties, since
it has non-negative order at all finite places and zero order at
all places in $\Sigma$.

END THEOREM

THEOREM D

A function can always be constructed with a simple pole at a specified
place $(\alpha, \beta)$, zero order at an additional finite set of
places $\Sigma$, and non-negative order at all other places.

PROOF

Begin with the function:

$$f(\alpha,y)\over(x-\alpha)(y-\beta)$$

where $f(x,y)=0$ is the defining polynomial of the algebraic extension.
Note that the division by $(y-\beta)$ will always be exact, since
$f(\alpha, \beta)=0$.  We now have a rational function
$P(y)\over Q(x)$, where $P(y)$ is a polynomial in $y$ and $Q(x)$ is a
polynomial in $x$ ($x-\alpha$, to be precise).  It has a simple pole at
$(\alpha, \beta)$ and non-negative order at all other places, which
is obvious except for places over $x=\alpha$.

(Prove this for other values over $\alpha$).

Now, compute the value of the function at all other places in
$\Sigma$, using either L'Hopital's rule or Puiseux expansion if some
of these are other places over $\alpha$.  If the value of the function
is non-zero at all of these places, then we are done.  Otherwise,
select a number $\gamma$ different from all of these values.  The
function:

$${f(\alpha,y)\over(x-\alpha)(y-\beta)} - \gamma$$

has the desired properties, since it still has a simple pole at
$(\alpha,\beta)$, and is now non-zero at all places in $\Sigma$.

SINGULARITIES?

END THEOREM

THEOREM E

A function can always be constructed with a finite set of poles and
zeros of specified integer orders at specified places, zero order at
an additional finite set of places, and non-negative order at all
other finite places.

PROOF

For each pole or zero, use Theorems C or D to construct a function
with a simple pole or a simple zero at that place, zero order at the
places of all other poles or zeros, and zero order at the additional
set of places where that is required.  Raise each of these function to
the integer power that is the order of the corresponding pole or zero,
then multiply them all together.

END THEOREM

THEOREM A

There is a one-to-one relationship between integral modules and
divisors; divisors and integral modules are said to be {\it
associated}.  An integral module consists of all multiples except at
infinity of its associated divisor.

PROOF

Given an integral module, construct its associated divisor by taking
at every place the minimum of the orders of the module's generators at
that place.  This is a Notherian ring, so there is always a finite set
of generators, and each function has only a finite number of zeros and
poles.  Since integral elements have non-negative order at all places
other than infinity, multiplying a generator by an integral element
can only raise its orders at finite places.  Likewise, adding two
elements produces orders at each place greater than or equal to the
minimum of the orders of the summands.  Since all elements of an
integral module have the form:

	$$\sum a_i g_i, \qquad a_i \in {\cal I}, g_i \in {\bf C}(x,y)$$

everything constructed from the module's generators must be a
multiple of the associated divisor except at infinity.

Conversely, I will show that the set of all multiples of a given
divisor $\cal D$ form an integral module.  Use the preceding theorems
to construct a function $f$ with poles and zeros exactly as required
by $\cal D$.  Since ${bf C}(x,y)$ is a field, the inverse of $f$ exists,
and any multiple $m$ of $\cal D$, when multiplied by $f^{-1}$, will be
integral, and can thus be expanded using an integral basis:

	$$mf^{-1} = \sum a_i b_i, \qquad a_i \in {\bf C}[x], b_i \in {\cal I}$$

Multiplying the integral basis $b_i$ though by $f$ produces a ${\bf
C}[x]$-module, which clearly contains all multiples $m$ of $\cal D$:

	$$m = \sum a_i (b_i f), \qquad a_i \in {\bf C}[x], b_i \in {\bf C}(x,y)$$

This ${\bf C}[x]$-module basis can now be used as the basis for an
${\cal I}$-module, which clearly contains the original ${\bf
C}[x]$-module, and thus all multiples of ${\cal D}$.  Now, since the
integral basis $b_i$ has non-negative order at all finite places, $b_i
f$ can have order no lower than that specified by ${\cal D}$, thus by
the first part of this theorem contains nothing but multiples of
${\cal D}$.  It is thus the integral module desired.

END THEOREM

Theorem A gives a constructive proof for forming an integral module
basis for a given divisor, but we can tighten the result and show that
only two generators are required, no matter how many generators form
the integral basis.

THEOREM B

For any integral module, the order of its associated divisor at every
place is the minimum of the orders of the module's generators at that
place.

END THEOREM

THEOREM F

Given a divisor ${\cal D}$, a pair of functions can always be
constructed that generate the divisor's associated integral module.

PROOF

Use Theorem E to construct a function $f$ with the divisor's required
poles and zeros, zero order at all other places conjugate to those
poles and zeros, and non-negative order elsewhere.  Construct $g$, a
polynomial in $x$ with n-th order roots at all points under n-th order
zeros.  $(f,g)$ is the required basis.  The only finite poles are
those of $f$ and $g$ has zero order everywhere except at $f$'s zeros
and their conjugates, so by Theorem B, $(f,g)$ forms a basis for
${\cal D}$'s associated ${\cal I}$-module.

END THEOREM

Once we have constructed a basis for the integral module, we need to
determine if the module is principal.



Senor Gonzalez, una otra vez

The Rothstein-Trager resultant allows us to compute all the residues
at once.  Trager, in his Ph.D. thesis, then showed how to construct a
function that is zero at all poles with a given residue, and non-zero
at all other poles, as well as at all places conjugate to a pole.
