
In C(x), we were working with the quotient field of a principal ideal
ring, so we could always find a single function to generate any ideal.

In C(x,y), we are no longer working with a principal ideal ring, so we
can't guarantee that any particular ideal can be generated by a single
function, but it turns out that every ideal can be generated by a {\it
pair} of functions.  Our course of attack is first to construct that
pair of functions, then use them to determine if in fact the ideal
is principal.

THEOREM A

There is a one-to-one relationship between integral modules and
divisors; divisors and integral modules are said to be {\it
associated}.  An integral module consists of all multiples except at
infinity of its associated divisor.

PROOF

Given an integral module, construct its associated divisor by taking
at every place the minimum of the orders of the module's generators at
that place (this is a Notherian ring; there are always a finite set of
generators).  Since integral elements have non-negative order at all
places other than infinity, multiplying a generator by an integral
element can only raise its orders at finite places.  Likewise, adding
two elements produces orders at each place greater than or equal to
the minimum of the orders of the summands.  Since all elements of an
integral module have the form:

	$\sum a_i g_i$

everything constructed from the module's generators must be a
multiple of the associated divisor except at infinity.

Likewise, consider a multiple of the associated divisor except at
infinity.

END THEOREM

THEOREM B

For any integral module, the order of its associated divisor at every
place is the minimum of the orders of the module's generators at that
place.

END THEOREM

THEOREM C

A function can always be constructed with unit order (a simple zero)
at a specified place $(\alpha, \beta)$, zero order at an additional
finite set of places, and non-negative order at all other places.

PROOF

Begin with the function $(y-\beta)$, which has a simple zero at
$(\alpha, \beta)$ and non-negative order at all other places.
If none of the other places have y-value $\beta$, then we
are done, since this function's zeros lie at exactly those
places where $y = \beta$.

Otherwise, compute $(y-\beta)\over(x-\alpha)$ at all places
that do {\it not} lie over $x = \alpha$.  Select a number
$\gamma$ different from all of these values.  The function
$(y-\beta) - \gamma (x-\alpha)$ has the desired properties.

END THEOREM

THEOREM D

A function can always be constructed with unit negative order (a
simple pole) at a specified place $(\alpha, \beta)$, zero order at an
additional finite set of places, and non-negative order at all other
places.

PROOF

Begin with the function:

$f(\alpha,y)\over(x-\alpha)(y-\beta)$,

where $f(x,y)$ is the minimum polynomial of the algebraic extension.
Note that the division by $(y-\beta)$ will always be exact.  We now
have a function with a simple pole at $(\alpha, \beta)$ and non-negative
order at all other places. (Prove this for other values over $\alpha$).

Now, compute the value of the function at all other places where it
is required to have zero order, using L'Hopital's rule at other
places over $\alpha$.  If the value of the function is non-zero
at all of these places, then we are done.  Otherwise, select a 
number $\gamma$ different from all of these values.  The function:

$f(\alpha,y)\over(x-\alpha)(y-\beta) - \gamma$,

has the desired properties.

END THEOREM

THEOREM E

A function can always be constructed with a finite set of poles and
zeros of specified integer orders at specified places, zero order at
an additional finite set of places, and non-negative order at all
other places.

PROOF

For each pole or zero, use Theorems C or D to construct a function
with a simple pole or a simple zero at that place, zero order at the
places of all other poles or zeros, and zero order at the additional
set of places where that is required.  Raise each of these function to
the integer order of the corresponding pole or zero, then multiply
them all together.

END THEOREM

THEOREM F

Given a divisor, a pair of functions can always be constructed that
generate the divisor's associated integral module.

PROOF

Use theorem E to construct a function with the divisor's required
poles and zeros, zero order at all other places over those poles
and zeros, and non-negative order elsewhere.  This function,
paired with a polynomial in x with roots at all points under
poles or zeros, forms the required basis.

END THEOREM

Once we have constructed a basis for the integral module, we need to
determine if the module is principal.



Senor Gonzalez, una otra vez

The Rothstein-Trager resultant allows us to compute all the residues
at once.  Trager, in his Ph.D. thesis, then showed how to construct a
function that is zero at all poles with a given residue, and non-zero
at all other poles, as well as at all places conjugate to a pole.
