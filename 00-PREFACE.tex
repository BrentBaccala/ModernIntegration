
\chapter*{Preface}

This book grew out of an abortive class in Risch Integration that I
taught at University of Maryland at College Park in the spring of
2006,\footnote{I am not a professor at UMCP, and am not affiliated
with the University of Maryland in any way other than having studied
physics there as an undergraduate and being a member of the University
Alumni Association.}  which I canceled after three weeks when I
had no students left.  Aside from the lack of student interest (it was
a non-credit class), another deficiency in the class became apparent
to me --- the lack of a good textbook.  So I am writing this book to
fill this perceived gap, the need for a senior level undergraduate
text on differential algebra, developing the subject so far as the
solution of the problem of integration in finite terms (the
integration problem), the theory's most famous application to date.

Why, first of all, should math students study this subject, and why
near the end of an undergraduate mathematics program?

First and foremost, for pedagogical reasons.  Almost all modern
college math curricula include higher algebra, yet this subject seems
to be taught in a very abstract context.  The integration problem puts
this abstraction into concrete form.  We have a specific goal in
mind --- the development of an algorithm that, given an integral
constructed from elementary functions, either solves that integral by
expressing it using elementary functions, or else proves that no such
expression is possible.  One of the best ways to learn a subject, or
at least to convince yourself that you understand it, is to apply it
in a specific and concrete way.  The greatest difficulties I have
encountered in math is when faced with abstract concepts lacking
concrete examples.  Such, in my mind, is the primary goal of studying
differential algebra near the end of an undergraduate program.  The
student has no doubt been exposed to higher algebra, now we want to
make sure we understand it by taking all those rings, fields, ideals,
extensions and what not and applying them to this specific goal.

Secondly, there is a sense of both historical and educational
completion to be obtained here.  Not only has the integration problem
challenged mathematicians since the development of the calculus, but
there is a real danger of getting through an entire calculus sequence
and be left thinking that if you really want to solve an integral, the
best way is to use a computer!  Due to the intricacy of the
calculations involved, the best way probably is to use a computer, but
the student is left at a vague but quite definite disadvantage without
the understanding that the integration problem has been solved and
without some familiarity with the techniques used to solve it.

Third, an introduction to differential algebra may be quite
appropriate at a point where students are starting to think about
research interests.  Though this field has profitably engaged the
attentions of a number of late twentieth century mathematicians, it is
still a young field that may turn out to be a major breakthrough in
the solution of differential equations.  It may also turn out to be a
dead end (``interesting but not compelling'' in the words of one
commentator), which I why I hesitate to list this reason first on my
list.  The big question, in my mind, is whether this theory can be
suitably extended to handle partial differential equations, as both
integrals and ordinary differential equations can now be adequately
handled using numerical techniques.  This question remains unanswered
at this time.

Finally, I have a strong personal motivation in writing this book.
I am not an expert in this field, really a student myself at this
point.  Another very good way to learn a subject, or at least to
convince two people that you understand it, is to explain it to
somebody else.

Since the available material on this subject is too sparsely spread
around among a variety of texts and research papers, I decided for all
of these reasons to compile, more so than write, a book targeted at an
undergraduate audience with some exposure to higher algebra.  However,
in keeping with my primarily pedagogical aims, I re-introduce all the
key concepts of algebra as they are needed.  This serves both to
refresh and reinforce concepts already learned and also to act a
convenient reference without having to flip constantly back and forth
between books.  This book should not be taken as a substitute for a
broader theory text, as I introduce only the concepts needed for my
particular application, and only at a level of detail that seems
appropriate.

Since the book is still a work in progress, I can't hope to
properly conclude this preface at this time.  I would,
however, like to specifically thank Dr. Denny Gulick, Undergradate
Chair of the UMCP Mathematics Department, for giving me
the opportunity to teach the class which lead directly
to this book.
