
\chapter*{Preface}

\begin{comment}
This book grew out of an abortive class in Risch Integration that I
taught at University of Maryland at College Park in the spring of
2006,\footnote{I am not a professor at UMCP, and am not affiliated
with the University of Maryland in any way other than having studied
physics there as an undergraduate and being a member of the University
Alumni Association.}  which I canceled after three weeks when I
had no students left.  Aside from the lack of student interest (it was
a non-credit class), another deficiency in the class became apparent
to me --- the lack of a good textbook.  So I am writing this book to
fill this perceived gap, the need for a senior level undergraduate
text on differential algebra, developing the subject so far as the
solution of the problem of integration in finite terms (the
integration problem), the theory's most famous application to date.
\end{comment}

In 1970, Robert Risch published \cite{risch}, which sketched in four
pages how to bound the torsion of a divisor on an algebraic curve, and
thus provided the ``missing link'' in a comprehensive algorithm that
would either find an elementary form for a given integral, or prove
that no such elementary form can exist.  Risch's method, suitably
enhanced, is currently used in the symbolic integration routines of
the most sophisticated computer algebra systems.

The goal of this book is to present the Risch integration algorithm in a manner
suitable to be understood by undergraduate mathematics students, the
prerequisites being calculus and abstract algebra, and the expected
context being a senior-level university class.

Why, first of all, should math students study this subject, and why
near the end of an undergraduate mathematics program?

First and foremost, for pedagogical reasons.  Almost all modern
college math curricula include higher algebra, yet this subject seems
to be taught in a very abstract context.  The integration problem puts
this abstraction into concrete form.
% We have a specific goal in
% mind --- the development of an algorithm that, given an integral
% constructed from elementary functions, either solves that integral by
% expressing it using elementary functions, or else proves that no such
% expression is possible.
One of the best ways to learn a subject
% , or at least to convince yourself that you understand it,
is to apply it in a specific and concrete way.  The greatest
difficulties I have encountered in math is when faced with abstract
concepts lacking concrete examples.  Such, in my mind, is the primary
benefit of studying Risch integration near the end of an undergraduate
program.  The student has no doubt been exposed to higher algebra, now
we want to make sure we understand it by taking all those rings,
fields, ideals, extensions and what not and applying them to a
specific goal.

Secondly, there is a sense of both historical and educational
completion to be obtained.  Not only has the integration problem
challenged mathematicians since the development of the calculus, but
there is a real danger of getting through an entire calculus sequence
and be left thinking that if you really want to solve an integral, the
best way is to use a computer!  Due to the intricacy of the
calculations involved, the best way probably is to use a computer, but
without studying the Risch algorithm, the student is left with a vague
sense that integration is nothing but a bag of tricks, and
a real deficiency without
understanding that the integration problem has been solved.

Third, an introduction to differential algebra may be quite
appropriate at a point where students are starting to think about
research interests.  Though this field has profitably engaged the
attentions of a number of late twentieth century mathematicians, it is
still a young field that may turn out to be a major breakthrough in
the solution of differential equations.  It may also turn out to be a
dead end (``interesting but not compelling'' in the words of one
commentator), which I why I hesitate to list this reason first on my
list.  The big question, in my mind, is whether this theory can be
suitably extended to handle partial differential equations, as both
integrals and ordinary differential equations can now be adequately
handled using numerical techniques.  This question remains unanswered
at this time, and that mystery has animated my own mathematical
research for a number of years.

% Finally, I have a strong personal motivation in writing this book.
% I am not an expert in this field, really a student myself at this
% point.  Another very good way to learn a subject, or at least to
% convince two people that you understand it, is to explain it to
% somebody else.

Furthermore, the available material on this subject is spread around
among some terse research papers, some sparse lecture notes, and a
single graduate level textbook (\cite{bronstein book}), that while
excellent, is unfortunately incomplete due to the untimely death of
its author prior to completing an anticipated second volume.
Having
slowly assimilated this material over the course of years of study,
and having given roughly a dozen lectures on Risch integration
without the benefit of a textbook, the lack of a suitable text
has become obvious.  Although I began work on this textbook
in 2006, I set it aside after a while and moved on to other
interests.
In the winter of 2016-17, I was once again
preparing to lecture on Risch integration, and once again
scrambling to pull everything together without a textbook.

Therefore, it seems appropriate to compile this knowledge together and
offer it back to the mathematical community.
Partly for the reasons I have listed above, and partly just to write
something different from \cite{bronstein book}, I have decided to
target this book at an undergraduate audience with some exposure to
higher algebra.

I have liberally used the computer algebra system
{\it Sage} in conjunction with the \LaTeX\ package {\tt pythontex}, which,
suitable extended, allows {\it Sage} code in the \LaTeX\ source to be
automatically processed and typeset into the output.  In keeping with
my Christian religious principles, the book is freely
available on the Internet, both in PDF form, and as \LaTeX\ source
in a {\tt github} repository.

Since the book is still a work in progress, I can't hope to properly
conclude this preface at this time.  I would, however, like to
specifically thank my dear friend Bruce Caslow, whose support and
encouragement has been invaluable in this, as well as many other
pursuits.

