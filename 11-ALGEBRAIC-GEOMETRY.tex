
\mychapter{Algebraic Geometry}

\vfill\eject

\begin{maximablock}
diff(sqrt(x^4+1),x);
diff(%,x);
diff(%,x);
diff(%,x);

diff(sqrt(x^3+1),x);
diff(%,x);
diff(%,x);
\end{maximablock}

\vfill\eject


\mysection{Valuations}
\qquad [van der Waerden], \S18.1

A {\it valuation} is a generalization of the absolute value.  A {\it
valuation} is a mapping $\phi$ from a field ${\bf K}$ to an ordered
field ${\cal R}$ (typically the reals) obeying the following axioms:

\begin{center}
\begin{supertabular}{l l l r}
   positivity	& $\forall a \in {\bf K},$ & $\phi(a) \ge 0$ &(V1)\cr
   definiteness & $\forall a \in {\bf K},$ & $\phi(a) > 0 \Longleftrightarrow a \ne 0$ &(V2)\cr
   homomorphism (on the multiplicative group) & $\forall a,b \in {\bf K},$ & $\phi(ab) = \phi(a)\phi(b)$ &(V3)\cr
   subadditivity (or triangle inequality) & $\forall a,b \in {\bf K},$ & $\phi(a+b) \le \phi(a) + \phi(b)$ &(V4)\cr
\end{supertabular}
\end{center}

A moment's thought will show that the standard absolute value on the
reals obeys these axioms, as does the modulus on the complex field.
Valuations are similar to norms, except that norms are defined on
vector spaces, while valuations are defined on fields.

A valuation is said to be {\it non-Archimedian} if it also satisfies
the following axiom, stronger than V4:

\begin{center}
\begin{supertabular}{l l l r}
   non-Archimedian axiom & $\forall a,b \in {\bf K},$ & $\phi(a+b) \le \max(\phi(a), \phi(b))$ &(V4')\cr
\end{supertabular}
\end{center}

In this case, we can switch from a multiplicative to an additive
notation and obtain {\it exponential valuation} by replacing $\phi(a)$
with $w(a) = -\ln \phi(a)$:

\begin{center}
\begin{supertabular}{l l l r}
   & $\forall a \in {\bf K},$ & $w(a) \in (-\infty, \infty]$ &(E1)\cr
   & $\forall a \in {\bf K},$ & $w(a) = \infty \Longleftrightarrow a = 0$ &(E2)\cr
   & $\forall a,b \in {\bf K},$ & $w(ab) = w(a) + w(b)$ &(E3)\cr
   & $\forall a,b \in {\bf K},$ & $w(a+b) \ge \min(w(a), w(b))$ &(E4)\cr
\end{supertabular}
\end{center}
