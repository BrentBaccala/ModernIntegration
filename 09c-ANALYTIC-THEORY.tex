
Definition: a rational function on an algebraic curve is a ratio
of polynomials in x and y

Definition: a meromorphic function on an algebraic curve is a function
analytic everywhere on the curve except at at isolated poles
(including infinity)

Theorem: the rational function field on an algebraic curve is identical
to the curve's meromorphic function field

->: rational functions are meromorphic

at regular points, we can use the implicit function theorem

IFT: [Baby Rudin 9.28; 2-dim complex version] Let f be an analytic
mapping of an open set $E \in {\mathbb C}^2$ into ${\mathbb C}$, such
that $f(x,y)=0$ and $\frac{df}{dx} \ne 0$, then an analytic
function $g(y)$ exists such that $f(x,g(y))=0$.

For infinity and/or poles, substitute z=1/x or v=1/y.

For multiple points, use isolation of the multiple point to justify a
substitution of the form $x=t^r+x_0$, then use composition of analytic
functions (x is analytic everywhere; y is analytic as a function of x
everywhere except at t=0, so y is analytic as a function of t
everywhere except at t=0) to establish that y is analytic everywhere
on the t-plane except possibly at the origin.  Then use existence of
the Laurent series (Silverman 11.2) and continuity of the roots (HOW?)
to establish analyticity at the multiple point.

<-: meromorphic functions are analytic

first, trace of a meromorphic function is meromorphic on C(x), and is
thus a rational function

Liouville's theorem: a bounded entire function is constant

Proof A: (Silverman) use a Taylor series expansion around z=0, which
is valid in the entire plane (since the function is entire).  Cauchy's
inequality |f| \le M ==> |c_n| \le M/{R^n} (eq. 10.8') as R->infty
implies that the function is constant.

Lemma: A entire function with no singularities, even at infinity, is
constant.

Proof: We can do a Taylor series expansion at the origin, whose
non-zero terms will correspond to the principle part of the expansion
at infinity, which must therefore be zero.

Next: A entire function with only a pole at infinity is a polynomial.
The principle part at infinity will be a polynomial.  Subtract it out
to get a function with no singularities, which must be constant.

Next: Given a function with only a finite number of finite poles,
multiply it by a polynomial (the denominator) matching the poles with
zeros.  Now we've got a function with only a pole at infinity, which
must be a polynomial (the numerator).

======

$y^2 = 1 - x^2$

Expand $\frac{1}{y}$ to find its logarithmic residues.

Start with the finite zeros of $y$ (remember $x=t^r+x_0$; $dx=r\,
t^{r-1}\, dt$, so ramification can only raise orders)

$x=1, y=0$

$x=t^2+1$; $x^2=t^4+2t^2+1$; $dx = 2 t dt$

$y=a_0 + a_1 t + a_2 t^2 + a_3 t^3 + \cdots$

$y^2 = a_0^2 + 2 a_0 a_1 t + (2 a_0 a_2 + a_1^2) t^2 + (2 a_0 a_3 + 2 a_1 a_2) t^3 + (2 a_0 a_4 + 2 a_1 a_3 + a_2^2) t^4 + \cdots$

$y^2 = 1 - x^2 = -2t^2 - t^4$

$a_0=0$

$a_1^2 = -2$; $a_1 = \sqrt{2}i$

$a_2 = 0$

$2 a_1 a_3 = -1$; $a_3 = \frac{\sqrt{2}}{4} i$

$y = \sqrt{2}it + \frac{\sqrt{2}}{4} it^3 + \cdots$

$y = t \left[ \sqrt{2}i + \frac{\sqrt{2}}{4} it^2 + \cdots \right]$

$\frac{1}{y} = t^{-1} \left[ -\frac{\sqrt{2}}{2}i + \cdots \right]$

$\frac{1}{y} dx = \left[ -\sqrt{2}i + \cdots \right] $
