
\mychapter{Abelian Integrals}

{\bf THIS CHAPTER IS VERY VAGUE AND INCOMPLETE.}

We can now use the machinery developed in the previous chapter to
solve Abelian integrals in general.

\theorem
Differentials of rational functions on an algebraic curve have no
simple (i.e, first order) poles.
\endtheorem

\theorem
Logarithmic differentials on an algebraic curve have only simple
poles.
\endtheorem

These two theorems, taken together with Liouville's Theorem, provide
our method of attack for integration of Abelian integrals.  We find
all of the poles of the differential, construct Puiseux expansions of
the differential there, and split the principle parts easily into two
sets.  The first order poles arise from logarithm components in the
solution; all the higher order poles must come from the rational
function.  To find the rational function, we integrate term-wise to
obtain its principle parts, and then solve a Mittag-Leffler problem to
see if such a function exists.  For the logarithmic components,
it's a little more complicated.

Once we've calculated the principle parts of an integrand, we can
integrate the resulting series to obtain the principle parts of the
integral.  This is possible due to the simple but crucial observation
that poles in the integral can only appear where there are poles in
the integrand.  Once we've determined the principle parts of the
integral, we need to solve a Mittag-Leffler problem to find an
algebraic function that matches the given principle parts.  This can
be done by finding a basis for a suitable Riemann-Roch space.

Thus, having identified the
locations and orders of the integral's principle parts, we can compute
a Riemann-Roch basis for all functions with suitable poles at those
locations.  Having done so, it then becomes a straightforward exercise
in linear algebra to find a combination of those basis functions
that match a specified set of principle parts.

\theorem
Finite poles are always located over a zero of the denominator.
\endtheorem

\begin{comment}
The techniques of the previous chapter suffice to compute the {\it
algebraic} portion of an Abelian integral, which is to say, an
algebraic function.  Liouville's theorem, however, tells us that there
can also be logarithmic components in the integral.  The two can be
easily separated, since the rational portion of the integral
corresponds to poles of second order poles and higher in the integrand,
while the logarithmic portion corresponds to first order poles
in the integrand.
\end{comment}

\vfill\eject
\mysection{Integration Examples}

\example Evaluate $\int \frac{x}{\sqrt{x^2+1}}\,dx$

This is a simple integral that can be easily solved using first year
calculus techniques, but let's see how to attack it using the more
sophisticated techniques of this chapter.

First, we convert the the integrand into
a rational function on an algebraic curve:

\begin{maximablock}
f : y^2 - (x^2 + 1)$

integrand : x/y$
\end{maximablock}

Next, we identify the poles of the integrand.  The finite
poles can only be located where the denominator is zero,
which is where $x^2+1=0$, over the points $x=\pm i$.

Let's compute the principle parts of $\frac{x}{y}$
at $x=i$.  Calling {\tt puiseux} with {\tt deg=-1} computes just
the principle part of the differential:

\begin{maximablock}
puiseux(f, x, y, %i, false, -1, x/y);
\end{maximablock}

There appears to be a pole here, but appearances are deceptive.
We need to expand the {\it differential}, not the {\it integrand}:

\begin{maximablock}
puiseux(f, x, y, %i, false, -1, x/y*del(x));
\end{maximablock}

So, even though the {\it integrand} has a pole at $(i,0)$, the {\it
differential} does not... and the differential is what matters!

The only other place we might have a pole is infinity.

\begin{maximablock}
puiseux(f, x, y, inf, false, 0, x/y);
\end{maximablock}

Now it appears that we have two sheets with no poles, the expansions indicating simply
that $\lim_{x\to\infty}\frac{x}{\sqrt{x^2+1}} = \pm 1$, depending
on whether we use the positive or negative square root,
but again
we have to take the differential into account.  Since $x=\frac{1}{t}$,
$dx=-\frac{1}{t^2} dt$, and we actually have second order poles
at infinity.

\begin{maximablock}
puiseux(f, x, y, inf, false, 0, x/y*del(x));
\end{maximablock}

Integrating termwise, we see that since our differential has second
order poles at infinity, our integral must have first order poles
at infinity, and theorem ? states that this completely
characterizes the integral.

%We already know of a rational function with first order poles
%at infinity -- $y$ itself!

What functions have first order poles at infinity and nowhere else?

\begin{maximablock}
basis : riemannroch(f, x, y, [[[inf, inf], -1]]);
\end{maximablock}

Our solution, if it exists, is in the vector space spanned by these
three basis elements.  $1$ is in the list, and we expect it to be
there, because of the presence of the constant of integration, so we
can always add a multiple of $1$ to our solution and get another
solution.

What about $x$ and $y$?  We want to
combine them in such a way as to match the principle parts of
our differential.

Let's expand them:

\begin{maximablock}
puiseux(f, x, y, inf, false, 0, x);
puiseux(f, x, y, inf, false, 0, y);
\end{maximablock}

No linear algebra games are required to see that $y$ will match the
principle parts of the differential if it is itself differentiated.
Therefore, $y=\sqrt{x^2+1}$ is our solution.

\begin{maximablock}
'integrate(x/sqrt(x^2+1), x)
   = integrate(x/sqrt(x^2+1), x);
\end{maximablock}

\endexample

\vfill\eject
\mysection{An integral Maxima can't solve}

\example
\label{an integral Maxima can't solve}
Integrate $\int \frac{x^9+2x^7-x}{(x^4+2x^2+1)\sqrt{x^8+1}}\, dx$

When I say that Maxima can't solve this integral, I mean that its
built-in integration routine can't solve the integral:

\begin{maximablock}
integrand: (x^9+2*x^7-x)/((x^4+2*x^2+1)*sqrt(x^8+1));

integrate(integrand, x);
\end{maximablock}

Now let's attack the problem using the techniques of this book.

\begin{maximablock}
f: y^2 - x^8 - 1$

lintegrand: subst(sqrt(x^8+1)=y, integrand);
\end{maximablock}

Inspecting the denominator, we see that since $x^4+2x^2+1=(x^2+1)^2$,
our finite poles can be at $x=\pm i$, as well as the eight roots of
$x^8+1=0$.

\begin{maximablock}

puiseux(f, x, y, %i, false, -1,
        lintegrand * del(x));
puiseux(f, x, y, -%i, false, -1,
        lintegrand * del(x));
\end{maximablock}
\begin{maximablocksmall}
for i:1 thru 15 step 2 do
   disp(puiseux(f, x, y, cis(i*%pi/8), 0, -1,
                lintegrand * del(x)));
\end{maximablocksmall}

We also need to check for poles at infinity.

\begin{maximablock}
puiseux(f, x, y, inf, inf, -1,
        lintegrand * del(x));
\end{maximablock}

We've found four second order poles at the ordinary points $(\pm i, \pm\sqrt{2})$,
as well as two third order poles at a singular point with two sheets at infinity.

Our next goal is to construct a basis for a suitable Riemann-Roch space.
We invert the signs of the finite poles,
since the convention for Riemann Roch spaces is their functions must
have order greater than the {\it negative} of a divisor, remember
that poles decrease in order by $1$ when they are integrated,
and conclude that the Riemann-Roch space that we're interested in is:

$${\cal L}(Z(i, \sqrt{2}) Z(-i, \sqrt{2}), Z(i, -\sqrt{2}) Z(-i, -\sqrt{2}) Z^2(\infty,\infty))$$

i.e, all the functions on our algebraic curve with at most first order
poles at $(\pm i,\pm\sqrt{2})$, no other finite poles, and at most
second order poles at infinity.

\begin{maximablock}
div : [[[%i,  sqrt(2)], -1],
       [[-%i,  sqrt(2)], -1],
       [[%i, -sqrt(2)], -1],
       [[-%i, -sqrt(2)], -1],
       [[inf, inf], -2]]$
basis : riemannroch(f, x, y, div);
\end{maximablock}

We now wish to see if a linear combination of these basis functions will
match the poles in the differential.  This problem is a bit more complicated
than the last one, so let's use the tools we developed in the
last chapter for solving Mittag-Leffler problems:

\begin{maximablock}
m : principle_parts_matrix(f,x,y, basis, div);
\end{maximablock}

Another Maxima function will extract the principle parts of the
differential, and divide them by the powers needed for term-wise
integration:

\begin{maximablock}
principle_parts2(f,x,y, func, divisor) :=
  flatten(map(lambda([d],
    map(lambda([e],
      map(lambda([order], coeff(e[1]/del(t), t, order-1)/order),
          makelist(order, order, d[2], -1))),
       puiseux(f, x, y, d[1][1], d[1][2], 0, func))), divisor)
  )
$

b : transpose(matrix(principle_parts2(f,x,y, lintegrand * del(x), div)));
\end{maximablock}

Now we have a matrix equation that we want to solve:

\begin{maximablock}
m . v = b;
\end{maximablock}

The equation has a non-trivial null space, which we expect because
of the presence of the constant of integration.

\begin{maximablock}
nullspace(m);
\end{maximablock}

To find out if there actually is a solution, we simply have to check:

\begin{maximablock}
v : pi.b;

is(m.v = b);
\end{maximablock}

So, yes, this system does have a solution.  Now let's multiply our
solution vector by the original basis:

\begin{maximablock}
basis.v;
\end{maximablock}

...and convert back to our original form:

\begin{maximablock}
solution : subst([y=sqrt(x^8+1)], %);
\end{maximablock}

Now we can verify the solution:

\begin{maximablock}
diff(solution,x) === integrand;
\end{maximablock}

...and confirm that:

\begin{maximablock}
'integrate(integrand, x) = solution;
\end{maximablock}

% https://ocw.mit.edu/courses/mathematics/18-782-introduction-to-arithmetic-geometry-fall-2013/lecture-notes/MIT18_782F13_lec21.pdf
% Theorem 21.9. For any divisor D we have dim L(D) ≤ deg D0 + 1.

% or use Riemann-Roch theorem directly

\endexample


\vfill\eject
\mysection{Examples}

\example Compute $\int {1\over{\sqrt{1-x^2}}} \,dx$

This is a familiar example from first year calculus, but let's
approach it using the techniques of this book.  We'll
use the algebraic extension $y^2=1-x^2$ and integrate ${1\over
y}\,dx$.

\begin{maximablock}
f : y^2 - (1 - x^2)$
\end{maximablock}

Any finite poles where the denominator is zero?

\begin{maximablock}
puiseux(f, x, y, 1, false,
        -1, 1/y * del(x));
\end{maximablock}

Poles at infinity?

\begin{maximablock}
puiseux(f, x, y, inf, false,
        -1, 1/y * del(x));
\end{maximablock}

Our only poles are at infinity, and they're first order poles,
so this solution will be a logarithm.

The poles' residues (coefficients) are $-i$ and $i$.  These
exist in the field ${\bf Q}[i]$, which can be regarded as a vector
field over ${\bf Q}$ with basis $\{1, i\}$, and we want to construct a
function whose poles and zeros match the $i$-component of the residues
(the 1-component is uniformly zero).

So, we have a singular point at infinity, and we want a function with
a simple zero on one cycle and a simple pole on the other.
The basis will have either one element or no elements, depending
on whether an algebraic function exists with the desired properties.

\begin{maximablock}
basis : riemannroch(f, x, y, [[[inf, inf], [-1,1]]]);
\end{maximablock}

Yes, it does exist.
Remembering that our residues came multiplied by a factor of $i$, we
conclude that our solution is $i\,\ln(y-ix)$, or:

\begin{eqnarray*}
\int {1\over{\sqrt{1-x^2}}} \,dx &=& i\,\ln(\sqrt{1-x^2}-ix) \\
                                 &=& -i\,\ln({1\over{\sqrt{1-x^2}-ix}}) \\
                                 &=& -i\,\ln({{\sqrt{1-x^2}+ix}\over{1-x^2+x^2}}) \\
                                 &=& -i\,\ln({\sqrt{1-x^2}+ix}) \\
                                 &=& \arcsin x \\
\end{eqnarray*}

where I used the negative of a logarithm being the logarithm of the
inverse, and the last transformation came from section
\ref{sec:Liouvillian Forms}.


\endexample

\example Compute $\int \sqrt{4-x^2} \,dx$

A solution method from first year calculus might be to note that this
integrand forms one leg of a right triangle with its other sides $2$
and $x$, but we'll attack this integral using the methods of this
chapter.

First, transform the problem into an algebraic curve:

$$\int y\,dx \qquad y^2 = 4-x^2$$

Since $\lim_{x\to\infty} y = \infty$, the integrand has poles at
infinity.  We want infinity to be an ordinary point of the curve (no
ramification; no singularities) with no poles in the integrand.  The
simplest transformation is again to exchange zero with infinity, and in this
case zero is an ordinary point with places $(0,2)$ and $(0,-2)$,
neither of which is a pole of the integrand.  So we'll invert
$x$ and $y$ into $u$ and $v$:

$$x=\frac{1}{u} \qquad y=\frac{1}{v}$$
$$\left(\frac{1}{v}\right)^2 = 4 - \left(\frac{1}{u}\right)^2 \Longrightarrow 4u^2v^2 - v^2 - u^2=0$$
$$\int\frac{1}{v} \, d\left(\frac{1}{u}\right) \Longrightarrow -\int\frac{1}{vu^2}\,du$$

The only poles in this integrand occur when either $u=0$ or $v=0$.
Substituting these values into $4u^2v^2 - v^2 -u^2=0$, we see that
these condiutions only occur at $(u,v)=(0,0)$, so let's analyze our
curve at that point, starting with the Newton polygon:

\begin{center}
$4 u^2 v^2 - v^2 - u^2 = 0$ \\
\setlength{\unitlength}{1cm}
\begin{picture}(3,3)
\put(0,0){\line(0,1){2.5}}
\put(0,0){\line(1,0){3}}
\put(1.9,-0.1){x}
\put(1.9,1.9){x}
\put(-0.1,1.9){x}
\thicklines
\put(0,2){\line(1,-1){2}}
\end{picture}
\end{center}

The Newton polygon has a single line segment of span 2 and slope -1, so
we have two cycles, each with ramification index one: a singularity.
Since there is no ramification, $u$ is a uniformizing parameter
and we expect to expand $v$ as follows:

$$v = c_1 u + c_2 u^2 + c_3 u^3 + \cdots$$
$$v^2 = c_1^2 u^2 + 2 c_1 c_2 u^3 + (2 c_1 c_3 + c_2^2) u^4 + \cdots$$

Substituting these expansions into $4u^2v^2 - v^2 - u^2 = 0$, we obtain:

$$ 4 c_1^2 u^4 + 8 c_1 c_2 u^5 + (8 c_1 c_3 + 4 c_2^2) u^6 + \cdots $$
$$ - c_1^2 u^2 - 2 c_1 c_2 u^3 - (2 c_1 c_3 + c_2^2) u^4 + \cdots - u^2 = 0$$

Equating terms in $u^2$, we see that $c_1 = \pm i$.  Each of these
two values corresponds to one branch of the singularity.  There
is only a single term in $u^3$, which forces $c_2$ to be zero,
and equating terms in $u^4$ produces $c_3 = 2 c_1$, so

$$v = \pm (iu + 2iu^3 + \cdots) \qquad @(0,0)$$

Inverting $v$ and substituting into our 1-form, we obtain

$$\frac{1}{v} = \pm (-i \frac{1}{u} + 2i u + \cdots) \qquad @(0,0)$$

$$\frac{1}{vu^2}\, du = \pm \left[ -i \frac{1}{u^3} + 2i \frac{1}{u} + \cdots \right] \, du \qquad @(0,0)$$

The $u^{-1}$ terms will integrate into logarithms, so let's ignore
them for the moment and concentrate on the $u^{-3}$ terms, which will
integrate into $u^{-2}$ terms, so we're looking for a function with
second order poles at both places at the $(0,0)$ singularity.

Starting with our standard basis for all rational functions,
$\{1,\,v\}$, we seek to modify it into a basis for
${\rm P}^2(0,0)_a{\rm P}^2(0,0)_b$.  Note first that $v$ has
poles at $u=\pm\frac{1}{2}$.  Using $y=1/u$, we analyze
at $(\pm\frac{1}{2}, \infty)$ as follows:

\begin{center}
$y^2\left[(u-\frac12)^2+(u-\frac12)+\frac14\right]-4(u-\frac12)^2-4(u-\frac12)$
\\
\setlength{\unitlength}{1cm}
\begin{picture}(3,3)
\put(0,0){\line(0,1){2.5}}
\put(0,0){\line(1,0){3}}
\put(0.9,-0.1){x}
\put(1.9,-0.1){x}
\put(-0.1,1.9){x}
\put(0.9,1.9){x}
\put(1.9,1.9){x}
\thicklines
\put(0,2){\line(1,-2){1}}
\end{picture}
\end{center}

Our line segment has span 1 and slope -2, indicating a single place
with ramification 2, and $y$ as a uniformizing parameter.  Setting

$$(u-\frac12) = c_1 y + c_2 y^2 + \cdots$$
$$(u-\frac12)^2 = c_1^2 y^2 + \cdots$$

Substituting, we find that $c_1 = 0$ and $c_2 = \frac{1}{16}$, so

$$(u-\frac12) = \frac{1}{16} y^2 + \cdots \qquad v=y^{-1} \qquad @(\frac12, \infty)$$

$$(u+\frac12) = \frac{1}{16} y^2 + \cdots \qquad v=y^{-1} \qquad @(-\frac12, \infty)$$

In short, $v$ has first order poles at $(\pm\frac12,\infty)$ and
$(u\pm\frac12)$ has second order zeros, so we can adjust our basis
accordingly and obtain $\{1,\,(4u^2-1)v\}$ for a basis with no finite
poles.  We can also use a theorem of Trager to shortcut this calculation.

Returning to our analysis at $(0,0)$, we see that 1 has zero order
(obviously) and $(4u^2-1)v$ has a first order zero at both sheets
there, since $4u^2-1=-1$ is finite and $v$ has first order zeros.
We also know that $u$ is a uniformizing parameter, so it's easy
to modify our basis and obtain

$$\left\{\frac{1}{u^2},\,\frac{4u^2-1}{u^3}v\right\} {\rm is\, a\,} {\bf C}[x]{\rm -basis\, for\, P^2(0,0)_aP^2(0,0)_b}$$

Is this basis normal at infinity?  Well, the representation order of
$\frac{1}{u^2}$ is 2 and its $u^-2$ coefficients at $(\infty, \pm
\frac12)$ are both 1, while the representation order of $\frac{4u^2-1}{u^3}v$
is 1, and its $u^-1$ coefficients are 2 and -2.  Since

$$\det C = \begin{array}{|cc|} 1 & 2 \\ 1 & -2 \end{array} = -4$$

is non-zero, the basis is normal at infinity.

The Riemann-Roch theorem says that the dimension of ${\mathfrak l}(D)$ is 5,
$\frac{1}{u^2}$ can be multiplied by any polynomial up to second
degree without introducing poles at infinity, and $\frac{4u^2-1}{u^3}v$
can be multiplied by any polynomial up to first degree, so

$$\left\{\frac{1}{u^2},\, \frac{1}{u},\, 1,\, \frac{4u^2-1}{u^3}v,\, \frac{4u^2-1}{u^2}v\right\}$$

is a ${\cal C}$-module basis for ${\mathfrak l}(D)$.

Any linear combination of these functions is a multiple of the
divisor, but not all of them produce the correct residues.  Looking at
the residues, we see that only $\frac{4u^2-1}{u^3}v = \frac{1}{uv}$
has residues of $\pm i$ on the two sheets at the $(0,0)$ singularity.
Dividing by 2 to correct for the 2 that will be introduced by the
integration, we conclude that $\frac{1}{2uv} = \frac{xy}{2} =
\frac{x\sqrt{4-x^2}}{2}$ is the desired function.

Next, we have to deal with the logarithms.  Going back to the
series expansions of our 1-form, we see that we have residues
of $\pm 2i$ on our two sheets at $(0,0)$.  The objective
now is a bit different; we want a function with exactly
the divisor $Z(0,0)_a P(0,0)_b$.  Starting with an integral basis:

$$\{1, (4u^2-1)v\}$$

we want to modify these functions to make them multiples
of $Z(0,0)_a P(0,0)_b$.  The pole isn't a problem for
an integral basis, and looking at the series expansion
for $v$ at $(0,0)$ we see that it (and therefore $(4u^2-1)v$)
has a simple zero there, but $1$ needs to be replaced with $u$:

$$\{u, (4u^2-1)v\}$$

Now we construct a matrix with the coefficients in the series expansions:

$$\left[ \begin{array}{cc} 1 & -i \\ 0 & 0 \end{array} \right] \begin{array}{ll} \leftarrow (0,0)_a \\ \leftarrow (0,0)_b \end{array} $$

$$\left[ \begin{array}{cc} 1 & -i \\ 0 & 0 \end{array} \right] \left[ \begin{array}{c} i \\ 1 \end{array} \right] = 0$$

The solution shows us how to modify the basis:

$$\{u, \frac{iu + (4u^2-1)v}{u}\} = \{u, i + \frac{(4u^2-1)v}{u}\}$$

$$\left[ \begin{array}{cc} 1 & 0 \\ 0 & 0 \end{array} \right] \begin{array}{ll} \leftarrow (0,0)_a \\ \leftarrow (0,0)_b \end{array} $$

$$\left[ \begin{array}{cc} 1 & 0 \\ 0 & 0 \end{array} \right] \left[ \begin{array}{c} 0 \\ 1 \end{array} \right] = 0$$

$$\{u, i\frac{1}{u} + \frac{(4u^2-1)v}{u^2}\}$$

$$\left| \begin{array}{cc} 1 & -2i \\ 0 & 2i \end{array} \right| = 2i$$

At the last step, the determinant is non-zero, which shows that we
now have a basis for multiples of the divisor except at infinity.
Is it normal at infinity?  $u$'s expansion at both places at infinity
is $\left(\frac{1}{u}\right)^{-1}$, so its representation order is -1,
and the second element's expansion at infinity starts $\pm 2 + \cdots$,
so its representation order is 0 and:

$$\det C = \begin{array}{|cc|} 1 & 2 \\ 1 & -2 \end{array} = -4$$

So the basis is normal at infinity.  If an exact multiple of
the divisor exists, it is one of the basis elements.  It's not $u$,
since $u$ has a pole at infinity, but the second element is exact:

$$i\frac{1}{u} + \frac{(4u^2-1)v}{u^2} = i\frac{1}{u} - \frac{1}{v} = ix-y$$

The desired residues are $\pm 2i$, so the function we want is

$$2i \ln(ix-y) = 2i \ln(\frac{y}{2}-i\frac{x}{2}) + 2i \ln(-2) $$
$$= 2i \ln\left(\sqrt{1-\left(\frac{x}{2}\right)^2} - i\frac{x}{2}\right) = 2i (-i \arcsin \frac{x}{2}) = 2 \arcsin \frac{x}{2}$$

(the constant disappears into the constant of integration) and the final answer is:

$$ \int \sqrt{4-x^2} \, dx  = 2\arcsin\frac{x}{2} + \frac{x \sqrt{4-x^2}}{2}$$

\endexample

\vfill\eject
\mysection{Geddes's example}

\example Compute $\int {1\over{x\sqrt{x^4+1}}} \, dx$

We'll use ${\bf C}(x,y); y^2=x^4+1$ and integrate ${1\over{xy}} =
{y\over{x^5+x}}$.  Inverting this field ($z={1\over x}$) shows that
this integrand has no poles at infinity, so we can proceed directly:

$$ {y\over{x^5+x}} = {y\over{x(x+\omega)(x-\omega)(x+i\omega)(x-i\omega)}} \qquad \omega = \sqrt{i} = {\sqrt{2}\over2} + {\sqrt{2}\over2} i $$

\bigskip
\begin{center}
\begin{supertabular}{l l l}
  $(0, 1)$  &  $\displaystyle {y\over{x^4+1}}$ @ $(0, 1)$     & = $1$    \cr
  $(0, -1)$  &  $\displaystyle {y\over{x^4+1}}$ @ $(0, -1)$     & = $-1$    \cr
  $(\omega, 0)$  &  $\displaystyle 2{y\over{x(x^2+i)(x+\omega)}}$ @ $(\omega, 1)$     & = $0$    \cr
  \multicolumn{2}{l}{$(-\omega, 0), (i\omega, 0), (-i\omega, 0)\qquad\cdots$}    & = $0$    \cr
\end{supertabular}
\end{center}

We now use theorem \ref{simple pole construction} to construct a
function with a simple pole at $(0,-1)$:

$${{f(0,y)}\over{x(y+1)}} = {{y^2-1}\over{x(y+1)}} = {{y-1}\over{x}} $$

This function has a zero at $(0,1)$, but, unfortunately, it is third order,
as can be seen from either L'H\^opital's rule:

$$y^2=x^4+1$$
$$2y\,dy=4x^3\,dx$$
$${{dy}\over{dx}} = 2{x^3\over y}$$

$${{y-1}\over{x}} @ (0,1) = \lim {{dy}\over{dx}} = 2 {{x^3}\over y} = 0$$
$${{y-1}\over{x^2}} @ (0,1) = \lim {1\over{2x}}{{dy}\over{dx}} = {{x^2}\over y} = 0$$
$${{y-1}\over{x^3}} @ (0,1) = \lim {1\over{3x^2}}{{dy}\over{dx}} = {2\over3}{{x}\over y} = 0$$
$${{y-1}\over{x^4}} @ (0,1) = \lim {1\over{4x^3}}{{dy}\over{dx}} = {1\over2}{1\over y} = {1\over2}$$

\vfil\eject

\ldots or from a series expansion of $y$ at (0,1):

$$y^2 = x^4 + 1 $$
$$(y-1)^2 = x^4 - 2(y-1)$$
$$(y-1) = {1\over2}x^4 - {1\over2}(y-1)^2$$
$$(y-1) = c_0 + c_1 x + c_2 x^2 + c_3 x^3 + \cdots$$
$$(y-1)^2 = c_0^2 + (2 c_0 c_1) x + (2 c_0 c_2 + c_1^2) x^2 + (2 c_0 c_3 + 2 c_1 c_2) x^3 + \cdots$$

$$ c_0, c_1, c_2, c_3 = 0 $$
$$ c_4 = {1\over2}$$
$$ c_5, c_6, c_7 = 0 $$
$$ c_8 = -{1\over8} $$

$$ (y-1) = {1\over2} x^4 - {1\over8} x^8 + \cdots$$

$$ {(y-1)\over x} = {1\over2} x^3 - {1\over8} x^7 + \cdots$$

\ldots or from the norm:

$$N({{y-1}\over{x}}) = {{y-1}\over{x}} \cdot {{-y-1}\over{x}} = - {{y^2-1}\over{x^2}} = - {{x^4}\over{x^2}} = - x^2$$

Since we know that the function has a simple pole at $(0,-1)$, so it
must have a third order zero at $(0,1)$ to form a norm with a second
order zero.

We can eliminate the inconvenient zero by adding a constant to the
function, say 1: ${{x+y-1}\over x}$.  We can now use theorem
\ref{simple zero construction} to create a simple zero at
(0,1) by multiplying by $x+y-1$:

$${{x+y-1}\over x} (x+y-1) = {{2x-2}\over x} y + {{x^4+x^2-2x+2}\over x} $$

This function has a simple pole at (0,-1) and a simple zero at (0,1),
but does it have other poles and zeros?  If so, can it be modified to
eliminate them?  To find out, we form the generators of an ${\cal I}$-module:

$$\{ {{2x-2}\over x} y + {{x^4+x^2-2x+2}\over x}, x \}$$

Noting that ${{x^4+x^2}\over{x}} = x(x^2+1)$ and $x^2+1 \in {\cal I}$,
we can simplify this:

$$\{ {{2x-2}\over x} y - {{2x-2}\over x}, x \}$$

\vfill\eject

Using the integral basis $\{1, y \}$, we convert this to a
${\bf C}[x]$-module:

$$\{ {{2x-2}\over x} y - {{2x-2}\over x}, {{2x-2}\over x} (x^4+1) - {{2x-2}\over x} y, x, xy \}$$

and since $(2x-2){{x^4}\over{x}} = x(2x^3-2x^2)$ and
$2x^3-2x^2 \in {\bf C}[x]$, we simplify:

$$\{ {{2x-2}\over x} y - {{2x-2}\over x}, {{2x-2}\over x} - {{2x-2}\over x} y, x, xy \}$$

and write it in matrix form:

% $$\pmatrix{-{{2x-2}\over{x}} & {{2x-2}\over{x}} \cr {{2x-2}\over{x}} & -{{2x-2}\over{x}} \cr x & 0 \cr 0 & x} \pmatrix{1 \cr y}$$

$${1\over x}\begin{pmatrix}-(2x-2) & 2x-2 \cr 2x-2 & -(2x-2) \cr x^2 & 0 \cr 0 & x^2\end{pmatrix} \begin{pmatrix}1 \cr y\end{pmatrix}$$

Elementary row operations\footnote{Read right to left; $R_{i,j,\lambda}$ adds $\lambda$ times row $j$ to row $i$; $R_{i,\alpha}$ multiplies row $i$ by $\alpha$ (a unit)} $R_{4,3,x^2} R_{1,3,(2x-2)} R_{3,4,-1} R_{3,1,{1\over2}(x+1)} R_{2,1,1} $ yield:

$${1\over x}\begin{pmatrix}1 & -1 \cr x^2 & 0\end{pmatrix} \begin{pmatrix}1 \cr y\end{pmatrix} = \begin{pmatrix}{{1-y}\over{x}} \cr x\end{pmatrix} $$

so $\{ {{1-y}\over{x}}, x \} $ forms a generator set for
the ${\bf C}[x]$-module of the finite multiples of $Z(0,1)P(0,-1)$.
We convert to a basis normal at infinity: $\{1, v\} = \{1, {y\over x^2}\}$:

$$\begin{pmatrix}{{1-y}\over{x}} \cr x\end{pmatrix} = \begin{pmatrix}{1\over x} & -x \cr x & 0\end{pmatrix} \begin{pmatrix}1 \cr {y\over x^2}\end{pmatrix}
= \begin{pmatrix}x & \cr & x\end{pmatrix} \begin{pmatrix}{1\over x^2} & -1 \cr 1 & 0\end{pmatrix} \begin{pmatrix}1 \cr {y\over x^2}\end{pmatrix}$$

$$\det_{@ \infty} \begin{pmatrix}{1\over x^2} & -1 \cr 1 & 0\end{pmatrix} = 1$$

so $\{ {{1-y}\over{x}}, x \} $ is normal at infinity.  $x$ clearly has
a pole at infinity, so it can't be the function we're looking for, but
what about ${{1-y}\over{x}}$?  Switching back to $\{u,v\}$
coordinates, we obtain ${{1-y}\over{x}} = {{u^2-v}\over u}$, which has
$\{u,v\}$ poles at both $\{0,1\}$ and $\{0,-1\}$, which translate into
poles at $x=\infty$.  Therefore, no rational function on this
algebraic curve has a simple pole at (0, -1), a simple zero at (0,1),
and no other poles or zeros.

\vfill\eject

So, let's try a double pole at (0,-1) and a double zero at (0,1).  We
can just square our previous generators:
$\{ {{1-y}\over{x}}, x \} $
to obtain: 
$\{ {(1-y)^2\over{x^2}}, x^2 \} = \{ {{1-2y+x^4+1}\over{x^2}}, x^2 \}$
which simplifies to
$\{ {{1-y}\over{x^2}}, x^2 \}$.  We again check for normalcy
at infinity:

$$\begin{pmatrix}
{1\over{x^2}} & -1 \cr
x^2 & 0\end{pmatrix} \begin{pmatrix}1 \cr {y\over{x^2}}\end{pmatrix} = \begin{pmatrix}1 & \cr & x^2\end{pmatrix}\begin{pmatrix}
{1\over{x^2}} & -1 \cr
1 & 0\end{pmatrix}
\begin{pmatrix}1 \cr {y\over{x^2}}\end{pmatrix} $$

$$\det_{@\infty}\begin{pmatrix}
{1\over{x^2}} & -1 \cr
1 & 0\end{pmatrix} = 1$$

So, $\{{{1-y}\over{x^2}}, x^2\}$ is a ${\bf C}[x]$-module, normal at
infinity, containing the finite multiples of $Z^2(0,1)P^2(0,-1)$.
$x^2$ has a pole at infinity, but does ${1-y}\over{x^2}$?
Switching to $x={1\over z}; y={u\over{z^2}}; u^2 = z^4 + 1$ and
Writing it as $z^2(1-{u\over{z^2}}) = z^2 - u$ shows that it has no
zero at $(z,u) = (0, \pm 1)$, and thus no pole at $x = \infty$.  It
is, therefore, the function we are looking for:

$$\int {1\over{x\sqrt{x^4+1}}} \, dx
   = {1\over2} \ln{{1-\sqrt{x^4+1}}\over{x^2}}$$

I'll now point out to you what's been pointed out to me, and that is a
traditional solution technique for this integral:

$$\int {1\over{x\sqrt{x^4+1}}} \, dx$$
$$x^2 = u \qquad 2 x\, dx = du$$
$$\int {1\over{2u\sqrt{u^2+1}}} \, du$$
$$u = \tan z \qquad du = \sec^2 z dz$$
$${1 \over 2} \int \csc z \, dz = {1\over 2} \ln \tan {z\over2}$$
$$={1\over2} \ln {{\sec z - 1}\over{\tan z}}$$
$$={1\over2} \ln {{\sqrt{u^2+1}-1}\over{u}}$$
$$={1\over2} \ln {{\sqrt{x^4+1}-1}\over{x^2}}$$

\vskip 0.5in

$${{1 - \cos z}\over{\sin z}} = {{1 - \cos^2 {z\over2} + \sin^2 {z\over2}}\over{2\sin{z\over2}\cos{z\over2}}} = {{2\sin^2 {z\over2}}\over{2\sin{z\over2}\cos{z\over2}}} = {{\sin {z\over2}}\over{\cos{z\over2}}} $$

\vfill\eject

We can also proceed like this:

$$\int {1\over{x\sqrt{x^4+1}}} \, dx$$
$$x^4 = u \qquad 4 x^3\, dx = du$$
$$\int {1\over{4u\sqrt{u+1}}} \, du$$
$$v = u + 1 \qquad dv = du$$
$$\int {1\over{4(v-1)\sqrt{v}}} \, dv$$
$$z^2 = v \qquad 2 z dz = dv$$
$$\int {1\over{4(z^2-1)z}} \, 2z\, dz$$
$$\int {1\over{2(z^2-1)}} \, dz$$
$${1\over{z^2-1}} = {1\over2}{1\over{z-1}} - {1\over2}{1\over{z+1}}$$
$${1\over4} \int {1\over{z-1}} - {1\over{z+1}} \, dz$$
$${1\over4} \ln (z-1) - \ln (z+1)$$
$${1\over4} \ln {{z-1}\over{z+1}}$$
$${1\over4} \ln {{\sqrt{v}-1}\over{\sqrt{v}+1}}$$
$${1\over4} \ln {{\sqrt{u+1}-1}\over{\sqrt{u+1}+1}}$$
$${1\over4} \ln {{\sqrt{x^4+1}-1}\over{\sqrt{x^4+1}+1}}$$

so from the previous page and this one, we conclude

$$\int {1\over{x\sqrt{x^4+1}}} \, dx
={1\over2} \ln {{\sqrt{x^4+1}-1}\over{x^2}}
={1\over4} \ln {{\sqrt{x^4+1}-1}\over{\sqrt{x^4+1}+1}}$$

Is this last equality true?  Well, $\ln f^2 = 2\ln f$, so
${1\over4}\ln f^2 = {1\over2}\ln f$, and\ldots

$$\Big({{\sqrt{x^4+1}-1}\over{x^2}}\Big)^2
= {{(\sqrt{x^4+1}-1)^2}\over{x^4}}$$

$${{\sqrt{x^4+1}-1}\over{\sqrt{x^4+1}+1}} \cdot
{{\sqrt{x^4+1}-1}\over{\sqrt{x^4+1}-1}}
= {{(\sqrt{x^4+1}-1)^2}\over{x^4+1-1}}$$




\endexample

\vfill\eject
\mysection{Chebyshev's Integral}

\example Compute:
\label{Chebyshev's Integral}
$$\int {{2x^6+4x^5+7x^4-3x^3-x^2-8x-8}\over{(2x^2-1)^2\sqrt{x^4+4 x^3+2 x^2+1}}} \,{\rm d}x$$

The polynomial under the square root is square-free:

\begin{maximablock}
num : 2*x^6 + 4*x^5 + 7*x^4-3*x^3-x^2-8*x-8;
den: 2*x^2-1;
root : x^4+4*x^3+2*x^2+1;
factor(root);
\end{maximablock}

\ldots so $y^2 = x^4+4 x^3+2 x^2+1$; $\{1, y\}$ is an integral basis;
and our normal form for this integral is:

$$\int {{(2x^6+4x^5+7x^4-3x^3-x^2-8x-8)y}\over{(2x^2-1)^2(x^4+4 x^3+2 x^2+1)}} \,{\rm d}x$$

Applying now Bronstein's Hermite reduction from
section 2.1 of his ``Symbolic Integration Tutorial'' with $v=2x^2-1$
to eliminate this square in the denominator:

\begin{maximablock}
gradef(y,x,ratsimp(D(root)/2*y/root));
D(y);
U: root;
V: den;
S2: ratsimp(U*V^2*D(y/V));
\end{maximablock}

Now we want to solve $f_2 S_2 = A_2 y$ where $A_2 y$ is our numerator.

\begin{maximablock}
kill(f);
f[2] ::: num*y/S2;
T[2] ::: num(f[2]);
/* Q::: S2/y; */
Q ::: denom(f[2]);

[A,R,g] : gcdex(V,Q,x);
/* T2: num; */
[Q2, B2]: divide(T[2]*R, V, x);
h ::: A*num*y/(V*U) - (D(V)*Q2+D(B2))*y/V + Q2*D(y);

simp: false$
'integrate(num*y/den,x) = ratsimp(B2*y/V) + 'integrate(h,x);
simp: true$
\end{maximablock}

\vfill\eject
\bigskip
\begin{center}
Non-zero residues

\begin{supertabular}{r @{} l | r @{} l | r @{} l}
\multicolumn{2}{c|}{x} & \multicolumn{2}{c|}{y} & \multicolumn{2}{c}{residue} \cr
\hline
&$\sqrt{2}\over 2$ & &${1\over 2} + \sqrt{2}$ & &${5\over2}$ \cr
&$\sqrt{2}\over 2$ & $-$&${1\over 2} - \sqrt{2}$ & $-$&${5\over2}$ \cr
$-$&${\sqrt{2}\over 2}$ & &${1\over 2} - \sqrt{2}$ & &${5\over2}$ \cr
$-$&${\sqrt{2}\over 2}$ & $-$&${1\over 2} + \sqrt{2}$ & $-$&${5\over2}$ \cr
\end{supertabular}
\end{center}


$$A(x) = 1023x^8+4104x^7+5048x^6+2182x^5+805x^4+624x^3+10x^2+28x$$
$$B(x) = 1025x^{10} + 6138x^9 + 12307x^8 + 10188x^7 + 4503x^6 + 3134x^5 + 1598x^4 + 140x^3 + 176x^2 +2$$
$$C(x) = 32x^{10}-80x^8+80x^6-40x^4+10x^2-1$$

$$y = \sqrt{x^4+4 x^3+2 x^2+1}$$

$$\int {{(2x^6+4x^5+7x^4-3x^3-x^2-8x-8)}\over{(2x^2-1)^2\sqrt{x^4+4 x^3+2 x^2+1}}} \,{\rm d}x
= {{(x+{1\over2})y}\over{2x^2-1}} + {1\over2}\ln{{A(x)y - B(x)}\over{C(x)}}
$$


\endexample

\vfill\eject
\mysection{The Risch Theorem: A First Look}

At this point, there is only one major missing piece in our
integration theory for Abelian integrals --- how do we limit the
multiples of a divisor to a testable set?  We've seen how to
repeatedly raise a divisor to higher and higher powers, but how do we
know when to stop?  At what point can we declare that a divisor has no
multiple that is principle?

We'll attack this problem the way Robert Risch discovered in 1970, by
mapping into a finite field, solving the corresponding problem there,
then lifting the result back to the original field.  The finite field
will be the integers modulo a prime.  Reducing modulo some primes
changes the genus of the algebraic curve, while other primes leave the
genus unchanged.  Those primes at which the genus of the curve remains
unchanged yield {\it good reduction}

The key theorems as stated by [Tr73] on page 67:

\theorem
The homomorphism between divisor class groups under good reduction
is an isomorphism when restricted to divisors whose orders are
relatively prime to the characteristic of the reduced function field.
\endtheorem

Let $p$ be the characteristic of the reduced field,

\theorem
if the divisor $D$ has order $p^k n$ where $\gcd(n,p)=1$, then
the reduction of $D$ must have order $p^j n$ for some $j\le k$.

\proof
Let the order of the reduction be $p^j m$.  Since reduction
is a group homomorphism, we must have $m|n$ and $j \le k$.
Since $D^{p^k}$ has order exactly $n$, its reduction
must have order exactly $n$.  But the order of its
reduction is a divisor of $m$ and thus $n|m$ and so
finally we have $n=m$.

\endtheorem


Since good reduction preserves the part of the divisor's order
relatively prime to the characteristic, by picking two different
primes we can completely reconstruct the divisor's order in
characteristic zero.

A property of algebraic curves over finite fields is that {\it all}
divisors of total degree zero have some multiple that is principal
(PROOF).  Thus, to carry out this program, we need to compute bases
for Riemann-Roch spaces in prime characteristic.  Then, we can keep
raising a divisor to higher and higher powers until we find its order.
Do this for two different primes, each exhibiting good reduction, and
then we can find the order of the original divisor in characteristic
zero.

To proof the above claims, I want to construct the Jacobian variety in
arbitrary characteristic, then show how Jacobians map under good
reduction.

\section{Simple Algebraic Extensions over Finite Fields}

Let's start with a simple but crucial observation:

\theorem

In an algebraic extension over a finite field, the evaluation field is
also finite.

\proof

Consider a finite field of constants ${\cal F}$, over which we'll
extend first into a rational function field ${\cal F}(x)$ and then add
an algebraic extension ${\cal F}(x,y)$, where $y$ satisfies some
minimial polynomial $f(x,y)=0$.  Start with the constant field, which
gives us a finite number of values for $x$.  Plugging each of these
values into the minimal polynomial gives a finite set of polynomials
$f(y_i)=0$.  By Theorem ?, we can extend ${\cal F}$ into a finite
extension field ${\cal F}[\gamma]$ where all the roots of the
polynomial exist.  Since there a only a finite number of polynomials,
we need at worst a finite set of extensions ${\cal
F}[\gamma_1,...,\gamma_k]$ to construct a field in which all the roots
of all the polynomials exist.  Using the Theorem of the Primitive
Element, we can collapse all of these into a single finite extension
field ${\cal F}[\phi]$.  Since all values of $x$ exist in ${\cal F}$,
and all values of $y$ exist in ${\cal F}[\phi]$, an evaluation
homomorphism carries any rational function in $x$ and $y$ into
${\cal F}[\phi]$.

\endtheorem

This theorem leads directly to the single more important difference
(to us) between divisors in an infinite field versus those in a finite
field.  {\it In a finite field, some multiple of every divisor is
principle.}  The reason is that the multiplicative group of the
evaluation field has finite order.  The simplest way to demonstrate
this is to construct theorems analogous to Theorems ? and ?:

\theorem

In an algebraic extension of a finite field with characteristic
greater than 2, a function can always be constructed with an $m^{\rm
th}$-order zero at a specified place $(\alpha, \beta)$ and zero order
at all other finite places, where $m$ is the multiplicative order of
the evaluation field.

\proof

The desired function is

$$(x-\alpha)^m + (y-\beta)^m$$.

Clearly, this function is zero at $(\alpha, \beta)$ and of $m^{\rm
th}$ order there (PROOF THIS).  At all other places one of the two
terms will be non-zero, and both exist in the evaluation field.  By
Theorem ?, any non-zero number raised to the multiplicative order of
its field is one.  Thus the value of this function will be either
$1+0$, $0+1$, or $1+1=2$, all finite and non-zero, and thus of zero
order.

\endtheorem

\theorem

In an algebraic extension of a finite field with characteristic
greater than 2, a function can always be constructed with an $m^{\rm
th}$-order pole at a specified place $(\alpha, \beta)$ and zero order
at all other finite places, where $m$ is the multiplicative order of
the evaluation field.

\proof

The desired function is

$${f(\alpha,y)^m\over(x-\alpha)^m(y-\beta)^m} + 1$$

where the division by $(y-\beta)^m$ is exact.
Clearly, this function has a pole at $(\alpha, \beta)$ and of $m^{\rm
th}$ order there (PROOF THIS).  CONSIDER OTHER PLACES OVER $\alpha$.
At all other places the denominator
term will be non-zero, and thus one, and the numerator will be
either zero or one (by Theorem ?)
Thus the value of this function at these places will be either
$0+1$ or $1+1=2$, both finite and non-zero, and thus of zero
order.

\endtheorem


\example

Show that some multiple of ${\mathrm Z}(1,1)$ is principle in
${\bf Z}_5(x,y); y^2=x$.

Let's first construct a multiplication table for ${\bf Z}_5$:

\begin{center}
\begin{tabular}{c|c c c c c}
  & 0 & 1 & 2 & 3 & 4 \cr
\hline
0 & 0 & 0 & 0 & 0 & 0 \cr
1 & 0 & 1 & 2 & 3 & 4 \cr
2 & 0 & 2 & 4 & 1 & 3 \cr
3 & 0 & 3 & 1 & 4 & 2 \cr
4 & 0 & 4 & 3 & 2 & 1 \cr
\end{tabular}
\end{center}

Now, let's list out the places on the Riemann surface for
${\bf Z}_5(x,y); y^2=x$.

\begin{center}
\begin{tabular}{c l}
$x$ & $(x,y)$ \cr
\hline
0 & (0,0) \cr
1 & (1,1) \quad (1,4) \cr
2 & $(2,\gamma) \quad (2,-\gamma); \quad \gamma^2 - 2 =0$ \cr
3 & $(3,\theta) \quad (3,-\theta); \quad \theta^2 - 3 =0$ \cr
4 & (4,2) \quad (4,3) \cr
\end{tabular}
\end{center}

It looks like we need ${\bf Z}_5[\gamma,\theta]$ to express these places.
It's simplest to collapse $\gamma$ and $\theta$ into a single algebraic
extension.  We could use the Theorem of the Primitive Element to
do this, but in this case just looking at the multiplication table
and noting that $3 = 2^3 = \gamma^6$ shows that $\theta = \pm \gamma^3$.
So, in fact, we only need ${\bf Z}_5[\gamma]$:

\begin{center}
\begin{tabular}{c l}
$x$ & $(x,y)$ \cr
\hline
0 & (0,0) \cr
1 & (1,1) \quad (1,4) \cr
2 & $(2,\gamma) \quad (2,-\gamma); \quad \gamma^2 - 2 =0$ \cr
3 & $(3,\gamma^3) \quad (3,-\gamma^3)$ \cr
4 & (4,2) \quad (4,3) \cr
\end{tabular}
\end{center}

Since ${\bf Z}_5[\gamma]$ has $5^2=25$ elements, its multiplicative
group has order one less than this.  We conclude that 24 is our
``magic'' multiple, and that ${\mathrm Z}^{24}(1,1)$ must be
principle in this field.  Its generator should be simply
$(x-1)^{24} + (y-1)^{24}$.  Clearly this function is zero for
$(x,y)=(1,1)$.  Let's verify that it's non-zero for some other
places on the Riemann surface:

\begin{eqnarray*}
(0,0) &:& (-1)^{24} + (-1)^{24} = 4^{24} + 4^{24} = 1+1 = 2 \cr
(1,4) &:& 3^{24} + 0^{24} = 1 + 0 = 1 \cr
(2,\gamma) &:& (\gamma-1)^{24} + (2-1)^{24} = 1+1 = 2 {\rm ,\quad since:} \cr
&&\cr
&& (\gamma-1)^2 = (\gamma^2-2\gamma+1) = 3-2\gamma \cr
&& (\gamma-1)^4 = (3-2\gamma)^2 = (9-12\gamma+4\gamma^2) = 2-2\gamma \cr
&& (\gamma-1)^8 = (2-2\gamma)^2 = (4-8\gamma+4\gamma^2) = 2-3\gamma \cr
&& (\gamma-1)^{12} = (2-2\gamma)(2-3\gamma) = (4-10\gamma+6\gamma^2) = 1 \cr
\end{eqnarray*}

In the final series of calculations, I used $\gamma^2=2$ and reduced
mod 5 repeatedly.  I think the pattern should be clear, and leave
further verification as an exercise.

\endexample


\vfill\eject
\mysection{Hermite reduction}

\cite{trager} shows how to extend Hermite reduction into the algebraic case.
This offers a means of calculating the rational parts of the integrals
without going through the calculations described in this chapter.  In
particular, Puiseux expansions can be completely avoided, which offers
savings in computational complexity.
