
\mychapter{Notes}

For a while I was thinking that the product of {\it prime} ideals is
equal to their {\it intersection}.  This is true in principal ideal
domains, but not in general.

For example, consider $I=(x,z)$ and $J=(x+z)$ in $K[x,z]$.
Now, $x+z \in I \cap J$, but $x+z \notin I \cdot J$.

Sage was useful in puzzling this out:

\begin{sageblock}[notes]
R.<x,z> = QQ[];

I = Ideal(x,z);
J = Ideal(x+z);

I.is_prime()
J.is_prime()

I.intersection(J) == I*J
\end{sageblock}

Looking at the primary decomposition, we see that the product is
smaller than the intersection, because there's an extra ideal that
needs to be intersected (the original ideal is the intersection of the
ideals in its primary decomposition).  Sage's comparison operator
for ideals also shows us that the product is contained in the
intersection.

\begin{sageblock}[notes]
I.intersection(J).primary_decomposition()
(I*J).primary_decomposition()

I.intersection(J) > I*J
\end{sageblock}

So now it's a question of finding something in the intersection that
isn't in the product.  The ideal quotient isn't useful for this,
probably because of its Zariski closure property.

\begin{sageblock}[notes]
I.intersection(J).quotient(I*J)
\end{sageblock}

\vfill\eject

\begin{maximablock}
diff(sqrt(x^4+1),x);
diff(%,x);
diff(%,x);
diff(%,x);

diff(sqrt(x^3+1),x);
diff(%,x);
diff(%,x);
\end{maximablock}

\vfill\eject


\mysection{Valuations}
\qquad [van der Waerden], \S18.1

A {\it valuation} is a generalization of the absolute value.  A {\it
valuation} is a mapping $\phi$ from a field ${\bf K}$ to an ordered
field ${\cal R}$ (typically the reals) obeying the following axioms:

\begin{center}
\begin{supertabular}{l l l r}
   positivity	& $\forall a \in {\bf K},$ & $\phi(a) \ge 0$ &(V1)\cr
   definiteness & $\forall a \in {\bf K},$ & $\phi(a) > 0 \Longleftrightarrow a \ne 0$ &(V2)\cr
   homomorphism (on the multiplicative group) & $\forall a,b \in {\bf K},$ & $\phi(ab) = \phi(a)\phi(b)$ &(V3)\cr
   subadditivity (or triangle inequality) & $\forall a,b \in {\bf K},$ & $\phi(a+b) \le \phi(a) + \phi(b)$ &(V4)\cr
\end{supertabular}
\end{center}

A moment's thought will show that the standard absolute value on the
reals obeys these axioms, as does the modulus on the complex field.
Valuations are similar to norms, except that norms are defined on
vector spaces, while valuations are defined on fields.

A valuation is said to be {\it non-Archimedian} if it also satisfies
the following axiom, stronger than V4:

\begin{center}
\begin{supertabular}{l l l r}
   non-Archimedian axiom & $\forall a,b \in {\bf K},$ & $\phi(a+b) \le \max(\phi(a), \phi(b))$ &(V4')\cr
\end{supertabular}
\end{center}

In this case, we can switch from a multiplicative to an additive
notation and obtain {\it exponential valuation} by replacing $\phi(a)$
with $w(a) = -\ln \phi(a)$:

\begin{center}
\begin{supertabular}{l l l r}
   & $\forall a \in {\bf K},$ & $w(a) \in (-\infty, \infty]$ &(E1)\cr
   & $\forall a \in {\bf K},$ & $w(a) = \infty \Longleftrightarrow a = 0$ &(E2)\cr
   & $\forall a,b \in {\bf K},$ & $w(ab) = w(a) + w(b)$ &(E3)\cr
   & $\forall a,b \in {\bf K},$ & $w(a+b) \ge \min(w(a), w(b))$ &(E4)\cr
\end{supertabular}
\end{center}

\vfill\eject
\mysection{Notes on Harris - Geometry of Alg Curves - Harvard 287}

Abel's theorem -- p. 29

Classical Jacobian discussed on pp. 28-31

``As we’ve defined it, the Jacobian is only a complex torus so far. Note that a
general complex torus is not embeddable in projective space. However, it turns
out that the Jacobian has enough meromorphic functions to embed in projective
space, so it is a projective variety.''


\vfill\eject
\mysection{Notes on [Fu08]}

[Fu08] is a good, freely available introduction to algebraic geometry.

{\small\begin{verbatim}


Riemann-Roch Theorem

Let C be an algebraic curve, let X be its non-singular model, and let
K be its function field.

Proposition 8.4.  Let x \in K, x \notin k. Let (x)_0 be the divisor
of zeros of x and let n=[K:k(x)].  Then

  1) (x)_0 is an effective divisor of degree n,
  2) There is a constant \tau such that l(r(x)_0) \ge rn-\tau \forall r.

Proof

Prop 6.9. K is an algebraic function field in one variable over k.  By
definition, this means that exists some t such that K is algebraic
over k(t).  So x \in K is algebraic over k(t), and \exists F \ in
k[X,T] such that F[x,t] = 0.  x is not algebraic over k (1-48), so t
must appear in F, so t is algebraic over k(x), and therefore k(x,t) is
algebraic over k(x) (1-50), so K is algebraic over k(x) (1-46).



Problem 1-54: If R is a domain with quotient field K, and L is a
finite algebraic extension of K, then there exists a basis for L over
K such that each basis element is integral over R.

Proof 1-54: Let {w_1, ..., w_n} be any basis for L over K.  Since each
basis element is algebraic over K, by clearing denominators we can
write:

a_{i0} w_i^{n_i} + a_{i1} w_{n_i-1} + \cdots = 0       a_{ij} \in R

We can pull a_{i0} into w_i, and thus adjust the w_i's to be integral
over R by multiplying each one by something in R.  Since anything
in L can be written

   l = \sum c_i w_i        c_i \in K

it can also be written

   l = \sum (c_i / r_i) w'_i

where the r_i adjust the w_i to be integral and c_i/r_i is still in K.



INTEGRAL ELEMENTS: w integral over k[x] means that w is finite
everywhere x is, and has poles only where x does.  w integral over
k[x^{-1}] means that w is finite everywhere x^{-1} is, and has poles
only where x has zeros.

So, k[x^{-1}] is a domain with quotient field k(x), and K is a finite
algebraic extension of k(x), so there exists a basis for K over k(x)
such that each basis element is integral over k[x^{-1}].

Let {w_1, ..., w_n} be such a basis for K over k(x).  We will show
that the poles of these functions must lie over the roots of x.

w_i^{n_i} + a_{i1} w_{n_i-1} + \cdots = 0       a_{ij} \in k[x^{-1}]

So, ord_P(a_ij) \ge 0 if P \ne S (zero set of x), since x^{-1} and
thus anything in k[x^{-1}] is finite away from S.



Problem 2-29: if for some i, ord(a_i) < ord(a_j) \forall j \ne i,
then a_1 + \cdots + a_n \ne 0

Proof of 2-29: Assume the contrary.  Then we can write a_i =
\sum_{i\ne j} a_j.  Taking ord of both sides, and using ord(a+b) \ge
min(ord(a), ord(b)), we see this is impossible.



Therefore, ord_P(w_i) \ge 0 if P \ne S, since otherwise
ord_P(w_i^{n_i}) < ord_P(a_ij w^{n_i-j}) \forall j.

Therefore, the poles of w_i are isolated at the zeros of x, and since
there are only a finite number of w_i and each has a finite number of
poles, then for some t, div(w_i) + tZ > 0 \forall i.

So w_i \in L(tZ), and if j \le r, then w_i x^{-j} \in L((r+t)Z)

Now w_i are independent over k(x) and 1, x^{-1}, ..., x^{-r} are
independent over k, so l((r+t)Z) \ge n(r+1).

Now, l((r+t)Z) = l(rZ) + dim(L((r+t)Z) / L(rZ))

dim(L((r+t)Z) / L(rZ)) \le tm (Prop 3-1), where m is the degree of Z.

So, l(rZ) \ge n(r+1) - tm, so pick \tau = tm-n, and

l(rZ) \ge nr - \tau, \forall r




Riemman-Roch

l(D) = deg(D) + 1 - g + l(W-D)

deg(W) = 2g-2    l(W) = g

l(0) = 1

l(D) = 0 if deg(D) < 0

If deg(D) > 2g-2, then l(D) = deg(D) + 1 - g

If deg(D) = 2g-2, then deg(W-D) = 0, and l(W-D) = 1 iff D-W is principal, otherwise l(W-D) = 0

Let D=W+X, where deg(X)=0, then l(D) = 2g-2 + 1 - g + (1/0) depending on whether X is principal
   l(D) = g (X is principal) or l(D) = g - 1 (X is not principal)


Goal: an g-dimensional algebraic variety that represents Pic0

Consider (2g-2)-dimensional symmetric space.  Each point corresponds to an effective
divisor of degree (2g-2).

Fix a (g-2)-tuple.  We're left with g free points.


Milne's construction

Use r-dimensional symmetric space, with r > 2g-2.  Pick an (r-g) tuple.

\end{verbatim}
}

\vfill\eject
\mysection{Notes on [Sh61]}

{\small\begin{verbatim}

A function (Y -> k) is regular at a point on a variety if there exists
an open neighborhood of the point where the function is given by a
rational function of polynomials, the denominator never zero.
(relation to coordinate ring?)

A map (Y -> k) is regular on Y if it is regular at every point of Y.

A morphism (X -> Y) between varieties is a continuous map (in the
Zariski topology) such that pullbacks of regular functions are
regular.

A rational map is a morphism defined only on an open subset.

Given a rational map from affine varieties X to Y, if [x,y,z] is Y's
coordinate system, then we pullback to a regular function, which is a
rational function in X's function field.  So Y's coordinates are given
by rational functions in X's coordinates.

A group variety is an algebraic variety equipped with a group
structure, where the group operation and group inversion are rational
maps.  If the group operation is commutative, it's an abelian variety.

A homomorphism between abelian varieties is a rational map that
commutes with the group operation.

An endomorphism is a homomorphism from the variety to itself.

Endomorphisms of a group form a ring.  Addition is performed by
mapping through both endomorphisms, then applying the group operation,
which is commutative, so endomorphism addition is commutative.
Multiplication is performed by composition, and need not be
commutative.

Using just the addition structure, we get an abelian group that can be
structued as a Z-module.  Multiplication by an integer is just
repeated application of the endomorphism.

We can promote the endomorphism ring into an algebra by tensoring with
Q, call this EndQ(A).  Elements of EndQ(A) are basically endomorphisms
with an associated 1/n denominator (numerators can be sucked into the
endomorphism).

A lattice is a free Z-module of the same rank as the algebra over Q.

An order is a lattice that is also a subring and contains the identity.

The order of A, written t, is the image of the endomorphism ring in
the endomorphism algebra.

a is a lattice in the endomorphism algebra contained in t, so it's a
collection of actual endomorphisms.

g(a,A) is the set of points on A mapped to 0 by every element of a.

Prop 16. Let a be an integral ideal (p. 49) of F.  Reduction mod p
defines a homomorphism of g(a,A) onto g(a,A^).  If a is prime to the
characteristic of k^, this homomorphism is an isomorphism.


Div(C) is group of divisors
Div0(C) is group of degree 0 divisors
P(C) is group of principal divisors

P(C) in Div0(C) in Div(C)

define Pic(C) = Div(C)/P(C)
define Pic0(C) = Div0(C)/P(C)

Pic0(C) is isomorphic to the Jacobian variety with a point O.

Given a degree zero divisor in Div0(C), we can use the group law on
the Jacobian to construct a single point corresponding to the divisor.
Asking if the divisor is principal is asking if this point is O.
Asking if any multiple of the divisor is principal is asking if any
multiple of this point is O.

We can define an endomorphism to be addition by a point, using the
group law.  NO - doesn't take identity to identity.

Let's consider [n], the endomorphism defined by applying the group
operation n times (n is an integer).  This should generate a lattice
contained in t, so it's an integral ideal.  g([n], A) is the set of
points whose n-multiples are principal.

Given a divisor whose (k p^q)-multiple is principal, let's multiply by
p^q and get a divisor whose k-multiple is principal.  Endomorphism
ideal [k] is prime to p.  Then this divisor point will be in g(a,A)
and g(a,A^), where a is [k].

We can determine that the divisor point is in g(a,A^) for a=[k] by
determining that the divisor's k-multiple is principal on the module
curve.  Since this is an isomorphism, the divisor point is also in
g(a,A) for a=[k].




Given a non-singular algebraic curve C, we reduce mod p to get Cp.
Good reduction implies that Cp is non-singular with the same genus.  C
has an associated Jacobian J.  Cp also has an associated Jacobian Jp.

PROBLEM: (hopefully) Show that J mod p is Jp.

Construct J as follows.  Pick r > 2g-2.  Symmetric group J^(r).  Pick
r-g extra points and find an open covering of J^(r).  Within each
open set, construct J locally.

\end{verbatim}
}
