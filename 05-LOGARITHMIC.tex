
\mychapter{The Logarithmic Extension}

\vfill\eject

\example Compute $\int {1\over{x \ln x}} {\rm d}x$

Operating in ${\bf C}(x, \theta = \ln x)$, we evaluate:

$$\int {1\over{\theta x}} = \int {{1\over x}\over{\theta}} = \int {{\theta '}\over{\theta}} = \ln \theta$$
$$\int {1\over{x \ln x}} {\rm d}x = \ln \ln x$$

\endexample

\example Compute $\int \ln x \,{\rm d}x$

Again we'll use ${\bf C}(x, \theta = \ln x)$

$$\int \theta = a\theta^2 + b\theta + \bar{c}$$

Differentiating both sides and collecting terms, we get:

$$\theta = a'\theta^2 + (2a\theta' + b')\theta + (b\theta' + \bar{c}')$$

Equating like terms:

$$a' = 0$$
$$2a\theta' + b' = 1 \qquad\Longrightarrow\qquad 2a\theta + b = \int 1 = x + d$$
$$a = 0 \qquad b=x+d$$
$$b\theta' + \bar{c}' = 0 \qquad\Longrightarrow\qquad 1 + {d\over x} + \bar{c}' = 0$$
$$c = -x \qquad d = 0$$
$$\int \theta = x\theta - x$$
$$\int \ln x \,{\rm d}x = x \ln x - x$$

\endexample

\vfill\eject

\example Compute $\int \tan^{-1} x\, {\rm d}x$

The ``standard'' approach to this integral would be to note that the
derivative of $\tan^{-1} x$ is $1/(x^2+1)$ and use integration by parts:

$$\int \tan^{-1} x\, {\rm d}x = x\, \tan^{-1}x - \int \frac{x\, {\rm d}x}{x^2+1}
= x\, \tan^{-1}x - \ln(x^2+1) + C$$

Using the differential algebra identity $\tan^{-1} x =
{1\over2}\,i\,\ln {{ix-1}\over{ix+1}}$, we use the differential field
${\mathbb C}(x,\theta)$; $\theta = \ln {{ix-1}\over{ix+1}}$;
$\theta' = - \frac{2i}{x^2+1}$ and compute

% {1\over2}\,i\,\ln {{ix-1}\over{ix+1}}

$$\int {1\over2}\,i\,\theta\, {\rm d}x$$

Because our integrand has no denominator in $\theta$, Theorem
\ref{basic logarithmic properties} tells us both that our
integral can have no denominator in $\theta$ and that its
$\theta$-degree can be at most two, so it must have the form

$$a_2\theta^2 + a_1\theta + a_0\qquad a_2,a_1 \in {\mathbb C}(x);\;
a_0 \in {\mathbb C}(x,\ldots)$$

Now let's differentiate and equate like terms in $\theta$.  First,

$$a_2' = 0$$

so $a_2$ must be a constant, say $c_2$.  Next,

$$c_2\theta' + a_1' = {1\over2}\,i$$

Integrating both sides we obtain:

$$c_2\theta + a_1 = {1\over2}\,i\,x + c_1$$

Equating coefficients again (and remembering that
$a_1 \in {\mathbb C}(x)$) leads us to conclude that

$$c_2=0 \qquad a_1 = {1\over2}\,i\,x + c_1$$

Finally,

$$a_1\theta' + a_0' = 0$$
$$\left({1\over2}\,i\,x + c_1\right)\theta' + a_0' = 0$$

Since $c_1$ is unknown, we'll leave it on the left-hand side along
with the unknown $a_0$ and move all of our knowns to the right-hand
side, taking advantage of our knowledge of $\theta'$:

$$c_1\theta' + a_0' = {1\over2}\,i\,x \frac{2i}{x^2+1}$$
$$c_1\theta' + a_0' = - \frac{x}{x^2+1}$$

Integrating both side we obtain:

$$c_1\theta + a_0 = - \ln(x^2+1)$$

and remembering that $a_0$ can include new logarithmic extensions, we
conclude that

$$c_1 = 0 \qquad a_0 = - \ln(x^2+1)$$

and therefore our integral is:

$${1\over2}\,i\,x \theta - \ln(x^2+1) = x\,\tan^{-1} x - \ln(x^2+1)$$


\endexample

\example Determine if $\sum_{n=0}^\infty \frac{1}{n^2} x^n$ is Liouvillian.

We can differentiate the series term wise, and if the resulting series
can be identified with a Liouvillian function, then we need only to
decide if that derivative can be integrated into a Liouvillian
function.  Using a standard identity for the Taylor series of $\ln
(1-x)$, we determine that

$$\frac{\ud}{\ud x} \left[ \sum_{n=0}^\infty \frac{1}{n^2} x^n \right] = \sum_{n=1}^\infty \frac{1}{n} x^{n-1} = \frac{1}{x}\ln (1-x)$$

so we need to determine if $\int \frac{1}{x}\ln (1-x) \ud x$ is Liouvillian.  Working in the
differential field ${\mathbb C}(x,\theta = \ln (1-x))$, we're trying to integrate
$\frac{1}{x}\theta \ud x$.  So, our integral must have the form

$$I = c_2 \theta^2 + a_1 \theta + \bar{a_0}$$

$$I' = \left[ c_2 \frac{1}{1-x} + a_1' \right] \theta + \left[ a_1 \frac{1}{1-x} + \bar{a_0}' \right] = \frac{1}{x}\theta$$

$$c_2 \frac{1}{1-x} + a_1' = \frac{1}{x}$$

$$a_1' = \frac{1}{x} - c_2 \frac{1}{1-x}$$

$\ln x$ is required to express $a_1$, but new logarithms are not
allowed at this point in the algorithm.  Therefore,
$\sum_{n=0}^\infty \frac{1}{n^2} x^n$ is not Liouvillian.

\endexample

\vfill\eject

$$\int{{x(x+1)\{(x^2e^{2x^2}-\ln^2(x+1))^2+2xe^{3x^2}(x-(2x^3+2x^2+x+1)\ln(x+1))\}}\over{((x+1)\ln^2(x+1) - (x^3+x^2)e^{2x^2})^2}} dx$$

$$\psi = \exp(x^2) \qquad \theta = \ln (x+1)$$

$$\int{{x(x+1)\{(x^2 \psi^2-\theta^2)^2+2x \psi^3(x-(2x^3+2x^2+x+1)\theta)\}}\over{((x+1)\theta^2 - (x^3+x^2)\psi^2)^2}} dx$$

$$\int{{x\{(x^2 \psi^2-\theta^2)^2+2x \psi^3(x-(2x^3+2x^2+x+1)\theta)\}}\over{(x+1)(\theta^2 - x^2\psi^2)^2}} dx$$


{\small\begin{verbatim}

(%i235) divide(expand(x*((x^2*p^2-t^2)^2 + 2*x*p^3*(x-(2*x^3+2*x^2+x+1)*t))),
                expand((x+1)*(t^2-x^2*p^2)^2), t);

                      x            3  5      3  4      3  3      3  2       3  3
(%o235)            [-----, t (- 4 p  x  - 4 p  x  - 2 p  x  - 2 p  x ) + 2 p  x ]
                    x + 1

\end{verbatim}}

{\LARGE$$\int {x\over{x+1}} dx +  \int{{{{-4x^5-4x^4-2x^3-2x^2}\over{x+1}}\psi^3\theta + {{2x^3}\over{x+1}}\psi^3}\over{(\theta^2 - x^2\psi^2)^2}} dx$$}

\vfill\eject
{\LARGE$$\int {x\over{x+1}} dx +  \int{{{{-4x^5-4x^4-2x^3-2x^2}\over{x+1}}\psi^3\theta + {{2x^3}\over{x+1}}\psi^3}\over{(\theta^2 - x^2\psi^2)^2}} dx$$}

$$V = \theta^2 - x^2\psi^2 \qquad V' = {2\over{x+1}}\theta - (2x+4x^3)\psi^2$$

$${{{-4x^5-4x^4-2x^3-2x^2}\over{x+1}}\psi^3\theta + {{2x^3}\over{x+1}}\psi^3}
  = sV + tV'$$

{\small\begin{verbatim}

(%i353) v: t^2-x^2*p^2;

                                              2    2  2
(%o353)                                      t  - p  x
(%i354) vtick: (2/(x+1))*t-(2*x+4*x^3)*p^2;

                                       2 t     2     3
(%o354)                               ----- - p  (4 x  + 2 x)
                                      x + 1
(%i355) r: gcdex(v,vtick,t) * (((-4*x^5-4*x^4-2*x^3-2*x^2)/(x+1))*p^3*t + (2*x^3/(x+1))*p^3);

                   3         5      4      3      2       3  3
                  p  t (- 4 x  - 4 x  - 2 x  - 2 x )   2 p  x
                  ---------------------------------- + -------
                                x + 1                   x + 1
(%o355)/R/ [---------------------------------------------------------,
             4     8      7      6      5      4      3    2     2  2
            p  (4 x  + 8 x  + 8 x  + 8 x  + 5 x  + 2 x  + x ) - p  x

     3      2             2        7      6      5      4      3      2       3
((2 x  + 2 x  + x + 1) p t  + ((4 x  + 8 x  + 8 x  + 8 x  + 5 x  + 2 x  + x) p  - x p) t

         5      4    3    2   3       6      5      4      3      2             2
 + (- 2 x  - 2 x  - x  - x ) p )/((4 x  + 8 x  + 8 x  + 8 x  + 5 x  + 2 x + 1) p  - 1),

      5      4      3      2   3        3  3
  (4 x  + 4 x  + 2 x  + 2 x ) p  t - 2 x  p
- ------------------------------------------]
                    x + 1
(%i356) ratsimp(r[1] - divide(r[1],vtick,t)[1]*vtick);

                                                2 p x
(%o356)                                       - -----
                                                x + 1
(%i357) ratsimp(r[2] + divide(r[1],vtick,t)[1]*v);

(%o357)                                        p t x

\end{verbatim}}

$${{{-4x^5-4x^4-2x^3-2x^2}\over{x+1}}\psi^3\theta + {{2x^3}\over{x+1}}\psi^3}
  = {-2x\over{x+1}}\psi V + x\psi\theta V'$$



$$\int {N\over{V^n}} = {A\over{V^{n-1}}} + \int {B\over{V^{n-1}}}$$

$${N\over{V^n}} = {A'\over{V^{n-1}}} - (n-1){AV'\over{V^n}} + {B\over{V^{n-1}}}$$

$$N = VA' - (n-1)AV' + BV$$
$$N = (A'+B)V - (n-1)AV'$$

$$A = -x\psi\theta$$
$$A' = -\psi\theta - x(2x)\psi\theta - x\psi{1\over{x+1}}$$
$$A' + B = {-2x\over{x+1}}\psi$$
$$B = (2x^2+1)\psi\theta - {x\over{x+1}}\psi$$

{\LARGE$$\int {x\over{x+1}} dx  - {x\psi\theta\over{\theta^2-x^2\psi^2}} +  \int{{(2x^2+1)\psi\theta - {x\over{x+1}}\psi}\over{\theta^2-x^2\psi^2}}$$}

{\LARGE$$x - \theta - {x\psi\theta\over{\theta^2-x^2\psi^2}} +  \int{{(2x^2+1)\psi\theta - {x\over{x+1}}\psi}\over{\theta^2-x^2\psi^2}}$$}

\vfill\eject

{\LARGE$$x - \theta - {x\psi\theta\over{\theta^2-x^2\psi^2}} +  \int{{(2x^2+1)\psi\theta - {x\over{x+1}}\psi}\over{\theta^2-x^2\psi^2}}$$}

{\small\begin{verbatim}

(%i374) resultant((2*x^2+1)*p*t-(x/(x+1))*p-z*vtick,v,t);

         2  2     2  6      2  5      2  4      2  3      2  2      2      2          2
(%o374) p  x  (4 p  x  + 8 p  x  + 8 p  x  + 8 p  x  + 5 p  x  + 2 p  x + p  - 1) (4 z  - 1)
(%i375) gcd((2*x^2+1)*p*t-(x/(x+1))*p-(1/2)*vtick,v);

                                              p x + t
(%o375)                                       -------
                                               x + 1
(%i376) gcd((2*x^2+1)*p*t-(x/(x+1))*p+(1/2)*vtick,v);

                                              p x - t
(%o376)                                       -------
                                               x + 1
\end{verbatim}}

{\LARGE$$x - \theta - {x\psi\theta\over{\theta^2-x^2\psi^2}} + {1\over2}\ln({{\theta + x \psi}\over{x+1}}) - {1\over2}\ln({{\theta - x \psi}\over{x+1}})$$}

% \vskip 2in

$$\int{{x(x+1)\{(x^2e^{2x^2}-\ln^2(x+1))^2+2xe^{3x^2}(x-(2x^3+2x^2+x+1)\ln(x+1))\}}\over{((x+1)\ln^2(x+1) - (x^3+x^2)e^{2x^2})^2}} dx$$

{\LARGE$$= x - \theta - {x\psi\theta\over{\theta^2-x^2\psi^2}} + {1\over2}\ln(\theta + x \psi) - {1\over2}\ln(\theta - x \psi)$$}

$$= x - \ln(x+1) - {x e^{x^2}\ln(x+1)\over{\ln^2(x+1)-x^2 e^{2x^2}}}$$
$$+ {1\over2}\ln\left[\ln(x+1) + x e^{x^2}\right] - {1\over2}\ln\left[\ln(x+1) - x e^{x^2}\right]$$
