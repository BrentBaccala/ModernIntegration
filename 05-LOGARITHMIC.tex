
\mychapter{The Logarithmic Extension}

When we express a function in Liouvillian form, we construct
a tower of nested fields, starting with ${\mathbb C}(x)$
at the bottom, and building up to the extension required
to express our integral.

For example, consider example \ref{double-log integral}:

$$\int \left[ (\ln(\ln x))^2 \ln x - \frac{2}{\ln x}(\ln(\ln x) +1) \right] dx$$

To express this integral, we need to construct $\ln (\ln x)$, which
requires $\ln x$ to be constructed first.  Thus, we obtain nested
fields according to the following structure:

% Define block style
\tikzstyle{extension} = [draw, fill=blue!20, text width=8em, text centered,
  minimum width=10em, minimum height=3em, rounded corners]

\begin{center}
\begin{tikzpicture}

\node (p3) [extension, label={[name=p2] ${\bf C}(x,\theta) \qquad \theta = \ln x$}] {${\bf C}(x)$};
\node [fit=(p3)(p2), draw, rounded corners, label={[name=p1] ${\bf C}(x,\theta,\psi) \qquad \psi = \ln \theta$}] {};
\node [fit=(p3)(p2)(p1), draw, rounded corners, inner sep=8pt] {};

\end{tikzpicture}
\end{center}

Our integrand can now be expressed as a rational function in the
top-most field:

$$\int \frac{\theta^2 \psi^2 - 2 \psi - 2}{\theta} dx$$

Our basic strategy for integration is to always work in the top most
field extension, reducing to some kind of problem that must be solved
in the next lower extension.  Since we only have three basic types of
field extension, our aim is to develop a theory to handle each type.

In this chapter, we'll analyze the logarithmic extension.

For logarithmic extensions, the problem is particularly easy,
since integration leads only to futher integration steps
in the base field.

\vfill\eject
\section{The Logarithmic Integration Theorem}

\theorem\label{logarithmic integration theorem}
Let $K$ be a differential field, let $K(\theta = \ln k)$ be a simple
logarithmic extension of $K$, let $n_i(\theta)$ be
irreducible polynomials in $K[\theta]$,
and let $f$ be an element of $K(\theta)$
with partial fractions expansion:

\begin{equation}
\label{logarithmic integration theorem - integrand}
f = \sum_{i=0}^n a_i \theta^i
+ \sum_{i=1}^\nu \sum_{j=1}^{m_i} \frac{b_{i,j}(\theta)}{n_i(\theta)^j}
\end{equation}

$$a_i \in K \qquad b_{i,j}(\theta),n_i(\theta) \in K[\theta]$$

If $f$ has
an elementary anti-derivative $F$, then $F \in K(\theta, \Psi)$,
where $K(\theta, \Psi)$ is a finite logarithmic extension
of $K(\theta)$, $F$ has a partial fractions expansion of the form:

% Needed to squeeze this theorem onto one page
% \vskip -10pt

\begin{equation}
\label{logarithmic integration theorem - integral}
F = c_{n+1} \theta^{n+1} + \sum_{i=1}^{n} \left[ A_i + c_i \right] \theta^i + A_0
+ \sum_{i=1}^\nu \sum_{j=1}^{m_i-1} \frac{B_{i,j}(\theta)}{n_i(\theta)^j}
+ \sum_{i=1}^\nu C_i \ln n_i(\theta)
\end{equation}

%% $$\overbar{A_0} \in K[\Psi] \qquad {A_i \in K}_{i \ne 0} \qquad B_{i,j}(\theta),n_i(\theta) \in K[\theta] \qquad c_i' = C_i' = 0$$
\begin{equation*}
\arraycolsep=10pt\def\arraystretch{0}
\begin{array}{cccc}
A_0 \in K(\Psi) & A_i \in K & B_{i,j}(\theta),n_i(\theta) \in K[\theta] & c_i' = C_i' = 0 \\
& {}_{i \ne 0} & &
\end{array}
\end{equation*}

and the following relationships hold:

\begin{subequations}
\begin{eqnarray}
\label{eq: logarithmic An}
\left[ A_n + (n+1) c_{n+1} \theta \right]' &=& a_n \\
\label{eq: logarithmic Ai's}
\left[ A_i + (i+1) c_{i+1} \theta \right]' &=& a_i - (i+1) \frac{k'}{k} A_{i+1} \qquad {}_{0\le i<n}
\end{eqnarray}
\end{subequations}

\begin{comment}
\begin{IEEEeqnarray*}{rClc}
R_{i,m_i-1}(\theta) &=& b_{i,m_i}(\theta) & \IEEEyessubnumber \\*[10pt]
R_{i,j}(\theta) &=& b_{i,j+1}(\theta) - B_{i,j+1}'(\theta) - Q_{i,j+1}(\theta) \qquad{}_{1\le j<m_i-1}&  \IEEEyessubnumber \\
B_{i,j}(\theta) &\equiv& - \frac{R_{i,j}(\theta)}{j n_i'(\theta) } \mod n_i(\theta) &  \IEEEyesnumber \\*[10pt]
Q_{i,j}(\theta) &=& - \frac{R_{i,j}(\theta) + j B_{i,j}(\theta) n_i'(\theta)}{n_i(\theta)} &  \IEEEyesnumber
\end{IEEEeqnarray*}
\end{comment}

\begin{subequations}
\begin{eqnarray}
R_{i,m_i-1}(\theta) &=& b_{i,m_i}(\theta) \\
R_{i,j}(\theta) &=& b_{i,j+1}(\theta) - B_{i,j+1}'(\theta) - Q_{i,j+1}(\theta) \qquad{}_{1\le j<m_i-1}
\end{eqnarray}
\end{subequations}

%\begin{IEEEeqnarray}{rCl}
\begin{eqnarray}
B_{i,j}(\theta) &\equiv& - \frac{R_{i,j}(\theta)}{j n_i'(\theta) } \mod n_i(\theta)  \\*[5pt]
Q_{i,j}(\theta) &=& - \frac{R_{i,j}(\theta) + j B_{i,j}(\theta) n_i'(\theta)}{n_i(\theta)}
\end{eqnarray}
%\end{IEEEeqnarray}

\begin{comment}
\begin{IEEEeqnarray*}{rCl?c}
\IEEEeqnarraymulticol{3}{c}{m_i > 1} & m_i = 1 \\*[10pt]
C_i &=& \frac{b_{i,1}(\theta) - B_{i,1}'(\theta) - Q_{i,1}(\theta)}{n_i'(\theta)} & C_i = \frac{b_{i,1}(\theta)}{n_i'(\theta)} \IEEEyesnumber
\end{IEEEeqnarray*}
\end{comment}

% IEEEeqnarray respects the settings of the environment, and we're
% in a theorem environment with italic text.  Change this behavior
% back to the default as described here:
%
%    http://moser-isi.ethz.ch/docs/typeset_equations.pdf (page 11)

\renewcommand{\theequationdis}{{\normalfont (\theequation)}}
\renewcommand{\theIEEEsubequationdis}{{\normalfont (\theIEEEsubequation)}}

\begin{subequations}
\begin{IEEEeqnarray}{rCl?r}
C_i &=& \frac{b_{i,1}(\theta) - B_{i,1}'(\theta) - Q_{i,1}(\theta)}{n_i'(\theta)} & {}_{m_i > 1} \\
C_i &=& \frac{b_{i,1}(\theta)}{n_i'(\theta)} & {}_{m_i = 1}
\end{IEEEeqnarray}
\end{subequations}


\proof

By Theorem \ref{weak Liouville theorem}, an elementary antiderivative
of $f$ can only exist in a finite logarithm extension $K(\theta, \Psi)$
of $K(\theta)$ and therefore must have the form:

$$F = R + \sum_{i=1}^\eta C_i \Psi_i$$

where $R \in K(\theta)$, the $C_i$ are constants, and the $\Psi_i$ are logarithms.

We can perform a partial fractions expansion on $R$:

$$F = \sum_{i=0}^N A_i \theta^i
+ \sum_{i=1}^\nu \sum_{j=1}^{M_i} \frac{B_{i,j}(\theta)}{n_i(\theta)^j}
+ \sum_{i=1}^\eta C_i \Psi_i$$

It will be convenient later to separate a constant from the $A_i$ terms,
so let's do that now:

$$F = \sum_{i=0}^N (A_i + c_i) \theta^i
+ \sum_{i=1}^\nu \sum_{j=1}^{M_i} \frac{B_{i,j}(\theta)}{n_i(\theta)^j}
+ \sum_{i=1}^\eta C_i \Psi_i$$

Our basic
logarithmic relationships:

$$\ln ab = \ln a + \ln b \qquad\qquad \ln\frac{a}{b} = \ln a - \ln b$$

allow us to assume, without loss of generality, that the $\Psi_i$'s
are logarithms of irreducible polynomials.  Some of those irreducible
polynomials will exist solely in our underlying field $K$, and those
we collapse into $A_0$, noting that this makes $A_0$ unique among the
$A_i$'s because it can include additional logarithms.  The remaining
irreducible polynomials (those involving $\theta$) can be identified
as $n_i(\theta)$'s, simply by increasing $i$ and adding new
$n_i(\theta)$'s if needed.

Now let's differentiate, remembering that $\theta' = \frac{k'}{k}$:

$$F' = \sum_{i=0}^N \left[ A_i' \theta^i + i \frac{k'}{k} (A_i + c_i) \theta^{i-1} \right]
+ \sum_{i=1}^\nu \sum_{j=1}^{M_i} \frac{B_{i,j}'(\theta) n_i(\theta) - j B_{i,j}(\theta) n_i'(\theta)}{n_i(\theta)^{j+1}}
  + \sum_{i=1}^\nu C_i \frac{n_i'(\theta)}{n_i(\theta)}$$

\begin{comment}
$$F' = A_0' + \sum_{i=1}^N ( A_i' \theta^i + i \frac{k'}{k} (A_i + c_i) \theta^{i-1} )
+ \sum_{i=1}^\nu \sum_{j=1}^{M_i} \frac{B_{i,j}'(\theta) n_i(\theta) - j B_{i,j}(\theta) n_i'(\theta)}{n_i(\theta)^{j+1}}
  + \sum_{i=1}^\nu C_i \frac{n_i'(\theta)}{n_i(\theta)}$$
\end{comment}

\begin{comment}
\begin{multline*}
F' = \sum_{i=0}^{N-1} \left[ A_i' + (i+1) \frac{k'}{k} A_{i+1} \right] \theta^{i} + A_N' \theta^N \\
+ \sum_{i=1}^\nu \sum_{j=1}^{M_i} \frac{B_{i,j}'(\theta) n_i(\theta) - j B_{i,j}(\theta) n_i'(\theta)}{n_i(\theta)^{j+1}}
  + \sum_{i=1}^\nu C_i \frac{n_i'(\theta)}{n_i(\theta)}
\end{multline*}
\end{comment}

\begin{multline*}
F' = A_N' \theta^N + \sum_{i=0}^{N-1} \left[ A_i' + (i+1) \frac{k'}{k} (A_{i+1} + c_{i+1}) \right] \theta^{i} \\
+ \sum_{i=1}^\nu \left[ \frac{- M_i B_{i,M_i}(\theta) n_i'(\theta)}{n_i(\theta)^{M_i+1}} + \sum_{j=1}^{M_i-1} \frac{B_{i,j+1}'(\theta) - j B_{i,j}(\theta) n_i'(\theta)}{n_i(\theta)^{j+1}}
  + C_i \frac{n_i'(\theta)}{n_i(\theta)} \right]
\end{multline*}

$F'$ has the form of a partial fractions decomposition, but it is not
a partial fractions decomposition because the numerators in the $B$
terms violate the partial fractions degree bounds.  The problem
is that the product $B_{i,j}(\theta) n_i'(\theta)$ might
have degree greater than $\deg_\theta n_i(\theta)$.
To fix this, let's
divide the $-jB_{i,j}(\theta)n_i'(\theta)$ terms by $n_i(\theta)$
(think polynomial long division) and rewrite them as a quotient
and a remainder:

$$ - j B_{i,j}(\theta) n_i'(\theta) = Q_{i,j}(\theta) n_i(\theta) + R_{i,j}(\theta)$$

This fixes the $B$ terms, since $\deg Q_{i,j}(\theta) < \deg n_i(\theta)$
and $\deg R_{i,j}(\theta) < \deg n_i(\theta)$.

\begin{comment}
\begin{multline*}
F' = A_N' \theta^N + \sum_{i=0}^{N-1} \left[ A_i' + (i+1) \frac{k'}{k} (A_{i+1} + c_{i+1} \right] \theta^{i} \\
+ \sum_{i=1}^\nu \left[ \sum_{j=1}^{M_i} \frac{B_{i,j+1}'(\theta) + Q_{i,j+1}(\theta) + R_{i,j}(\theta)}{n_i(\theta)^{j+1}}
  + \frac{B_{i,1}'(\theta) + Q_{i,1}(\theta) + C_i n_i'(\theta)}{n_i(\theta)} \right]
\end{multline*}
\end{comment}

\begin{multline*}
F' = A_N' \theta^N + \sum_{i=0}^{N-1} \left[ A_i' + (i+1) \frac{k'}{k} (A_{i+1} + c_{i+1})\right] \theta^{i} \\
+ \sum_{i=1}^\nu \left[ \frac{R_{i,M_i}(\theta)}{n_i(\theta)^{M_i+1}}
+ \sum_{j=1}^{M_i-1} \frac{B_{i,j+1}'(\theta) + Q_{i,j+1}(\theta) + R_{i,j}(\theta)}{n_i(\theta)^{j+1}}
  + \frac{B_{i,1}'(\theta) + Q_{i,1}(\theta) + C_i n_i'(\theta)}{n_i(\theta)} \right]
\end{multline*}

What is the degree of $F'$?  It's $N$, if $A_N$ is not constant.  If $A_N$ is constant and not zero, then
the degree of $F'$ is $N-1$, since otherwise the $N-1$ coefficient would be zero:

$$A_{N-1}' + N \frac{k'}{k} A_N = 0  \qquad\Longrightarrow\qquad A_{N-1}' = (- N A_N ) \frac{k'}{k}$$

Since $A_N$ is constant, this could only be satisfied by $A_{N-1} = - N A_N \theta$,
contradicting the assumption that $A_{N-1} \in K$.

Performing a partial fractions decomposition of $f$:

$$f = \sum_{i=0}^n a_i \theta^i
+ \sum_{i=1}^\nu \sum_{j=1}^{m_i} \frac{b_{i,j}(\theta)}{n_i(\theta)^j}$$

setting $F' = f$ and equating like terms, we establish the
relationships listed in the statement of the theorem.

For the degree of $F'$ to be $n$, either $\deg_\theta F = n+1$ and $A_{n+1}$ is
constant, or $\deg_\theta F = n$.

Since the highest order denominators in $F'$ have order $M_i + 1$, and
they must match with $f$'s denominators of order $m_i$, we conclude
that $M_i = m_i - 1$.

To establish the remaining relationships, let's remember the definition of
$R_{i,j}$ and $Q_{i,j}$:

$$-j B_{i,j}(\theta)n_i'(\theta) = Q_{i,j}(\theta) n_i(\theta) + R_{i,j}(\theta)$$

Reducing this equation modulo $n_i(\theta)$, we obtain:

$$-j B_{i,j}(\theta)n_i'(\theta) \equiv R_{i,j}(\theta) \mod n_i(\theta)$$

Now we use the fact that $n_i(\theta)$ is {\it irreducible},
and invoke Theorem ??, which states the quotient ring
modulo a prime ideal is a field, so we can perform division:

$$ B_{i,j}(\theta) \equiv - \frac{R_{i,j}(\theta)}{jn_i'(\theta)} \mod n_i(\theta)$$

This equation seems to identify $B_{i,j}(\theta)$ up to a multiple of $n_i(\theta)$,
but if we remember our degree bound on partial fractions expansions,
$\deg_\theta B_{i,j}(\theta) < \deg_\theta n_i(\theta)$, we see
that in fact we've completely determined $B_{i,j}(\theta)$ from
$R_{i,j}(\theta)$.

\endtheorem

A few comments are in order.  First, let's recall equations
\eqref{eq: logarithmic An} and \eqref{eq: logarithmic Ai's}:

\begin{equation}
\left[ A_n + (n+1) c_{n+1} \theta \right]' = a_n
\tag{\ref{eq: logarithmic An}}
\end{equation}

\begin{equation}
\left[ A_i + (i+1) c_{i+1} \theta \right]' = a_i - (i+1) \frac{k'}{k} A_{i+1}
\tag{\ref{eq: logarithmic Ai's}}
\end{equation}

These equations both require integration of the right hand side in the
underlying field $K$.  The resulting integral must take the form of a
element of $K$ (that's the $A_i$), plus possibly a constant times a
single additional logarithm, $\theta$ itself.  Liouville's
theorem \ref{weak Liouville theorem} tells us that integrals can
generally include an arbitrary number of additional logarithms.  The
integrations required by
\eqref{eq: logarithmic An} and \eqref{eq: logarithmic Ai's}
are more restrictive; the only additional logarithm they can include
is $\theta$, which makes sense because $\theta$ is not part of the
underlying field $K$, but could easily appear in the final result
because it already appears in the original integrand.

The exception is $A_0$, since it can include arbitrary additional
logarithms, not just $\theta$.  It's version of equation
\eqref{eq: logarithmic Ai's} reads:

\begin{equation}
\left[ A_0 + c_1 \theta \right]' = a_0 - \frac{k'}{k} A_1
\end{equation}

So, this integral can include a logarithm $\theta$, just like the
others, as well as additional logarithms collapsed into the $A_0$
term.  The $c_0$ term, by the way, does not appear in any of the
theorem's equations, but a moment's thought shows that $c_0$ is
merely the constant of integration.

\vfill\eject

\example Compute $\int {1\over{x \ln x}} {\rm d}x$

Operating in ${\bf C}(x, \theta = \ln x)$, we evaluate:

$$\int {1\over{\theta x}} \ud x = \int {{1\over x}\over{\theta}} \ud x$$

This has the form of equation \eqref{logarithmic integration theorem - integrand}
with $n_1(\theta) = \theta$, $m_1 = 1$ and $b_{1,1}(\theta) = \frac{1}{x}$.

$$C_1 = \frac{b_{1,1}(\theta)}{n_1'(\theta)} = \frac{\frac{1}{x}}{\frac{1}{x}} = 1$$

Plugging $C_1$ into equation \eqref{logarithmic integration theorem - integral} we get:

$$\int {1\over{x \ln x}} {\rm d}x = \ln n_1(\theta) = \ln \ln x$$

\endexample

\example
\label{ex: integrate ln x}
Compute $\int \ln x \,{\rm d}x$

Again we'll use ${\bf C}(x, \theta = \ln x)$

$$\int \theta \ud x$$

This has the form of equation \eqref{logarithmic integration theorem - integrand}
with $n=1$ and $a_1 = 1$, so

$$\left[ A_1 + 2 c_2 \theta \right]' = a_1 = 1$$

$$A_1 + 2 c_2 \theta = x$$

Since $A_1 \in {\bf C}(x)$, it can not involve $\theta$, so $A_1 = x$ and $c_2 = 0$.

$$\left[ A_0 + c_1 \theta \right]' = a_0 - \frac{k'}{k} A_1 = 0 - \frac{1}{x} x = -1$$

$$A_0 + c_1 \theta = -x$$

So $A_0 = -x$ and $c_1 = 0$.  Plugging $A_0$, $A_1$, $c_1$ and $c_2$ into
equation \eqref{logarithmic integration theorem - integral} we get:

$$\int \theta \ud x = x\theta - x$$
$$\int \ln x \,{\rm d}x = x \ln x - x$$

\endexample

\vfill\eject

\example Compute $\int \tan^{-1} x\, {\rm d}x$

\begin{comment}

The ``standard'' approach to this integral would be to note that the
derivative of $\tan^{-1} x$ is $1/(x^2+1)$ and use integration by parts:

$$\int \tan^{-1} x\, {\rm d}x = x\, \tan^{-1}x - \int \frac{x\, {\rm d}x}{x^2+1}
= x\, \tan^{-1}x - \ln(x^2+1) + C$$

\end{comment}

Using the differential algebra identity $\tan^{-1} x =
{1\over2}\,i\,\ln {{ix-1}\over{ix+1}}$ from
Section \ref{sec:Liouvillian Forms}, we use the differential field
${\mathbb C}(x,\theta)$; $\theta = \ln {{ix-1}\over{ix+1}}$; $\theta'
= - \frac{2i}{x^2+1}$ and compute

% {1\over2}\,i\,\ln {{ix-1}\over{ix+1}}

$$\int {1\over2}\,i\,\theta\, {\rm d}x$$

This is in the form of equation \eqref{logarithmic integration theorem - integrand}
with $n=1$ and $a_1=\frac{1}{2}i$, so
Theorem \ref{logarithmic integration theorem} tells us that
the $\theta$-degree of our integral is at most two.

$$\left[ A_1 + 2 c_2 \theta \right]' = a_1 = {1\over2}\,i$$

$$A_1 + 2 c_2 \theta = {1\over2}\,i x$$

Since $A_1 \in {\mathbb C}(x)$, $c_2$ is zero and $A_1 = \frac{1}{2}ix$.

$$\left[ A_0 + c_1 \theta \right]' = a_0 - A_1 \frac{k'}{k} = - \frac{1}{2}ix \frac{-2i}{x^2+1}$$

\begin{comment}
Integrating both sides we obtain:

$$c_2\theta + a_1 = {1\over2}\,i\,x + c_1$$

Equating coefficients again (and remembering that
$a_1 \in {\mathbb C}(x)$) leads us to conclude that

$$c_2=0 \qquad a_1 = {1\over2}\,i\,x + c_1$$

Finally,

$$a_1\theta' + a_0' = 0$$
$$\left({1\over2}\,i\,x + c_1\right)\theta' + a_0' = 0$$

Since $c_1$ is unknown, we'll leave it on the left-hand side along
with the unknown $a_0$ and move all of our knowns to the right-hand
side, taking advantage of our knowledge of $\theta'$:

$$c_1\theta' + a_0' = {1\over2}\,i\,x \frac{2i}{x^2+1}$$
\end{comment}

$$\left[ A_0 + c_1\theta \right]' = - \frac{x}{x^2+1}$$

%% Integrating both side we obtain:

$$A_0 + c_1\theta = - \frac{1}{2}\ln(x^2+1)$$

Remembering that $A_0$ can include new logarithmic extensions, we
conclude that

$$c_1 = 0 \qquad A_0 = - \frac{1}{2}\ln(x^2+1)$$

and therefore our solution is:

$$\int {1\over2}\,i\,\theta\, {\rm d}x = {1\over2}\,i\,x \theta - \frac{1}{2}\ln(x^2+1)$$
$$\int \tan^{-1} x\, {\rm d}x = x\,\tan^{-1} x - \frac{1}{2}\ln(x^2+1)$$


\endexample

\vfill\eject

\example Determine if $\sum_{n=0}^\infty \frac{1}{n^2} x^n$ is an elementary function.

We can differentiate the series term wise, and if the resulting series
can be identified with an elementary function, then we need only to
decide if that derivative can be integrated into an elementary
function.  Using a standard identity for the Taylor series of $\ln
(1-x)$, we determine that

$$\frac{\ud}{\ud x} \left[ \sum_{n=0}^\infty \frac{1}{n^2} x^n \right] = \sum_{n=1}^\infty \frac{1}{n} x^{n-1} = \frac{1}{x}\ln (1-x)$$

so we need to determine if $\int \frac{1}{x}\ln (1-x) \ud x$ is elementary.  Working in the
differential field ${\mathbb C}(x,\theta = \ln (1-x))$, we're trying to integrate
$\frac{1}{x}\theta \ud x$.  Equation \eqref{eq: logarithmic An} reads:

$$\left[ A_1 + 2 c_2 \theta \right]' = \frac{1}{x}$$

$$A_1 + 2 c_2 \theta = \ln x$$

$\ln x$ is required to express $A_1$, but new logarithms are not
allowed at this point in the algorithm.  Therefore,
$\sum_{n=0}^\infty \frac{1}{n^2} x^n$ is not elementary.

\endexample

\vfill\eject

\example\label{hard log-exp integral}
Compute

%% $$\int{{x(x+1)\{(x^2e^{2x^2}-\ln^2(x+1))^2+2xe^{3x^2}(x-(2x^3+2x^2+x+1)\ln(x+1))\}}\over{((x+1)\ln^2(x+1) - (x^3+x^2)e^{2x^2})^2}} dx$$

$$\int{{x\{(x^2e^{2x^2}-\ln^2(x+1))^2+2xe^{3x^2}(x-(2x^3+2x^2+x+1)\ln(x+1))\}}\over{(x+1)(\ln^2(x+1) - x^2e^{2x^2})^2}} dx$$

\begin{maximacode}
integrand:
   x*((x^2*exp(2*x^2)-log(x+1)^2)^2
      +2*x*exp(3*x^2)*(x-(2*x^3+2*x^2+x+1)*log(x+1)))
   / ((x+1)*(log(x+1)^2 - x^2*exp(2*x^2))^2);
\end{maximacode}

We begin by putting our integral into Liouvillian form, assigning $\psi = \exp(x^2)$ and $\theta = \ln (x+1)$ to obtain:

%% $$\int{{x(x+1)\{(x^2 \psi^2-\theta^2)^2+2x \psi^3(x-(2x^3+2x^2+x+1)\theta)\}}\over{((x+1)\theta^2 - (x^3+x^2)\psi^2)^2}} dx$$

%% Cancelling $x+1$ between the numerator and denominator:

$$\int{{x\{(x^2 \psi^2-\theta^2)^2+2x \psi^3(x-(2x^3+2x^2+x+1)\theta)\}}\over{(x+1)(\theta^2 - x^2\psi^2)^2}} dx$$

\begin{maximacode}
lintegrand:
   subst([log(x+1) = t,
          exp(x^2) = p,
          exp(2*x^2) = p^2,
          exp(3*x^2) = p^3],
      integrand);
\end{maximacode}

Since $\psi$ and $\theta$ are both simple extensions of ${\mathbb C}(x)$, we can operate in either
${\mathbb C}(x,\psi,\theta)$ or ${\mathbb C}(x,\theta,\psi)$.  Since ${\mathbb C}(x,\theta,\psi)$
is an exponential extension of ${\mathbb C}(x,\theta)$, working in ${\mathbb C}(x,\theta,\psi)$
would require evaluating an integral in an exponential extension, which we won't study until
the next chapter.  Therefore, we'll work in ${\mathbb C}(x,\psi,\theta)$, a logarithmic
extension of ${\mathbb C}(x,\psi)$, and hope that we won't have to do anything
too complicated in ${\mathbb C}(x,\psi)$!

To use Maxima for this calculation, let's begin by defining
a function that performs our differentiation.

\begin{maximacode}
depends([t,p], x)$
D(f) := subst(['diff(p,x) = 2*x*p,
               'diff(t,x) = 1/(x+1)],
              diff(f,x))$
\end{maximacode}

We need to put our integrand into partial fractions form, so let's begin by
using Maxima's {\ttfamily divide} function, which performs polynomial long division
with respect to a specific variable, and obtain:

\begin{maximacode}
[a, N]:
   divide(num(lintegrand),
          denom(lintegrand), t);
\end{maximacode}


$$\int {x\over{x+1}} + {{(-4x^5-4x^4-2x^3-2x^2)\psi^3\theta + {2x^3}\psi^3}\over{(x+1)(\theta^2 - x^2\psi^2)^2}} dx$$

The $a_0 = \frac{x}{x+1}$ term is easy.

\begin{maximacode}
A : integrate(a, x);
\end{maximacode}

Let's consider the fractional
term.  Polynomial factorization can be difficult, but this denominator
is easy -- it's just a difference of squares: $(\theta^2 - x^2\psi^2) = (\theta-x\psi)(\theta+x\psi)$.
Maxima's {\tt partfrac} function performs partial fractions
expansion with respect to a given variable, in this case $\theta$.

\begin{maximacode}
pf: partfrac(N/denom(lintegrand), t);
\end{maximacode}

\begin{comment}
\begin{multline*}
\int {x\over{x+1}} + \frac{(2x^4+2x^3+x^2+x)\psi^2 + x \psi}{2(x+1)(\theta + x \psi)^2}
       + \frac{1}{2(x+1)(\theta + x \psi)} \\
       - \frac{(2x^4+2x^3+x^2+x)\psi^2 - x \psi}{2(x+1)(\theta - x \psi)^2}
       - \frac{1}{2(x+1)(\theta - x \psi)} dx$$
\end{multline*}
\end{comment}

\begin{multline*}
\int {x\over{x+1}} + \frac{\frac{2x^4+2x^3+x^2+x}{2(x+1)}\psi^2 + \frac{x}{2(x+1)} \psi}{(\theta + x \psi)^2}
       + \frac{\frac{1}{2(x+1)}}{(\theta + x \psi)} \\
       - \frac{\frac{2x^4+2x^3+x^2+x}{2(x+1)}\psi^2 - \frac{x}{2(x+1)} \psi}{(\theta - x \psi)^2}
       - \frac{\frac{1}{2(x+1)}}{(\theta - x \psi)} dx
\end{multline*}

Next, we want to extract the irreducible factors from
the denominators of our partial fractions expansion.
To facilitate this, let's start by defining some
helper functions:
% to convert the partial fractions
% expansion to a list, and to extract base factors
% from powers of factors:

\begin{maximacode}
is_power(expr) := is(part(expr, 0) = "^")$
base(e) :=
   if is_power(e) then part(e,1) else e$
power(e) :=
   if is_power(e) then part(e,2) else 1$
\end{maximacode}

Now we can extract the portions of the
denominators that involve the variable $\theta$:

\begin{maximacode}
den(pl) := partition(denom(pl),t)[2]$
denoms: map(lambda([u], den(u)),
            partlist(pf));
\end{maximacode}

We extract the irreducible factors:

\begin{maximacode}
n: unique(map(base, denoms));
\end{maximacode}

Now, for each fraction in the partial fractions
expansion, we determine which irreducible
factor is in its denominator, to what power
it appears, and use this information to
assign the rest of the fraction to the
variable $b_{i,j}$:

\begin{maximacode}
map(lambda([u], which(n, base(u))), denoms);
map(power, denoms);
map(lambda([u],
       b[which(n, base(den(u))),
         power(den(u))]: u*den(u)),
    partlist(pf))$
displayarray(b)$
\end{maximacode}


We have two irreducible factors: $n_1(\theta) = \theta - x \psi$ and
$n_2(\theta) = \theta + x \psi$.  Each has $\theta$-degree 1, so
the numerators in our partial fractions expansions have
$\theta$-degree 0, as expected.

Now we're ready to apply
Theorem \ref{logarithmic integration theorem}.  Let's start with
$n_1(\theta)$.  $m_1=2$, so

$$R_{1,1}(\theta) = b_{1,2}(\theta) = \frac{2x^4+2x^3+x^2+x}{2(x+1)}\psi^2 + \frac{x}{2(x+1)} \psi$$

\begin{maximacode}
R[1,1] : b[1,2];
\end{maximacode}

and we wish to compute

$$ B_{1,1}(\theta) \equiv - \frac{R_{1,1}(\theta)}{n_1'(\theta) } \mod n_1(\theta)$$

As modulo calculations go, this one is easy.

\begin{maximacode}
B[1,1] ::: - R[1,1] / D(n[1]);
\end{maximacode}
% divide(%, n[1], t);

Now we wish to compute $Q_{1,1}(\theta)$:

$$ Q_{1,1}(\theta) = - \frac{R_{1,1}(\theta) + B_{1,1}(\theta) n_1'(\theta)}{n_1(\theta)}$$

\begin{maximacode}
Q[1,1] :::
   - (R[1,1] + B[1,1] * D(n[1])) / n[1];
\end{maximacode}

This division to obtain $Q_{1,1}$ will always be exact.  What might not be exact is the following division
to obtain $C_{1}$.  If the division isn't exact, or if $C_1$ isn't a constant, then the integral is not elementary.

$$ C_1 = \frac{b_{1,1}(\theta) - B_{1,1}'(\theta) - Q_{1,1}(\theta)}{n_1'(\theta)} $$

\begin{maximacode}
C[1] :::
   (b[1,1] - D(B[1,1]) - Q[1,1]) / D(n[1]);
\end{maximacode}

A similar calculation handles the other irreducible factor:

\begin{maximacode}
R[2,1] : b[2,2];
B[2,1] ::: - R[2,1] / D(n[2]);
Q[2,1] :::
   - (R[2,1] + B[2,1] * D(n[2])) / n[2];
C[2] :::
   (b[2,1] - D(B[2,1]) - Q[2,1]) / D(n[2]);
\end{maximacode}

Plugging everything together, we conclude that our solution is:

\begin{maximacode}
A + ratsimp(sum(B[i,1]/n[i],i,1,2))
   + logcontract(2*sum(C[i] * log(n[i]),i,1,2))/2;
\end{maximacode}

\begin{comment}
\begin{multline*}
\int{{x\{(x^2 \psi^2-\theta^2)^2+2x \psi^3(x-(2x^3+2x^2+x+1)\theta)\}}\over{(x+1)(\theta^2 - x^2\psi^2)^2}} dx \\
= x - \theta -\frac{\frac{x}{2}\psi}{\theta + x \psi} - \frac{\frac{x}{2}\psi}{\theta - x \psi} + \frac{1}{2}\ln(\theta + x\psi)- \frac{1}{2}\ln(\theta - x \psi)
\end{multline*}

$$ = x - \theta -\frac{x \psi \theta}{\theta^2 - x^2 \psi^2} + \frac{1}{2}\ln\frac{\theta + x\psi}{\theta - x \psi}$$
\end{comment}

Converting back to our original form:

\begin{maximacode}
subst([t=log(x+1), p=exp(x^2)], %);
\end{maximacode}

Finally, we verify that this is, in fact,
an anti-derivative of the original integrand:

\begin{maximacode}
diff(%,x) === integrand;
\end{maximacode}

$$\int{{x\{(x^2e^{2x^2}-\ln^2(x+1))^2+2xe^{3x^2}(x-(2x^3+2x^2+x+1)\ln(x+1))\}}\over{(x+1)(\ln^2(x+1) - x^2e^{2x^2})^2}} dx$$
$$= x - \ln(x+1) - {x e^{x^2}\ln(x+1)\over{\ln^2(x+1)-x^2 e^{2x^2}}}
+ {1\over2}\ln\frac{\ln(x+1) + x e^{x^2}}{\ln(x+1) - x e^{x^2}}$$

\endexample

\vfill\eject

\example
\label{double-log integral}
Compute
$$\int \left[ (\ln(\ln x))^2 \ln x - \frac{2}{\ln x}(\ln(\ln x) +1) \right] dx$$

We'll use the extension ${\bf C}(x,\theta,\psi)$
where $\theta = \ln x$ and $\psi = \ln \theta = \ln \ln x$.
Converting to Liouvillian form, our integral becomes

$$\int \left[ \psi^2 \theta - \frac{2}{\theta}(\psi+1) \right] dx
= \int \left[ \theta \psi^2 - \frac{2}{\theta}\psi - \frac{2}{\theta} \right] dx$$

Since we need $\theta$ to construct $\psi$, our extensions nest
in only a single way:

\begin{center}
\begin{tikzpicture}

\node (p3) [extension, label={[name=p2] ${\bf C}(x,\theta) \qquad \theta = \ln x$}] {${\bf C}(x)$};
\node [fit=(p3)(p2), draw, rounded corners, label={[name=p1] ${\bf C}(x,\theta,\psi) \qquad \psi = \ln \theta$}] {};
\node [fit=(p3)(p2)(p1), draw, rounded corners, inner sep=8pt] {};

\end{tikzpicture}
\end{center}

So, we'll work in ${\bf C}(x,\theta,\psi)$, recursing into
${\bf C}(x,\theta)$ and ${\bf C}(x)$.  In ${\bf C}(x,\theta,\psi)$,
$n=2$, and we identify the coefficients of our integrand:

$$a_2 = \theta \qquad a_1 = - \frac{2}{\theta} \qquad a_0 = - \frac{2}{\theta}$$

So equation \eqref{eq: logarithmic An} reads:

$$\left[ A_2 + 3 c_3 \psi \right]' = a_2 = \theta$$

We already performed this integration in example \ref{ex: integrate ln x}
with the result:

$$A_2 + 3 c_3 \psi = x\theta - x$$

In this example, $A_2 \in {\mathbb C}(x,\theta)$, so $A_2 = x\theta -x$, $c_3=0$.

$$\left[ A_1 + 2 c_2 \psi \right]' = a_1 - 2 \frac{\frac{1}{x}}{\theta}[x\theta - x]$$
$$= -\frac{2}{\theta} - \frac{2}{x \theta}[x\theta - x] = -2$$

$$A_1 + 2 c_2 \psi = -2x$$

So $A_1 = -2x$ and $c_2 = 0$.  Finally,

$$A_0' = a_0 - \frac{1}{x \theta}(-2x) = -\frac{2}{\theta} + \frac{2}{\theta} = 0$$

So $A_0 = C$ and our result becomes:

$$\int \left[ \theta \psi^2 - \frac{2}{\theta}(\psi+1) \right] dx = (x\theta -x)\psi^2 - 2x \psi$$


\endexample

\vfill\eject

\section{Hermite Reduction}

Another, more efficient, approach to handling the polynomials in
denominators is to reduce their order until our denominator has only
factors of multiplicity one.  We're attempting to do this:

$$\int {N\over{V^n}} = {A\over{V^{n-1}}} + \int {B\over{V^{n-1}}}$$

Differentiating:

$${N\over{V^n}} = {A'\over{V^{n-1}}} - (n-1){AV'\over{V^n}} + {B\over{V^{n-1}}}$$

and multiplying through by $V^n$:

$$N = VA' - (n-1)AV' + BV$$
$$N = (A'+B)V - (n-1)AV'$$

This equation has the form of a polynomial Diophantine equation, and
since we know $N$, $V$ and $V'$, we can use the extended Euclidian
algorithm to find $(n-1)A$ and $(A'+B)$, which easily gives us $A$ and
$B$.  So long as $V$ is square-free, we know that $\gcd(V,V')=1$
(EXPLAIN), so we're guaranteed a solution (STATE THEOREM).

\example
Redo Example \ref{hard log-exp integral} using Hermite reduction.

$$\int{{x\{(x^2e^{2x^2}-\ln^2(x+1))^2+2xe^{3x^2}(x-(2x^3+2x^2+x+1)\ln(x+1))\}}\over{(x+1)(\ln^2(x+1) - x^2e^{2x^2})^2}} dx$$

We proceed as before, putting the integral into Liouvillian form
and dividing out $\frac{x}{x+1}$ to obtain a proper fraction:

$$\int {x\over{x+1}} + {{(-4x^5-4x^4-2x^3-2x^2)\psi^3\theta + {2x^3}\psi^3}\over{(x+1)(\theta^2 - x^2\psi^2)^2}} dx$$

Now we apply the Hermite reduction, using:

$$V = \theta^2 - x^2\psi^2 \qquad V' = {2\over{x+1}}\theta - (2x+4x^3)\psi^2$$

$$N = {{{-4x^5-4x^4-2x^3-2x^2}\over{x+1}}\psi^3\theta + {{2x^3}\over{x+1}}\psi^3} $$

\vfill\eject

Our polynomial Diophantine equation is:

$${{{-4x^5-4x^4-2x^3-2x^2}\over{x+1}}\psi^3\theta + {{2x^3}\over{x+1}}\psi^3}
  = sV + tV'$$

\begin{maximacode}
v: t^2-x^2*p^2;
vtick: D(v);
N/denom(lintegrand);
r: gcdex(v,vtick,t) * N/(x+1)$
S::: r[1] - divide(r[1],vtick,t)[1]*vtick;
T::: r[2] + divide(r[1],vtick,t)[1]*v;
\end{maximacode}

So, our solution is:

$${{{-4x^5-4x^4-2x^3-2x^2}\over{x+1}}\psi^3\theta + {{2x^3}\over{x+1}}\psi^3}
  = {-2x\over{x+1}}\psi V + x\psi\theta V'$$

\begin{maximacode}
A: -T;
B: S - D(A);
\end{maximacode}

$$A = -x\psi\theta$$
$$A' = -\psi\theta - x(2x)\psi\theta - x\psi{1\over{x+1}}$$
$$A' + B = {-2x\over{x+1}}\psi$$
$$B = (2x^2+1)\psi\theta - {x\over{x+1}}\psi$$

Which reduces our integral to:

\begin{maximacode}
A[0] + A/v + 'integrate(B/v, x);
\end{maximacode}

%% {\LARGE$$\int {x\over{x+1}} - {x\psi\theta\over{\theta^2-x^2\psi^2}} +  \int{{(2x^2+1)\psi\theta - {x\over{x+1}}\psi}\over{\theta^2-x^2\psi^2}} dx$$}

{\LARGE$$x - \theta - {x\psi\theta\over{\theta^2-x^2\psi^2}} +  \int{{(2x^2+1)\psi\theta - {x\over{x+1}}\psi}\over{\theta^2-x^2\psi^2}}$$}

%% \vfill\eject

%% {\LARGE$$x - \theta - {x\psi\theta\over{\theta^2-x^2\psi^2}} +  \int{{(2x^2+1)\psi\theta - {x\over{x+1}}\psi}\over{\theta^2-x^2\psi^2}}$$}

Now we can compute a Rothstein-Trager resultant:

\begin{maximacode}
resultant(B - z * D(v), v, t);
\end{maximacode}

This result is in ${\mathbb C}(x,\psi)[z]$, so the first two factors are just
{\it content} (EXPLAIN THIS TERM).

\begin{maximacode}
partition(%,z)[2];
\end{maximacode}

The result is really just $4z^2-1$,
which has two solutions: $\pm\frac{1}{2}$.  Substituting in these two
values for $z$, we obtain the corresponding logarithms:

\begin{maximacode}
gcd(B - (1/2)*D(v), v);
gcd(B + (1/2)*D(v), v);
\end{maximacode}


{\LARGE$$x - \theta - {x\psi\theta\over{\theta^2-x^2\psi^2}} + {1\over2}\ln({{\theta + x \psi}\over{x+1}}) - {1\over2}\ln({{\theta - x \psi}\over{x+1}})$$}

% \vskip 2in

$$\int{{x(x+1)\{(x^2e^{2x^2}-\ln^2(x+1))^2+2xe^{3x^2}(x-(2x^3+2x^2+x+1)\ln(x+1))\}}\over{((x+1)\ln^2(x+1) - (x^3+x^2)e^{2x^2})^2}} dx$$

{\LARGE$$= x - \theta - {x\psi\theta\over{\theta^2-x^2\psi^2}} + {1\over2}\ln(\theta + x \psi) - {1\over2}\ln(\theta - x \psi)$$}

$$= x - \ln(x+1) - {x e^{x^2}\ln(x+1)\over{\ln^2(x+1)-x^2 e^{2x^2}}}$$
$$+ {1\over2}\ln\left[\ln(x+1) + x e^{x^2}\right] - {1\over2}\ln\left[\ln(x+1) - x e^{x^2}\right]$$

\endexample
