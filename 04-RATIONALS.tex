
\mychapter{Integration of Rational Functions}

Since our strategy will be to reduce integrals in complex fields by
stripping away their extensions and obtaining integrals in the simpler
underlying fields, it follows that we should start this discussion by
describing how to integrate in ${\bf C}(x)$, the field that underlies
all the others.

Perhaps this seems pedantic.  After all, didn't we go over all this in
first year Calculus?  Don't we already know everything we need to
about integrating rational functions?  We just factor the denominator,
do a partial fractions expansion, plug in some simple known integrals,
and we're done, right?

Not so fast.  To begin with, there's that business of ``just''
factoring the denominator.  As we've already seen, factoring a large
polynomial can be quite a daunting undertaking.  Techniques have been
developed to avoid it as much as possible.  Also, if you're seeing
Liouville's theorem for the first time, then a whole new dimension to
things like $\arctan$ open up when you regard them as complex
logarithms.  And finally, the multi-valued nature of complex
logarithms and related functions make them very slippery little
beasts.  It's easy to get nonsense answers from the simplest
calculations if you're not careful.

\section{Logarithms and related functions}

Let's start with a simple calculation, one we learned back in Calc I:

$$\int{1\over{x^2-1}}dx = \arctan x$$

Now, we've already learned that the way to handle arctangents and the
like is to convert them to E-L-R form, which for $\arctan$ is:

$$\arctan x = {1\over2}\,i\,\ln {{ix-1}\over{ix+1}}$$

Interesting, but not very illuminating.  Nevertheless, as Sherlock
Holmes was wont to say, ``once you have eliminated the impossible\ldots''

% Let's examine the improbable remains in an attempt to find the truth:

\begin{eqnarray*}
\int{1\over{x^2-1}} dx &=& \int{1\over{(x+i)(x-i)}} dx \\
\end{eqnarray*}

%Applying one or another of our techniques for partial fractions
%expansion, we compute:

\begin{eqnarray*}
{1\over{(x+i)(x-i)}} &=& {1\over2}\Big[{i\over{x+i}} - {i\over{x-i}}\Big]\\
              &=& -{1\over2}i\Big[{i\over{-ix+1}} - {i\over{-ix-1}}\Big]\\
\end{eqnarray*}

%On that last step, I multiplied through by 1 in the form
%${-i}\over{-i}$ for reasons that I'll explain later.
%But now, since each numerator is just the negative
%of its denominator's derivative, we proceed:

\begin{eqnarray*}
\int{1\over{x^2-1}}dx &=& -{1\over2}i\int\Big[{i\over{-ix+1}} - {i\over{-ix-1}}\Big]dx\\
	&=& -{1\over2}i\Big[-\ln(-ix+1) + \ln(-ix-1)\Big] \\
	&=& -{1\over2}i\Big[\ln(-ix-1) - \ln(-ix+1)\Big] \\
\end{eqnarray*}

%How do we evaluate something like $\ln(-ix-1)$?  Well, Euler's identity
%is a good place to start:

\begin{eqnarray*}
e^{i\theta} &=& i\sin\theta + \cos\theta \\
{i\theta} &=& \ln[i\sin\theta + \cos\theta] \\
\end{eqnarray*}

%We can always factor a complex number into its modulus and its angle:

\begin{eqnarray*}
a+bi &=& \sqrt{a^2+b^2}\Big({a\over\sqrt{a^2+b^2}} + {b\over\sqrt{a^2+b^2}}i\Big) \\
\end{eqnarray*}

%Now, the expression in parenthesis on the right is just an x-y
%coordinate pair on the unit circle, which form the sine and cosine of
%an angle.  What angle?  Well, the tangent of the angle is going to be
%the ratio between the y (imaginary) and x (real) coordinates, so its
%$b\over a$, and therefore the angle must be $\arctan {b\over a}$.
%Combining this logic with the last two equations and the addition law
%of logarithms lets us obtain a general expression for imaginary
%logarithms:

\begin{eqnarray*}
\ln(a+bi) &=& \ln\sqrt{a^2+b^2} +i\arctan{b\over a} \\
\end{eqnarray*}

%Plugging this back into our integral, and using the fact that
%$\arctan$ is an odd function, we conclude:

\begin{eqnarray*}
\int{1\over{x^2-1}}dx &=& -{1\over2}i\Big[\ln(-ix-1) - \ln(-ix+1)\Big] \\
&=& -{1\over2}i \Big[\big(\ln\sqrt{x^2+1} + i\arctan{x}) - (\ln\sqrt{x^2+1} + i\arctan(-x)\big) \Big] \\
&=& -{1\over2}i \Big[2 i\arctan{x} \Big] \\
&=& \arctan{x}
\end{eqnarray*}

\section{Multi-valued logarithms}

The complex logarithm is a multi-valued function, since adding $2\pi i$
to any logarithm produces another power for the same value.

In {\it Symbolic Integration I}, Manuel Bronstein gave a detailed
analysis, which was so enlightening to me that I will repeat
and expand it here, of the following definite integral:

\vfill\eject

$$\int {{x^4-3x^2+6}\over{x^6-5x^4+5x^2+4}} dx $$

{\small\begin{verbatim}

(%i165) a: x^4-3*x^2+6;

                                            4      2
(%o165)                                    x  - 3 x  + 6
(%i166) b: x^6-5*x^4+5*x^2+4;

                                         6      4      2
(%o166)                                 x  - 5 x  + 5 x  + 4
(%i167) gcd(b,diff(b,x));

(%o167)                                          1
(%i168) m: sylvester[a-z*diff(b,x),b,x];

         [ - 6 z    1    20 z    - 3   - 10 z    6       0       0       0       0     0 ]
         [                                                                               ]
         [   0    - 6 z    1    20 z    - 3    - 10 z    6       0       0       0     0 ]
         [                                                                               ]
         [   0      0    - 6 z    1     20 z    - 3    - 10 z    6       0       0     0 ]
         [                                                                               ]
         [   0      0      0    - 6 z    1      20 z    - 3    - 10 z    6       0     0 ]
         [                                                                               ]
         [   0      0      0      0    - 6 z     1      20 z    - 3    - 10 z    6     0 ]
         [                                                                               ]
(%o168)  [   0      0      0      0      0     - 6 z     1      20 z    - 3    - 10 z  6 ]
         [                                                                               ]
         [   1      0     - 5     0      5       0       4       0       0       0     0 ]
         [                                                                               ]
         [   0      1      0     - 5     0       5       0       4       0       0     0 ]
         [                                                                               ]
         [   0      0      1      0     - 5      0       5       0       4       0     0 ]
         [                                                                               ]
         [   0      0      0      1      0      - 5      0       5       0       4     0 ]
         [                                                                               ]
         [   0      0      0      0      1       0      - 5      0       5       0     4 ]
(%i169) d: ratsimp(determinant(m));

                                     6            4           2
(%o169)                     2930944 z  + 2198208 z  + 549552 z  + 45796
(%i170) e: gcd(d,diff(d,z));

                                           4           2
(%o170)                            732736 z  + 366368 z  + 45796
(%i171) gcd(e, diff(e,z));

                                                 2
(%o171)                                  183184 z  + 45796
(%i172) expand(gcd(e, diff(e,z)) / 45796);

                                                 2
(%o172)                                       4 z  + 1
(%i173) expand((4*z^2+1)^3*45796);

                                     6            4           2
(%o173)                     2930944 z  + 2198208 z  + 549552 z  + 45796


\end{verbatim}}

\vfill\eject

{\small\begin{verbatim}
(%i225) ratexpand((-%i*256/963)*gcdex(a-(%i/2)*diff(b,x),b,x))[3];

                                       3       2
(%o225)                               x  + %I x  - 3 x - 2 %I
(%i226) ratexpand((%i*256/963)*gcdex(a+(%i/2)*diff(b,x),b,x))[3];

                                       3       2
(%o226)                               x  - %I x  - 3 x + 2 %I
\end{verbatim}}

$$\int {{x^4-3x^2+6}\over{x^6-5x^4+5x^2+4}} dx =
   {i\over2}\ln(x^3+ix^2-3x-2i) - {i\over2}\ln(x^3-ix^2-3x+2i)$$
$$=\tan^{-1}({{x^3-3x}\over{x^2-2}})$$

\vfill

\begin{center}
\includegraphics[width=0.7\textwidth]{04-RATIONALS-EXAMPLE1a.pdf}
\end{center}
$$(x^3-3x)+(x^2-2)i$$

%Let us note briefly that the integrand is clearly positive over the
%entire real line.  Now, using a straightforward application of
%the method of partial fractions, we conclude:
%
%$$\int {{x^4-3x^2+6}\over{x^6-5x^4+5x^2+4}} dx =
%   \sum_{\alpha\mid 4\alpha^2+1=0} \alpha \log(x^3+2\alpha x^2-3x-4\alpha)$$
%
%Since the zeros of $4\alpha^2+1$ are $\alpha=\pm i/2$, we evaluate the
%definite integral using the indefinite integral, expand the complex
%logarithms, and obtain:
%
%$$\int_1^2 {{x^4-3x^2+6}\over{x^6-5x^4+5x^2+4}} dx $$
%$$= \Big({i\over2}\log(2+2i)-{i\over2}\log(2-2i)\Big)
% -\Big({i\over2}\log(-2-i)-{i\over2}\log(-2+i)\Big)$$
%$$=-{5\pi\over4}+\arctan({1\over2}) \approx -3.46$$
%
%Since the integral was positive over the entire range of integration,
%this answer can not possibly be correct.
%
%Alternately, we can apply the $\arctan$ identity from the last
%chapter and conclude:
%
%$$\int {{x^4-3x^2+6}\over{x^6-5x^4+5x^2+4}} dx $$
%$$=\sum_{\alpha\mid 4\alpha^2+1=0} \alpha \log(x^3+2\alpha x^2-3x-4\alpha)$$
%$$={i\over2}\ln(x^3+ix^2-3x-2i)-{i\over2}\ln(x^3-ix^2-3x+2i)
%  ={i\over2}\ln{\Big({{x^3-3x+(x^2-2)i}\over{x^3-3x-(x^2-2)i}}\Big)}$$
%$$={i\over2}\ln{\Big({{(x^3-3x)i-(x^2-2)}\over{(x^3-3x)i+(x^2-2)}}\Big)}
%  =\arctan{\Big({{x^3-3x}\over{x^2-2}}\Big)}$$
%
%$$\int_1^2 {{x^4-3x^2+6}\over{x^6-5x^4+5x^2+4}} dx = \arctan 1 - \arctan 2 \approx -0.32$$
%
%What went wrong?
%We gain a
%key insight, as is so often the case, by graphing first the integrand:

\vfill\eject

$${{x^4-3x^2+6}\over{x^6-5x^4+5x^2+4}}$$

\begin{center}
\includegraphics[width=0.65\textwidth]{04-RATIONALS-EXAMPLE1e.pdf}
\end{center}

%And now the (indefinite) integral:

$$\int {{x^4-3x^2+6}\over{x^6-5x^4+5x^2+4}} dx =
\tan^{-1}({{x^3-3x}\over{x^2-2}})$$

\begin{center}
\includegraphics[width=0.65\textwidth]{04-RATIONALS-EXAMPLE1f.pdf}
\end{center}

%A discontinuity has appeared, seemingly out of nowhere.  A closer
%inspection reveals that the break occurs suspiciously close to
%$\sqrt{2}$ --- exactly where a plot of $(x^3-3x)+(x^2-2)i$ in
%the complex plane crosses the negative real axis:

%\begin{center}
%\includegraphics[width=0.7\textwidth]{04-RATIONALS-EXAMPLE1a.pdf}
%\end{center}

%OK, so here's what happened.  When we evaluate a complex logarithm, we
%implicitly select one blade of $\ln$'s Riemann surface.  There's no
%way to avoid this.  $\ln$ is a multi-valued function, the different
%values corresponding to different blades of the Riemann surface, and
%if we want to actually obtain a numerical result from $\ln$ we need to
%pick one of those values.  In the context of an indefinite integral,
%which is only defined within an unspecified additive constant, the
%choice is completely arbitrary.
%
%So far, so good.  Yet now we want a definite integral.  Now we're
%going to evaluate not just a single logarithm, but we're going to
%trace out a curve along the Riemann surface.  As I noted above, if we
%assign a single fixed value to $\ln x$, then there's no way to avoid a
%discontinuity {\it somewhere} in the Riemann surface.  In this
%specific case, we used the identity $\ln(a+bi) = \ln\sqrt{a^2+b^2}
%+i\arctan{b\over a}$, which just converts the discontinuity in $\ln$
%to one expressed in terms of $\arctan$, also a multi-valued function.
%See, there's no way to completely avoid this.
%
%Where's the discontinuity in $\arctan$?  Typically, the function is
%defined so that it ranges from $-{\pi\over2}$ to ${\pi\over2}$ and is
%continuous over finite numbers, so its discontinuity is where its
%argument becomes infinite; i.e, where its denominator goes to zero.
%In the $\ln(a+bi)$ expansion, this occurs where $a$ becomes zero; i.e,
%where $a+bi$ crosses the negative real axis.  This is easier to visualize
%if we reproduce that last graph in 3-D, superimposing $(x^3-3x)+(x^2-2)i$
%on $\arctan$'s Riemann surface:

\begin{center}
\includegraphics[width=0.81\textwidth]{04-RATIONALS-EXAMPLE1c.pdf}
\end{center}

%The z-axis is now the value of $\arctan$.  Along the x-axis, it is
%zero, and it grows positive as we move counter-clockwise, and negative
%as we move clockwise.  The resulting discontinuity along the negative
%real axis, where $\arctan$ jumps from $\pi\over2$ to $-{\pi\over2}$,
%clearly creates a matching discontinuity in the plot of
%$(x^3-3x)+(x^2-2)i$.

%\begin{center}
%\includegraphics[width=\textwidth]{04-RATIONALS-EXAMPLE1b.pdf}
%\end{center}

%Of course, this isn't really $\arctan$'s Riemann surface, only a slice
%of it, and now we begin to find our solution.  $\arctan$'s complete
%Riemann surface looks like an infinite screw, with an infinite series
%of blades, each spaced $2\pi$ apart in the z-direction.  So, when we
%plot our function, what we really want is something more like this,
%moving smoothly along the Riemann surface without any discontinuity.

\begin{center}
\includegraphics[width=0.81\textwidth]{04-RATIONALS-EXAMPLE1d.pdf}
\end{center}

%Alas, while easy enough to graph, and easy enough to understand once
%you're thinking about the continuity of a Riemann surface, this is
%just asking a bit much from poor old $\ln$ (or $\arctan$).  There's no
%way for the function to know which result is needed based solely on a
%single value as the argument.
%
%However, there is a way out, at least in the case of rational
%functions.  The method, due to R. Rioboo, is to take advantage of two
%things.  First, the addition law of logarithms ($\ln ab = \ln a + \ln
%b$), combined with Euler-?? interpretation of complex numbers, lets us
%split a complex logarithm into two logarithms, each of which require
%only half the range of angles of the original logarithm.  Second,
%since a rational function only intersects the real axis in a finite
%number of points (the finite number of zeros of its real component), a
%finite (and easily computed) number of reductions converts the
%logarithm into a sum of logarithms, none of whose arguments cross
%the real axis.

\vfill\eject

$$\int {{x^4-3x^2+6}\over{x^6-5x^4+5x^2+4}} dx =
   {i\over2}\ln(x^3+ix^2-3x-2i) - {i\over2}\ln(x^3-ix^2-3x+2i)$$

{\small\begin{verbatim}
(%i205) gcdex(x^3-3*x,x^2-2);

                                                 2
                                             x  x  - 1
(%o205)/R/                                [- -, ------, 1]
                                             2    2

(%i214) expand(((x^3-3*x)+(x^2-2)*%i)*((x^2-1)-x*%i));

                                         5      3
(%o214)                                 x  - 3 x  + x + 2 %I
\end{verbatim}}

% $$[(x^3-3x)+(x^2-2)i][(x^2-1)-xi] = x^5-3x^3+x+2i$$

$$\ln\Big[(x^3-3x)+(x^2-2)i\Big] = \ln\Big[x^5-3x^3+x+2i\Big] - \ln\Big[(x^2-1)-xi\Big]$$

$$\ln(x^3+ix^2-3x-2i) = \ln(x^5-3x^3+x+2i) - \ln(x^3-i) + \ln(x+i)$$

$$\ln(x^3-ix^2-3x+2i) = \ln(x^5-3x^3+x-2i) - \ln(x^3+i) + \ln(x-i)$$

%$$\int {{x^4-3x^2+6}\over{x^6-5x^4+5x^2+4}} dx =
%   {i\over2}\ln(x^5-3x^3+x+2i) - {i\over2}\ln(x^5-3x^3+x-2i)$$
%$$   + {i\over2}\ln(x^3+i) - {i\over2}\ln(x^3-i)
%   + {i\over2}\ln(x+i) - {i\over2}\ln(x-i)$$

$$\int {{x^4-3x^2+6}\over{x^6-5x^4+5x^2+4}} dx =
\tan^{-1}({{x^5-3x^2+x}\over2}) + \tan^{-1}(x^3) + \tan^{-1}(x)$$

\begin{center}
\includegraphics[width=0.7\textwidth]{04-RATIONALS-EXAMPLE1g.pdf}
\end{center}

\vfill\eject


\section{A Bit Of Perspective}

Now, something interesting happened during the course of this chapter.
Something subtle enough that I didn't notice it for the first year or
so that I worked with this theory.  Maybe you're more cleaver than I
am and have already seen it (or maybe I just organized this book well
enough that it's more obvious to you that it was to me).

How did we start with a problem that was pure algebra, and end up with
a solution that involved topology?

Let's review.  We defined a differential field using mapping between
elements as the derivation.  No topology yet --- the field was purely
discrete, with no concept of ``closeness'' between the elements.  We
went looking for an element that got mapped onto some specified
element.  Again, no topology.  And finally, we extended to a
logarithmic extension, defined purely as a transcendental extension
with a specific differential mapping.  Still no topology.

So why have we spent the last ? pages discussing topology?

The change happened when we went from regarding ``$\ln x$'' as a
transcendental element over the field ${\bf C}(x)$ to regarding it as
$\ln(x)$, a function mapping one complex number to another...


In conclusion, we need to be careful when working with multi-valued
functions like $\ln$ and $\arctan$, but the concept of a Riemann
surface provides an important conceptual tool for dealing with them.
Liouville's theorem tells us that additional logarithms can be
introduced by integration, and the multi-valued nature of the complex
logarithm leaves us with a choice as to which branch of the function
(or blade of the Riemann surface) to use.  Yet fundamental principles
of calculus tell us that an indefinite integral is only defined within
a constant of integration, so it doesn't matter exactly which value of
the logarithm we choose to use.  Having made that choice, however, we
then need to remain consistent by preserving continuity on the Riemann
surface during the evaluation of any definite integral.  In the
specific case of rational function integration, Rioboo's method gives
us an algorithm that automatically preserves this continuity.  For
more complicated integrals that introduce new logarithms, we apply
the more general concept of continuity on the Riemann surface.
