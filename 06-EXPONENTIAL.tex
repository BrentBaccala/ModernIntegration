
\setcounter{chapter}{5}
\mychapter{The Exponential Extension}

The two distinctive features of exponential extensions are the
presence of {\it special} polynomials, which divide their own
derivatives, and the appearance of the Risch differential equation.

Recall our basic theorem on the behavior of exponential extensions:

\begin{customthm}{\ref{basic exponential properties}}
Let $E=K(\theta)$ be a simple transcendental exponential extension of
a differential field $K$ with the same constant subfield as $K$,
let $p=\sum p_i \theta^i$ be a polynomial in $K[\theta]$ ($p_i \in K$),
and let $r$ be a rational function in $K(\theta)$. Then:

\begin{enumerate}
\item ${\rm Deg}_\theta\, p' = {\rm Deg}_\theta\, p$
\item If $p$ is monic and irreducible, then $p' \mid p$ if and only if $p = \theta$.
\item If an irreducible monic factor other than $\theta$ appears in $r$'s
denominator with multiplicity $m$,
then it appears in $r'$'s denominator with multiplicity $m+1$
\item $r' \in K$ if and only if $r \in K$
\end{enumerate}

\end{customthm}

Contrast this theorem with the behavior of the ordinary polynomials
that we're accustomed to.  Ordinary polynomials never behave in the
manner described in (2); polynomials that do are called {\it special}.
Instead, ordinary polynomials always behave in the way described in
(3); such polynomials are called {\it normal}.

Irreducible polynomials are characterized as either normal or special,
depending on whether they divide their own their derivatives.
Theorem \ref{basic exponential properties} (2) states that in a
exponential extension, the only special irreducible polynomial is
$\theta$ itself.

\begin{comment}
In a hypertangent extension, the only special
irreducible polynomial is $(\theta^2+1)$.
\end{comment}

We attack integrands in exponential extensions in much the same way as
we attack ordinary polynomials: we factor the denominator into
irreducible factors and perform a partial fractions expansion.  In
this case, however, we have to classify the denominator factors as
either normal or special.  Normal factors can be handled in much the
same way as we're used to, but special factors are treated in a manner
similar to polynomials.

\begin{comment}

t=tan x
t^2+1 = tan^2 x + 1 = sec^2 x
d(t^2+1) = 2tdt = 2 tan x sec^2 x

dt/dx = sec^2 x = (1 + tan^2 x)
Dt = t^2 + 1
D(t^2+1) = 2t(t^2+1)

\end{comment}

Let $p=\sum p_i \theta^i$ be a polynomial in $K[\theta]$,
and take its derivation:

$$p' = \sum_{i=0}^n (p_i' + i p_i k') \theta^i$$

Notice that, unlike the logarithm or rational cases, there is no
interdependence between the various elements of the sum; each term is
completely independent.  Instead, each coefficient of $\theta^i$ has
the form $p_i' + A p_i$, and equating the $p'$ polynomial to the
integrand's polynomial produces a differential equation of the form:

$$p_i' + A p_i = B \qquad A,B,p_i \in K$$

This is called a {\it Risch equation} and is a primary object of our
study.  Solving Risch equations in a differential extension is the
principle problem that we need to solve in order to carry out our
program of symbolic integration.

Notice that special factors in the denominator behave in almost
exactly the same way as polynomials.  Consider the derivative of
$p=a \theta^{-i}$:

$$p' = (a' - i a k') \theta^{-i}$$

Thus, polynomial terms and partial fractions terms involving special
polynomials can both be treated in the same way.  They both give rise
to Risch equations that need to be solved recursively in the underlying
field.  On the other hand, partial fractions terms involving
normal polynomials give rise to rational functions and logarithms
in the result.

\vfill\eject
\section{The Exponential Structure Theorem}

\theorem\label{exponential structure theorem}
Let $K$ be a differential field, let $K(\theta = \exp k)$ be a simple
exponential extension of $K$, let $n_i(\theta)$ be
normal irreducible polynomials in $K[\theta]$,
and let $f$ be an element of $K(\theta)$
with partial fractions expansion:

$$f = \sum_{i=0}^n a_i \theta^i + \sum_{j=1}^{l} \frac{b_{j}}{\theta^j}
+ \sum_{i=1}^\nu \sum_{j=1}^{m_i} \frac{c_{i,j}(\theta)}{n_i(\theta)^j}$$

$$a_i, b_j \in K \qquad c_{i,j}(\theta),n_i(\theta) \in K[\theta]$$

If $f$ has
a Liovillian anti-derivative $F$, then $F \in K(\theta, \Psi)$,
where $K(\theta, \Psi)$ is a finite logarithm extension
of $K(\theta)$, $F$ has a partial fractions expansion of the form:

$$F = \sum_{i=0}^n A_i \theta^i + \sum_{j=1}^{l} \frac{B_{j}}{\theta^j}
+ \sum_{i=1}^\nu \sum_{j=1}^{m_i+1} \frac{C_{i,j}(\theta)}{n_i(\theta)^j}
+ \sum_{i=1}^\eta D_i \Psi_i$$

$$A_i, B_j \in K \qquad C_{i,j}(\theta),n_i(\theta) \in K[\theta] \qquad D_i' = 0$$

and the following relationships hold:

$$A_i' + i A_i k' = a_i  \qquad  B_{j}' - j k' B_{j} = b_j$$

$$-jC_{i,j}(\theta)n_i'(\theta) = Q_{i,j}(\theta) n_i(\theta) + R_{i,j}(\theta)$$

\begin{align*}
R_{i,m_i}(\theta) & = c_{i,m_i+1} \\
C_{i,j}'(\theta) + Q_{i,j}(\theta) + R_{i,j-1}(\theta) & = c_{i,j} \qquad {}_{1<j\le m_i}\\
C_{i,1}'(\theta) + Q_{i,1}(\theta) + D_i n_i(\theta)' & = c_{i,1}
\end{align*}

$$-jC_{i,j}(\theta)n_i'(\theta) \equiv R_{i,j}(\theta) \mod n_i(\theta)$$


\proof

% Let's see how to integrate a function $f$ in an exponential extension
% $K(\theta)$, $\theta = \exp(k)$.

By Theorem \ref{weak Liouville theorem}, a Liouvillian antiderivative
of $f$ can only exist in a finite logarithm extension $K(\theta, \Psi)$
of $K(\theta)$ and therefore must have the form:

$$F = R + \sum_{i=1}^n D_i \Psi_i$$

where $R \in K(\theta)$, and the $D_i$ are constants.

Constructing a partial fractions expansion of $R$,
separating the normal and special components of its denominator,
and using the fact that $s_1 = \theta$ is the only
special irreducible polynomial
(Theorem \ref{basic exponential properties}):

\begin{comment}
$$F = \sum_{i=0}^n A_i \theta^i + \sum_{i=1}^\mu \sum_{j=1}^{l_i} \frac{B_{ij}(\theta)}{s_i(\theta)^j}
+ \sum_{i=1}^\nu \sum_{j=1}^{m_i} \frac{C_{i,j}(\theta)}{n_i(\theta)^j}
+ \sum_{i=1}^\eta D_i \Psi_i$$

Let's use the fact that $s_1 = \theta$ is the only
special irreducible polynomial:
\end{comment}

$$F = \sum_{i=0}^n A_i \theta^i + \sum_{j=1}^{l} \frac{B_{j}}{\theta^j}
+ \sum_{i=1}^\nu \sum_{j=1}^{m_i} \frac{C_{i,j}(\theta)}{n_i(\theta)^j}
+ \sum_{i=1}^\eta D_i \Psi_i$$

Now let's differentiate, remembering that $\theta' = k'\theta$:

$$F' = \sum_{i=0}^n (A_i' + i A_i k' )\theta^i
  + \sum_{j=1}^{l} \frac{B_{j}' - j k' B_{j}}{\theta^j}
  + \sum_{i=1}^\nu \sum_{j=1}^{m_i} \frac{C_{i,j}'(\theta) n_i(\theta) - j C_{i,j}(\theta) n_i'(\theta) }{n_i(\theta)^{j+1}}
  + \sum_{i=1}^\eta D_i \frac{E_i'(\theta)}{E_i(\theta)}$$

Let's examine the logarithmic elements $e_i$.  If $e_i$ doesn't involve $\theta$, i.e, $e_i \in K$,
then we can collapse $d_i \Psi_i$ into $\bar{c}_0$, with the understanding that when we recurse
into $K$ to solve for $c_0$, additional logarithm terms are allowed.

Now let's consider what happens if $e_i$ is a polynomial in $K[\theta]$.  If it's reducible, then
the basic properties of logarithms let us split it into multiple irreducible elements.
Otherwise, it's irreducible and therefore either normal or special.  If it's special, then it would
divide its derivative $e_i'$... So it must be normal, and therefore $F'$ must have the form:

$$F' = \bar{A}_0 + \sum_{i=1}^n (A_i' + i A_i k' )\theta^i
  + \sum_{j=1}^{l} \frac{B_{j}' - j k' B_{j}}{\theta^j}
  + \sum_{i=1}^\nu \sum_{j=1}^{m_i} \frac{C_{i,j}'(\theta) n_i(\theta) - j C_{i,j}(\theta) n_i'(\theta) }{n_i(\theta)^{j+1}}
  + \sum_{i=1}^\eta D_i \frac{n_i(\theta)'}{n_i(\theta)}$$

$F'$ has the form of a partial fractions decomposition, but it is not
a partial fractions decomposition because the numerators in the $C$
terms violate the partial fractions degree bounds.  To fix this, let's
divide the $-jC_{i,j}(\theta)n_i'(\theta)$ terms by $n_i(\theta)$
(think polynomial long division) and rewrite them as a quotient
and a remainder:

$$-jC_{i,j}(\theta)n_i'(\theta) = Q_{i,j}(\theta) n_i(\theta) + R_{i,j}(\theta)$$

\begin{comment}

\begin{multline*}
F' = \sum_{i=-l}^n (A_i' + i A_i k' )\theta^i \\
  + \sum_{i=1}^\nu \sum_{j=1}^{m_i} \frac{C_{i,j+1}'(\theta) + Q_{i,j+1}(\theta) + R_{i,j}(\theta) }{n_i(\theta)^{j+1}}
  + \sum_{i=1}^\eta \frac{C_{i,1}'(\theta) + Q_{i,1}(\theta) + D_i n_i(\theta)'}{n_i(\theta)}
\end{multline*}

\begin{multline*}
F' = \sum_{i=-l}^n (A_i' + i A_i k' )\theta^i \\
  + \sum_{i=1}^\nu \frac{R_{i,m_i}(\theta) }{n_i(\theta)^{m_i}}
  + \sum_{i=1}^\nu \sum_{j=2}^{m_i+1} \frac{C_{i,j}'(\theta) + Q_{i,j}(\theta) + R_{i,j-1}(\theta) }{n_i(\theta)^{j}}
  + \sum_{i=1}^\eta \frac{C_{i,1}'(\theta) + Q_{i,1}(\theta) + D_i n_i(\theta)'}{n_i(\theta)}
\end{multline*}

\end{comment}

\begin{multline*}
F' = \bar{A}_0 + \sum_{i=1}^n (A_i' + i A_i k' )\theta^i
  + \sum_{j=1}^{l} \frac{B_{j}' - j k' B_{j}}{\theta^j} \\
  + \sum_{i=1}^\nu \left[ \frac{R_{i,m_i}(\theta) }{n_i(\theta)^{m_i+1}}
  + \sum_{j=2}^{m_i} \frac{C_{i,j}'(\theta) + Q_{i,j}(\theta) + R_{i,j-1}(\theta) }{n_i(\theta)^{j}}
  + \frac{C_{i,1}'(\theta) + Q_{i,1}(\theta) + D_i n_i(\theta)'}{n_i(\theta)} \right]
\end{multline*}

Now $F'$ is an actual partial fractions decomposition.  It not only has
the right form, but all of the other conditions, specifically
the degree bounds, are met.
Therefore, we can perform
a partial fractions decomposition of $f$:

$$f = \sum_{i=0}^n a_i \theta^i + \sum_{j=1}^{l} \frac{b_{j}}{\theta^j}
+ \sum_{i=1}^\nu \sum_{j=1}^{m_i} \frac{c_{ij}(\theta)}{n_i(\theta)^j}$$

Setting $F' = f$ and equating like terms, we see that the polynomial terms
and special denominators give rise to Risch equations:

$$A_i' + i A_i k' = a_i  \qquad  B_{j}' - j k' B_{j} = b_j$$

Normal denominators give rise to terms:

\begin{align*}
R_{i,m_i}(\theta) & = c_{i,m_i+1} \\
C_{i,j}'(\theta) + Q_{i,j}(\theta) + R_{i,j-1}(\theta) & = c_{i,j} \qquad {}_{1<j\le m_i}\\
C_{i,1}'(\theta) + Q_{i,1}(\theta) + D_i n_i(\theta)' & = c_{i,1}
\end{align*}

The highest order term gives us an $R_{i,j}$.  If we have a method of
computing $C_{i,j}$ and $Q_{i,j}$ from $R_{i,j}$, then we can just
move down the list, solving this system of equations from highest
order terms to lowest.  Once we get to the end, we need to see if the
bottom equation can be solved using a constant $D_i$.  If not,
then the equation has no solution.

So how can we calculate $C_{i,j}$?  Remember the definition of
$R_{i,j}$ and $Q_{i,j}$:

$$-jC_{i,j}(\theta)n_i'(\theta) = Q_{i,j}(\theta) n_i(\theta) + R_{i,j}(\theta)$$

Reducing this equation modulo $n_i(\theta)$, we obtain:

$$-jC_{i,j}(\theta)n_i'(\theta) \equiv R_{i,j}(\theta) \mod n_i(\theta)$$

Now we use the fact that $n_i(\theta)$ is {\it irreducible},
and invoke Theorem ??, which states the quotient ring
modulo a prime ideal is a field, so we can perform division:

$$C_{i,j}(\theta) \equiv - \frac{R_{i,j}(\theta)}{jn_i'(\theta)} \mod n_i(\theta)$$

This equation seems to identify $C_{i,j}(\theta)$ up to a multiple of $n_i(\theta)$,
but if we remember our degree bound on partial fractions expansions,
$\deg_\theta C_{i,j}(\theta) < \deg_\theta n_i(\theta)$, we see
that in fact we've completely determined $C_{i,j}(\theta)$ from
$R_{i,j}(\theta)$.

\endtheorem

This analysis wouldn't be terribly useful if we lacked an algorithm to
perform the calculation.  Then we'd have an existence proof without a
constructive procedure, a unfortunate situation that does arise at
times in mathematics.  In this case, fortunately, we do have an
algorithm: the extended Euclidean algorithm from
Section \ref{sec:Polynomial Diophantine Equations}, which is our major
computational tool for calculating inverses in quotient rings.



\begin{comment}
$$F' = \sum_{i=-l}^n (c_i' + i c_i k' )\theta^i
  + \sum_{i=1}^\nu \sum_{j=1}^{m_i} \frac{b_{ij}'(\theta) n_i(\theta) - j b_{ij}(\theta) n_i'(\theta) }{n_i(\theta)^{j+1}}
  + \sum_{i=1}^\eta d_i \frac{n_i(\theta)'}{n_i(\theta)}$$
\end{comment}

\vfil\eject

\example Compute $\int \sin x \, {\rm d}x$

We'll operate in ${\mathbb C}(x, \phi = \exp \,ix)$ and evaluate

$$\int \frac{\phi - \phi^{-1}}{2i} \,dx$$

The first step is write the integrand in partial fractions form:

$$\int \left[ \frac{1}{2i} \phi - \frac{1}{2i} \frac{1}{\phi} \right] \,dx $$

By Theorem \ref{exponential structure theorem}, the integral must have the
form $a_1 \phi + a_{-1} \frac{1}{\phi}$ with $a_1, a_{-1} \in {\mathbb
C}(x)$ and must satisfy the equations:

$$\frac{1}{2i} = a_1' + i a_1 \qquad - \frac{1}{2i} = a_{-1}' - i a_{-1}$$ 

These are very simple Risch equations that can be solved by inspection
to obtain $a_1 = a_{-1} = -\frac{1}{2}$, so

$$\int \frac{\phi - \phi^{-1}}{2i} \,dx = -\frac{1}{2}(\phi + \phi^{-1})
 = -\frac{1}{2}(e^{ix} + e^{-ix}) = -\cos x$$

\endexample

\vfil\eject

\example Compute $\int \tan x \, {\rm d}x$

We'll operate in ${\mathbb C}(x, \phi = \exp \,ix)$ and evaluate

$$-i \int \frac{\phi - \phi^{-1}}{\phi + \phi^{-1}} \,dx$$

The first step is write the integrand in partial fractions form:

$$\frac{\phi - \phi^{-1}}{\phi + \phi^{-1}} = \frac{\phi^2 - 1}{\phi^2 + 1}
= 1 - \frac{2}{\phi^2 + 1}
= 1 - \frac{2}{(\phi + i)(\phi - i)} $$

$$ = 1 + \frac{i}{\phi - i} - \frac{i}{\phi + i} $$

We're now trying to solve these equations:

$$ D_1 n_1(\theta)' = c_{1,1} \qquad D_2 n_2(\theta)' = c_{2,1}$$

$n_1(\theta) = \phi -i$, so $n_1'(\theta) = i \phi$

\endexample

\vfill\eject
\section{Risch Equations in ${\mathbb C}(x)$}

Risch equations in ${\mathbb C}(x)$ arise when our integrand exists in
a {\it simple} exponential extension of ${\mathbb C}(x)$,
i.e, an integrand formed as a rational function of $x$ and a
single exponential of a rational function of $x$.  The integration step
described above produces Risch equations in the underlying field;
in this case, ${\mathbb C}(x)$.

Consider such a Risch equation:

$$r' + S r = T \qquad S,T,r \in {\mathbb C}(x)$$

\begin{comment}
Recall that in ${\mathbb C}(x)$, irreducible factors in the
denominator always increase in degree on differentiation, so $A$'s
factors are the only factors that can appears in $q$'s denominator,
because they must cancel against $q'$.  Thus, we can easily identify
which irreducible factors can appear in $q$'s denominator, and we next
wish to calculate the multiplicities with which they appear.
\end{comment}

The only poles that can appear in $r$'s denominator must appear in
either $S$ or $T$'s denominator, so let's consider a single pole at
$\gamma$, expand $S$, $T$, and $r$ using partial fractions, and look
at the highest powers of $(x-\gamma)$ in the denominator:

$$r = \frac{a}{(x-\gamma)^j} + \cdots  \qquad  r' = \frac{-ja}{(x-\gamma)^{j+1}} + \cdots$$

$$S = \frac{b}{(x-\gamma)^k} + \cdots$$

$$T = \frac{c}{(x-\gamma)^l} + \cdots$$

Combining everything into the Risch equation, we find:

$$\frac{-ja}{(x-\gamma)^{j+1}} + \cdots + \frac{ba}{(x-\gamma)^{j+k}} + \cdots = \frac{c}{(x-\gamma)^l} + \cdots$$

We can classify the equation into three basic cases, based on the value of $k$:

\begin{enumerate}

\item $k=0$.  In this case, the $\frac{-ja}{(x-\gamma)^{j+1}}$ term dominates the left hand side,
and $j = l-1$ in order to match the right hand side.

\item $k=1$.  Here, the high order terms on the left are equal, so either $j=l-1$ in order to match
the right hand side, or $j=b$ and $j>l-1$ in order for the left hand terms to exactly cancel.

\item $k>1$.  Now the $\frac{ba}{(x-\gamma)^{j+k}}$ term dominates the left hand side, so $j=l-k$ in
order to match the right hand side.

\end{enumerate}

By checking all of $S$'s and $T$'s poles using this technique, we can
identify all the poles in $r$'s denominator and determine the
multiplicity with which they appear.

Having done so, we can multiply through by this common denominator,
clear the denominators, and produce a polynomial Risch equation:

$$A p' + B p = C \qquad A,B,C,p \in {\mathbb C}[x]$$

There are three cases now.

First, the highest terms on the left can have higher degree than any term on
the right, and so must cancel against each other.  For this to occur,
$\deg A = \deg B + 1$ (since $\deg p$ drops by one on differentiation),
and we can determine $\deg p$ by looking at the leading
coefficients in $A$ and $B$:

$$A = a_j x^j + \cdots \qquad B = b_{j-1} x^{j-1} + \cdots \qquad p = p_k x^k \cdots$$

$$A p' + B p = (k a_j p_k + b_{j-1} p_k) x^{j+k-1} \cdots$$

In order for this coefficient to be zero, $k=-b_{j-1}/a_j$.
So, if these conditions are met:

$$\deg A = \deg B + 1 \qquad k=-\frac{b_{j-1}}{a_j} $$

then $p$ may exist as a $k^{\rm th}$ degree polynomial.

Otherwise, the leading terms of $A p' + B p$ do not cancel out,
so they must match the leading term of $C$.  This can only
occur if

$$\deg p = \deg C - \max(\deg A - 1, \deg B)$$

Having determined the degree of $p$, we can now determine its
coefficients.

The final case we need to consider is when $p$ is a constant, which
would solve the Risch equation if and only if $C$ was a constant
multiple of $B$, irregardless of $A$.

\vfill\eject

\example Prove that $\int e^{-x^2}\, {\rm d}x$ has no Liouvillian form

We'll use ${\mathbb C}(x, \phi = \exp\, -x^2)$, so $\phi' = -2x$ and
study

$$\int \phi \, {\rm d}x$$

We know from Theorem \ref{basic exponential properties} that our
solution, if it exists, must have the form $a\phi$, where $a \in
{\mathbb C}(x)$, and $a$ must satisfy the Risch equation:

$$a' - 2x a = 1$$

This is already a polynomial Risch equation, and $a'$ has only
a constant coefficient, so $a$ can not have a non-trivial denominator.
Futhermore, identifying $A$ as $1$, $B$ as $-2x$, and $C$ as $1$,
we see that $\deg A = 0$ and $\deg B = 1$.  Since $\deg A \ne \deg B + 1$,
the leading terms on the left hand side can not cancel,
so they must match the leading term on the right.
We compute:

$$\deg p = \deg C - \max(\deg A - 1, \deg B) = 0 - 1 = -1$$

so that doesn't work.  Futhermore, $C$ is not a constant
multiple of $B$, so a constant $p$ can't solve our equation.

We conclude that no solution to this Risch equation exists in ${\mathbb C}(x)$,
so the integral can not be expressed in Liouvillian form.

\endexample


\example Prove that $\int \frac{\sin x}{x} \, {\rm d}x$ has no Liouvillian form

As we often do with trigonometric integrals, we'll operate in
${\mathbb C}(x, \phi = \exp \,ix)$, use Euler's relationship
$e^{ix}=i\sin x + \cos x$, and evaluate

$$\int \frac{\phi - \phi^{-1}}{2ix} \,dx$$

Let's begin by writing the integrand in the form of a rational
function in ${\mathbb C}(x)(\phi)$, i.e, a ratio
of $\phi$-polynomials, with coefficients in ${\mathbb C}(x)$:

$$\frac{1}{2i} \int \left[ \frac{1}{x}\phi - \frac{1}{x}\frac{1}{\phi} \right]\,dx$$

\begin{comment}
% This text doesn't belong here
We want to split the denominator into its normal and special
components, by factoring it into irreducible polynomials and
classifying each one as normal or special.  In this case, the
factoriziation is trivial, and we know from theorem \ref{basic
exponential properties} that $\phi$ is special.

Can we have any logarithms in our integral?  Let's see.
Any logarithm of a rational function can be factored and
split into separate logarithms using basic properties
of a logarithms:

$$\ln ab = \ln a + \ln b \qquad\qquad \ln\frac{a}{b} = \ln a - \ln b$$

So, we need only consider logarithms of irreducible polynomials.

Theorem \ref{basic exponential properties} also tells us that we can
have no normal polynomials in denominator of our integral,
\end{comment}


The integral must have the form $a_1 \phi + a_{-1} \frac{1}{\phi}$
with $a_1, a_{-1} \in {\mathbb C}(x)$ and must satisfy the equations:

$$\left[ a_1 \phi + a_{-1}\frac{1}{\phi} \right]' = (a_1' + i a_1 ) \phi + (a_{-1}' - i a_{-1} ) \frac{1}{\phi}
= \left[ \frac{1}{x}\phi - \frac{1}{x}\frac{1}{\phi} \right]$$

$$\frac{1}{x} = a_1' + i a_1 \qquad - \frac{1}{x} = a_{-1}' - i a_{-1}$$

$$a_1' + a_1 = \frac{1}{x}$$

We've got a single pole in the denominator of $T$, so $k=0$, $l=1$,
and $j=l-1=0$, so there are no poles in the denominator of our solution.
However, there is then no way to produce the denominator on the
right, so the Risch equation has no solution in ${\mathbb C}(x)$.


Thus, the integral
can not be expressed in Liouvillian form.

\endexample

\vfil\eject

Partial fractions terms involving normal polynomials are handled the
same way as, well, normal polynomials.  Terms with simple denominators
give rise to logarithms in the solution, while terms with higher
powered denominators give rise to rational functions in the solution.

One unusual feature of exponential extensions is that the numerator of
a derivative will have the same degree as the denominator, so a long
division step is needed to make the fraction proper, and this will
produce a constant that will modify the integrand.  For this reason,
it's best to handle the denominator's normal factors first,


\example Compute $\int \csc x \, {\rm d}x$

We'll operate in ${\mathbb C}(x, \phi = \exp \,ix)$ and evaluate

$$\int {{2i}\over{\phi - \phi^{-1}}} \,dx = \int 2i {{\phi}\over{\phi^2 - 1}} \,dx$$

Now we want a partial fractions expansion.  We could use a resultant,
or the Euclidean G.C.D. algorithm, but it's simpler to just note that
the denominator's a difference of squares and compute:

$${{c_1}\over{\phi-1}} + {{c_2}\over{\phi+1}} = {{\phi}\over{\phi^2-1}} $$

$${{c_1}(\phi+1)} + {{c_2}(\phi-1)} = \phi \qquad c_1 = c_2 = {1\over2} $$

giving us

$$\int i \big[ {{1}\over{\phi-1}} + {{1}\over{\phi+1}} \big] dx$$

Now, we have irreducible, square-free, non-monomial denominators, so
the answer (if it exists) must be expressed in terms of their
logarithms (Theorem \ref{basic exponential properties}):

$$ d \ln(\phi+1) = {{d(\phi+1)}\over{\phi+1}} = {{i\phi\,dx}\over{\phi+1}} = [1 - {{1}\over{\phi+1}}]i\,dx$$

$$ d \ln(\phi-1) = {{d(\phi-1)}\over{\phi-1}} = {{i\phi\,dx}\over{\phi-1}} = [1 + {{1}\over{\phi-1}}]i\,dx$$

so the integral can be written:

$$\int {{d(\phi-1)}\over{\phi-1}} - {{d(\phi+1)}\over{\phi+1}} = \ln (\phi-1) - \ln(\phi+1) = \ln({{\phi-1}\over{\phi+1}})$$
$$ = \ln ({{e^{ix}-1}\over{e^{ix}+1}}) = \ln ({{e^{ix/2}-e^{-ix/2}}\over{e^{ix/2}+e^{-ix/2}}}) = \ln i {{\sin {x\over2}}\over{\cos {x\over2}}} = \ln \tan {x\over2} $$

where I dropped the $i$ at the end because, as a constant multiple
inside a logarithm, it disappears into the constant of integration,
and we conclude that

$$\int \csc x \,{\rm d}x = \ln \tan {x\over2} $$

\endexample

\vfil\eject

\example Compute $\int {{4^x-1}\over{2^x+1}} {\rm d}x$
\label{integrate 4^x-1/2^x+1}

We'll use the field ${\mathbb C}(x,\Psi = \exp(x \ln 2))$; $\Psi' =
(\ln 2)\Psi$ and the representation (see Example
\ref{represent 4^x+1/2^x+1}):

$$ \frac{\Psi^2-1}{\Psi+1} = \Psi-1$$

So we need to find a solution of the form $a\Psi + \bar{b}$ ($\bar{b}$
can include additional logarithmic elements) that satisfy the Risch
equations:

$$a'\Psi + a\Psi' = a'\Psi + a(\ln 2)\Psi = \Psi \qquad a' + a(\ln 2) = 1$$
$$\bar{b}' = -1$$

Both equations have fairly obvious solutions:

$$a = \frac{1}{\ln 2} \qquad \bar{b}=-x$$

So our solution is

$$\int {{4^x+1}\over{2^x+1}} {\rm d}x = \frac{1}{\ln 2}\Psi - x =
\frac{1}{\ln 2}2^x - x$$

\endexample


\vfil\eject

\example Compute $\int {{4^x+1}\over{2^x+1}} {\rm d}x$
\label{integrate 4^x+1/2^x+1}

We'll use the field ${\mathbb C}(x,\Psi = \exp(x \ln 2))$; $\Psi' =
(\ln 2)\Psi$ and the representation (see Example
\ref{represent 4^x+1/2^x+1}):

$$ \frac{\Psi^2+1}{\Psi+1} = \Psi-1+\frac{2}{\Psi+1}$$

Let's do the fractional part first since it will modify the polynomial
part.

$$\left[\ln(\Psi+1)\right]' = \frac{(\Psi + 1)'}{\Psi+1} = \frac{(\ln
2)\Psi}{\Psi + 1} = \ln 2 - \frac{\ln 2}{\Psi+1}$$

$$-\frac{2}{\ln 2}\left[\ln(\Psi+1)\right]' = \frac{2}{\Psi+1} - 2$$

We can rewrite our integrand as:

$$ \Psi + 1 - \frac{2}{\ln 2}\left[\ln(\Psi+1)\right]'$$

So we need to find a solution of the form $a\Psi + \bar{b}$ ($\bar{b}$
can include additional logarithmic elements) that satisfy the Risch
equations:

$$a'\Psi + a\Psi' = a'\Psi + a(\ln 2)\Psi = \Psi \qquad a' + a(\ln 2) = 1$$
$$\bar{b}' = 1$$

Both equations have fairly obvious solutions:

$$a = \frac{1}{\ln 2} \qquad \bar{b}=x$$

So our solution is

$$\int {{4^x+1}\over{2^x+1}} {\rm d}x = \frac{1}{\ln 2}\Psi + x  - \frac{2}{\ln 2}\left[\ln(\Psi+1)\right] =
\frac{2^x}{\ln 2} + x - \frac{2}{\ln 2}\ln(2^x+1) $$

\endexample


\begin{comment}
\vfill\eject

Consider an irreducible factor $F$ that appears both as a factor of $A$
and also in $q$'s denominator, with
multiplicity $m \ge 1 $, so $q = N/(F^m D)$, and we rewrite the Risch equation:

$$A F q' - B q = C$$

$$A F \frac{N' F D - m N F' D - N F D'}{F^{m+1} D^2} - B \frac{N}{F^m D} = C$$

$$A F N' D - m A F' N D - A F N D' - B N D = C D^2 F^{m}$$

$$ - m A F' N D - B N D  = C D^2 F^{m} - A F N' D + A F N D'$$

$$ - (m A F' - B ) N D  = F (C D^2 F^{m-1} - A N' D - A N D')$$

Now, the right hand side of this equation is a multiple of $F$, so the
left hand side must also be a multiple of $F$.  However, $F$ is
irreducible and is relatively prime to both $N$ and $D$, so the only way
the left hand side can be a multiple of $F$ is if $m A F' - B$ is a
multiple of $F$.

Why?  All of these variables
are polynomials in ${\mathbb C}[x]$, so this equation is an equality
between polynomials.  Because ${\mathbb C}[x]$ is a {\it unique factorization domain}, its
polynomials can factor in essentially one way only, so if $F$ factors
the right hand side, it must also factor the left.  If $F$ were not
irreducible, then it might have two factors, one contributed by $N$
and the other contributed by $D$.  It's the irreducibility of $F$,
the unique factorization of polynomials in ${\mathbb C}[x]$, and
our assumption that $N$ and $D$ are relatively prime to $F$
that makes this argument work.

The simplest way to enforce this factorization requirement is to use
the resultant (Theorem \ref{resultant theorem}):

$${\rm res}_x(m A F' - B, F) = 0$$

We calculate this resultant for each irreducible factor $F$ of $A$,
and this gives us $m$, the power to which $F$ appears in $q$'s
denominator.  If this resultant equation has no positive integer
solution, then $F$ can not appear at all in $q$'s denominator.

\end{comment}
