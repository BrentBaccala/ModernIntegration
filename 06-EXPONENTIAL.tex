
\setcounter{chapter}{5}
\mychapter{The Exponential Extension}

The two distinctive features of exponential extensions are the presence
of {\it special} polynomials, which are divisible by their own
derivatives, and the appearance of the Risch differential equation.

Recall our basic theorem on the behavior of exponential extensions:

\begin{customthm}{\ref{basic exponential properties}}
Let $E=K(\theta)$ be a simple transcendental exponential extension of
a differential field $K$ with the same constant subfield as $K$,
let $p=\sum p_i \theta^i$ be a polynomial in $K[\theta]$,
and let $r=a/b$ be a rational function in $K(\theta)$
($a, b \in K[\theta]$).  Then:

\begin{enumerate}
\item ${\rm Deg}_\theta\, x' = {\rm Deg}_\theta\, x$
\item $p' \mid p$ if and only if $p$ is a power of $\theta$.
\item If an irreducible factor other than $\theta$ appears in $r$'s
denominator with multiplicity $m$,
then it appears in $r'$'s denominator with multiplicity $m+1$
\item $r' \in K$ if and only if $r \in K$
\end{enumerate}

\end{customthm}

Contrast this theorem with the behavior of the ordinary polynomials
that we're accustomed to.  Ordinary polynomials never behave in the
manner described in (2); polynomials that do are called {\it special}.
Instead, ordinary polynomials always behave in the way described in
(3); such polynomials are called {\it normal}.

Irreducible polynomials are characterized as either normal
or special, depending on whether they can be divided by
their derivatives.

In a exponential extension, the only special irreducible polynomial is
$\theta$ itself.  In a hypertangent extension, the only special
irreducible polynomial is $(\theta^2+1)$.

We attack integrands in exponential extensions in much the same way as
we attack ordinary polynomials: we factor the denominator into
irreducible factors and perform a partial fractions expansion.  In
this case, however, we have to classify the factors as either normal
or special.  Normal factors can be handled in much the same way as
we're used to, but special factors are treated in a manner similar to
polynomials.

\begin{comment}

t=tan x
t^2+1 = tan^2 x + 1 = sec^2 x
d(t^2+1) = 2tdt = 2 tan x sec^2 x

dt/dx = sec^2 x = (1 + tan^2 x)
Dt = t^2 + 1
D(t^2+1) = 2t(t^2+1)

\end{comment}

Let $p=\sum p_i \theta^i$ be a polynomial in $K[\theta]$,
and take its derivation:

$$p' = \sum_{i=0}^n (p_i' + i p_i k') \theta^i$$

Notice that, unlike the logarithm or rational cases, there is no
interdependence between the various elements of the sum; each term is
completely independent.  Instead, each coefficient of $\theta^i$ has
the form $p_i' + A p_i$, and equating the $p'$ polynomial to the
integrand's polynomial produces a differential equation of the form:

$$x' + A x = B \qquad A,B \in K$$

This is called a {\tt Risch equation} and is a primary object of our
study.  Solving Risch equations in a differential extension is the
principle problem that we need to solve in order to carry out our
program of symbolic integration.

Notice that special factors in the denominator behave in almost
exactly the same way as polynomials.  Consider the derivative of
$p=a \theta^{-i}$:

$$p' = (a' - i a k') \theta^{-i}$$

Thus, polynomial terms and partial fractions terms involving special
polynomials can both be treated in the same way.  They both give rise
to Risch equations that need to be solved recursively in the underlying
field.

\vfil\eject

\example Compute $\int \sin x \, {\rm d}x$

We'll operate in ${\mathbb C}(x, \phi = \exp \,ix)$ and evaluate

$$\int \frac{\phi - \phi^{-1}}{2i} \,dx$$

The first step is to normalize the integrand, first by transforming it
into a rational function:

$$\int \frac{\phi^2 - 1}{2i \phi} \,dx $$

and next by using polynomial division to split it into a polynomial
and a proper fraction with a monic denominator:

$$\int \left[ \frac{1}{2i} \phi - \frac{1}{2i} \frac{1}{\phi} \right] \,dx $$

The integral must have the form $a_1 \phi + a_{-1} \frac{1}{\phi}$
with $a_1, a_{-1} \in {\mathbb C}(x)$ and must satisfy the equations:

$$\frac{1}{2i} = a_1' + i a_1 \qquad - \frac{1}{2i} = a_{-1}' - i a_{-1}$$ 

These are very simple Risch equations that can be solved by inspection
to obtain $a_1 = a_{-1} = -\frac{1}{2}$, so

$$\int \frac{\phi - \phi^{-1}}{2i} \,dx = -\frac{1}{2}(\phi + \phi^{-1})
 = -\frac{1}{2}(e^{ix} + e^{-ix}) = -\cos x$$

\endexample

Slightly more complicated are Risch equations in ${\mathbb C}(x)$.
These arise when our integrand exists in a simple exponential
extension of ${\mathbb C}(x)$, i.e, an integrand formed as a rational
function of $x$ and a single exponential.  Consider such a Risch
equation:

$$x' - A x = B \qquad A,B,x \in {\mathbb C}(x)$$

We can clear our denominators and obtain a polynomial Risch equation:

$$A x' - B x = C \qquad A,B,C \in {\mathbb C}[x] \\qquad x \in {\mathbb C}(x)$$


\example Prove that $\int e^{-x^2}\, {\rm d}x$ has no elementary form

We'll use ${\mathbb C}(x, \phi = \exp\, -x^2)$, so $\phi' = -2x$ and
study

$$\int \phi \, {\rm d}x$$

We know from Theorem \ref{basic exponential properties} that our
solution, if it exists, must have the form $a\phi$, where $a \in
{\mathbb C}(x)$, and $a$ must satisfy the Risch equation:

$$a' - 2x a = 1$$

As Risch equations go, this one is easy!  Let's first note that
$a$ can not have a non-trivial denominator, since

$$a' = 1 + 2x a$$

We factor $a$'s denominator into irreducible factors, recall that in
${\mathbb C}(x)$, irreducible factors in the denominator always
increase in degree on differentiation, and see that the right hand
side of this equation has no denominator, therefore neither can the
left hand side ($a'$) or $a$ itself.

So $a$ must be a polynomial.  Returning to the original Risch
equation, we see that $a'$ will have degree ${\rm Deg\,} a - 1$ while
$2xa$ will have degree ${\rm Deg\,} a + 1$, so the left hand side must
have degree ${\rm Deg\,} a + 1$, while the right hand side has degree 0.

We conclude that no solution to this equation exists in ${\mathbb C}(x)$,
so the integral can not be expressed in elementary form.

\endexample

\vfil\eject

Partial fractions terms involving normal polynomials are handled the
same way as, well, normal polynomials.  Terms with simple denominators
give rise to logarithms in the solution, while terms with higher
powered denominators give rise to rational functions in the solution.

One unusual feature of exponential extensions is that the numerator of
a derivative will have the same degree as the denominator, so a long
division step is needed to make the fraction proper, and this will
produce a constant that will modify the integrand.  For this reason,
it's best to handle the denominator's normal factors first,


\example Compute $\int \csc x \, {\rm d}x$

We'll operate in ${\mathbb C}(x, \phi = \exp \,ix)$ and evaluate

$$\int {{2i}\over{\phi - \phi^{-1}}} \,dx = \int 2i {{\phi}\over{\phi^2 - 1}} \,dx$$

Now we want a partial fractions expansion.  We could use a resultant,
or the Euclidean G.C.D. algorithm, but it's simpler to just note that
the denominator's a difference of squares and compute:

$${{c_1}\over{\phi-1}} + {{c_2}\over{\phi+1}} = {{\phi}\over{\phi^2-1}} $$

$${{c_1}(\phi+1)} + {{c_2}(\phi-1)} = \phi \qquad c_1 = c_2 = {1\over2} $$

giving us

$$\int i \big[ {{1}\over{\phi-1}} + {{1}\over{\phi+1}} \big] dx$$

Now, we have irreducible, square-free, non-monomial denominators, so
the answer (if it exists) must be expressed in terms of their
logarithms (Theorem \ref{basic exponential properties}):

$$ d \ln(\phi+1) = {{d(\phi+1)}\over{\phi+1}} = {{i\phi\,dx}\over{\phi+1}} = [1 - {{1}\over{\phi+1}}]i\,dx$$

$$ d \ln(\phi-1) = {{d(\phi-1)}\over{\phi-1}} = {{i\phi\,dx}\over{\phi-1}} = [1 + {{1}\over{\phi-1}}]i\,dx$$

so the integral can be written:

$$\int {{d(\phi-1)}\over{\phi-1}} - {{d(\phi+1)}\over{\phi+1}} = \ln (\phi-1) - \ln(\phi+1) = \ln({{\phi-1}\over{\phi+1}})$$
$$ = \ln ({{e^{ix}-1}\over{e^{ix}+1}}) = \ln ({{e^{ix/2}-e^{-ix/2}}\over{e^{ix/2}+e^{-ix/2}}}) = \ln i {{\sin {x\over2}}\over{\cos {x\over2}}} = \ln \tan {x\over2} $$

where I dropped the $i$ at the end because, as a constant multiple
inside a logarithm, it disappears into the constant of integration,
and we conclude that

$$\int \csc x \,{\rm d}x = \ln \tan {x\over2} $$

\endexample

\vfil\eject

\example Compute $\int {{4^x-1}\over{2^x+1}} {\rm d}x$
\label{integrate 4^x-1/2^x+1}

We'll use the field ${\mathbb C}(x,\Psi = \exp(x \ln 2))$; $\Psi' =
(\ln 2)\Psi$ and the representation (see Example
\ref{represent 4^x+1/2^x+1}):

$$ \frac{\Psi^2-1}{\Psi+1} = \Psi-1$$

So we need to find a solution of the form $a\Psi + \bar{b}$ ($\bar{b}$
can include additional logarithmic elements) that satisfy the Risch
equations:

$$a'\Psi + a\Psi' = a'\Psi + a(\ln 2)\Psi = \Psi \qquad a' + a(\ln 2) = 1$$
$$\bar{b}' = -1$$

Both equations have fairly obvious solutions:

$$a = \frac{1}{\ln 2} \qquad \bar{b}=-x$$

So our solution is

$$\int {{4^x+1}\over{2^x+1}} {\rm d}x = \frac{1}{\ln 2}\Psi - x =
\frac{1}{\ln 2}2^x - x$$

\endexample


\vfil\eject

\example Compute $\int {{4^x+1}\over{2^x+1}} {\rm d}x$
\label{integrate 4^x+1/2^x+1}

We'll use the field ${\mathbb C}(x,\Psi = \exp(x \ln 2))$; $\Psi' =
(\ln 2)\Psi$ and the representation (see Example
\ref{represent 4^x+1/2^x+1}):

$$ \frac{\Psi^2+1}{\Psi+1} = \Psi-1+\frac{2}{\Psi+1}$$

Let's do the fractional part first since it will modify the polynomial
part.

$$\left[\ln(\Psi+1)\right]' = \frac{(\Psi + 1)'}{\Psi+1} = \frac{(\ln
2)\Psi}{\Psi + 1} = \ln 2 - \frac{\ln 2}{\Psi+1}$$

$$-\frac{2}{\ln 2}\left[\ln(\Psi+1)\right]' = \frac{2}{\Psi+1} - 2$$

We can rewrite our integrand as:

$$ \Psi + 1 - \frac{2}{\ln 2}\left[\ln(\Psi+1)\right]'$$

So we need to find a solution of the form $a\Psi + \bar{b}$ ($\bar{b}$
can include additional logarithmic elements) that satisfy the Risch
equations:

$$a'\Psi + a\Psi' = a'\Psi + a(\ln 2)\Psi = \Psi \qquad a' + a(\ln 2) = 1$$
$$\bar{b}' = 1$$

Both equations have fairly obvious solutions:

$$a = \frac{1}{\ln 2} \qquad \bar{b}=x$$

So our solution is

$$\int {{4^x+1}\over{2^x+1}} {\rm d}x = \frac{1}{\ln 2}\Psi + x  - \frac{2}{\ln 2}\left[\ln(\Psi+1)\right] =
\frac{2^x}{\ln 2} + x - \frac{2}{\ln 2}\ln(2^x+1) $$

\endexample

\example Show that $\int \frac{\sin x}{x} \, {\rm d}x$ has no Liouvillian form

Again, we'll operate in ${\mathbb C}(x, \phi = \exp \,ix)$, use
Euler's relationship $e^{ix}=i\sin x + \cos x$, and evaluate

$$\int \frac{\phi - \phi^{-1}}{2ix} \,dx$$

Let's begin by writing the integrand in the form of a rational
function in ${\mathbb C}(x)(\phi)$, i.e, a ratio
of $\phi$-polynomials, with coefficients in ${\mathbb C}(x)$:

$$2i \int \frac{\frac{1}{x}\phi^2 - \frac{1}{x}}{\phi} \,dx$$

We want to split the denominator into its normal and special
components, by factoring it into irreducible polynomials and
classifying each one as normal or special.  In this case, the
factoriziation is trivial, and we know from theorem \ref{basic
exponential properties} that $\phi$ is special.

Can we have any logarithms in our integral?  Let's see.
Any logarithm of a rational function can be factored and
split into separate logarithms using basic properties
of a logarithms:

$$\ln ab = \ln a + \ln b \qquad\qquad \ln\frac{a}{b} = \ln a - \ln b$$

So, we need only consider logarithms of irreducible polynomials.

Theorem \ref{basic exponential properties} also tells us that we can
have no normal polynomials in denominator of our integral,


The integral must have the form $a_1 \phi + a_{-1} \frac{1}{\phi}$
with $a_1, a_{-1} \in {\mathbb C}(x)$ and must satisfy the equations:

$$\frac{1}{2ix} = a_1' + i a_1 \qquad - \frac{1}{2ix} = a_{-1}' - i a_{-1}$$

\endexample
