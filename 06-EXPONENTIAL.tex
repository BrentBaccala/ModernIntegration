
\setcounter{chapter}{5}
\mychapter{The Exponential Extension}

\vfil\eject

\example Compute $\int \sin x \, {\rm d}x$

We'll operate in ${\mathbb C}(x, \phi = \exp \,ix)$ and evaluate

$$\int \frac{\phi - \phi^{-1}}{2i} \,dx$$

The first step is to normalize the integrand, first by transforming it
into a rational function:

$$\int \frac{\phi^2 - 1}{2i \phi} \,dx $$

and next by using polynomial division to split it into a polynomial
and a proper fraction with a monic denominator:

$$\int \left[ \frac{1}{2i} \phi - \frac{1}{2i} \frac{1}{\phi} \right] \,dx $$

The integral must have the form $a_1 \phi + a_{-1} \frac{1}{\phi}$
with $a_1, a_{-1} \in {\mathbb C}(x)$ and must satisfy the equations:

$$\frac{1}{2i} = a_1' + i a_1 \qquad - \frac{1}{2i} = a_{-1}' - i a_{-1}$$ 

These are very simple Risch equations that can be solved by inspection
to obtain $a_1 = a_{-1} = -\frac{1}{2}$, so

$$\int \frac{\phi - \phi^{-1}}{2i} \,dx = -\frac{1}{2}(\phi + \phi^{-1})
 = -\frac{1}{2}(e^{ix} + e^{-ix}) = -\cos x$$

\endexample

\vfil\eject

\example Compute $\int \csc x \, {\rm d}x$

We'll operate in ${\mathbb C}(x, \phi = \exp \,ix)$ and evaluate

$$\int {{2i}\over{\phi - \phi^{-1}}} \,dx = \int 2i {{\phi}\over{\phi^2 - 1}} \,dx$$

Now we want a partial fractions expansion.  We could use a resultant,
or the Euclidean G.C.D. algorithm, but it's simpler to just note that
the denominator's a difference of squares and compute:

$${{c_1}\over{\phi-1}} + {{c_2}\over{\phi+1}} = {{\phi}\over{\phi^2-1}} $$

$${{c_1}(\phi+1)} + {{c_2}(\phi-1)} = \phi \qquad c_1 = c_2 = {1\over2} $$

giving us

$$\int i \big[ {{1}\over{\phi-1}} + {{1}\over{\phi+1}} \big] dx$$

Now, we have irreducible, square-free, non-monomial denominators, so
the answer (if it exists) must be expressed in terms of their
logarithms (Theorem \ref{basic exponential properties}):

$$ d \ln(\phi+1) = {{d(\phi+1)}\over{\phi+1}} = {{i\phi\,dx}\over{\phi+1}} = [1 - {{1}\over{\phi+1}}]i\,dx$$

$$ d \ln(\phi-1) = {{d(\phi-1)}\over{\phi-1}} = {{i\phi\,dx}\over{\phi-1}} = [1 + {{1}\over{\phi-1}}]i\,dx$$

so the integral can be written:

$$\int {{d(\phi-1)}\over{\phi-1}} - {{d(\phi+1)}\over{\phi+1}} = \ln (\phi-1) - \ln(\phi+1) = \ln({{\phi-1}\over{\phi+1}})$$
$$ = \ln ({{e^{ix}-1}\over{e^{ix}+1}}) = \ln ({{e^{ix/2}-e^{-ix/2}}\over{e^{ix/2}+e^{-ix/2}}}) = \ln i {{\sin {x\over2}}\over{\cos {x\over2}}} = \ln \tan {x\over2} $$

where I dropped the $i$ at the end because, as a constant multiple
inside a logarithm, it disappears into the constant of integration,
and we conclude that

$$\int \csc x \,{\rm d}x = \ln \tan {x\over2} $$

\endexample

\example Prove that $\int e^{-x^2}\, {\rm d}x$ has no elementary form

We'll use ${\mathbb C}(x, \phi = \exp\, -x^2)$, so $\phi' = -2x$ and
study

$$\int \phi \, {\rm d}x$$

We know from Theorem \ref{basic exponential properties} that our
solution, if it exists, must have the form $a\phi$, where $a \in
{\mathbb C}(x)$, and $a$ must satisfy the Risch equation:

$$a' - 2x a = 1$$

As Risch equations go, this one is easy!  Let's first note that
$a$ can not have a non-trivial denominator, since

$$a' = 1 + 2x a$$

We factor $a$'s denominator into irreducible factors, recall that in
${\mathbb C}(x)$, irreducible factors in the denominator always
increase in degree on differentiation, and see that the right hand
side of this equation has no denominator, therefore neither can the
left hand side ($a'$) or $a$ itself.

So $a$ must be a polynomial.  Returning to the original Risch
equation, we see that $a'$ will have degree ${\rm Deg\,} a - 1$ while
$2xa$ will have degree ${\rm Deg\,} a + 1$, so the left hand side must
have degree ${\rm Deg\,} a + 1$, while the right hand side has degree 0.

We conclude that no solution to this equation exists in ${\mathbb C}(x)$,
so the integral can not be expressed in elementary form.

\endexample

\vfil\eject

\example Compute $\int {{4^x-1}\over{2^x+1}} {\rm d}x$
\label{integrate 4^x-1/2^x+1}

We'll use the field ${\mathbb C}(x,\Psi = \exp(x \ln 2))$; $\Psi' =
(\ln 2)\Psi$ and the representation (see Example
\ref{represent 4^x+1/2^x+1}):

$$ \frac{\Psi^2-1}{\Psi+1} = \Psi-1$$

So we need to find a solution of the form $a\Psi + \bar{b}$ ($\bar{b}$
can include additional logarithmic elements) that satisfy the Risch
equations:

$$a'\Psi + a\Psi' = a'\Psi + a(\ln 2)\Psi = \Psi \qquad a' + a(\ln 2) = 1$$
$$\bar{b}' = -1$$

Both equations have fairly obvious solutions:

$$a = \frac{1}{\ln 2} \qquad \bar{b}=-x$$

So our solution is

$$\int {{4^x+1}\over{2^x+1}} {\rm d}x = \frac{1}{\ln 2}\Psi - x =
\frac{1}{\ln 2}2^x - x$$

\endexample


\vfil\eject

\example Compute $\int {{4^x+1}\over{2^x+1}} {\rm d}x$
\label{integrate 4^x+1/2^x+1}

We'll use the field ${\mathbb C}(x,\Psi = \exp(x \ln 2))$; $\Psi' =
(\ln 2)\Psi$ and the representation (see Example
\ref{represent 4^x+1/2^x+1}):

$$ \frac{\Psi^2+1}{\Psi+1} = \Psi-1+\frac{2}{\Psi+1}$$

Let's do the fractional part first since it will modify the polynomial
part.

$$\left[\ln(\Psi+1)\right]' = \frac{(\Psi + 1)'}{\Psi+1} = \frac{(\ln
2)\Psi}{\Psi + 1} = \ln 2 - \frac{\ln 2}{\Psi+1}$$

$$-\frac{2}{\ln 2}\left[\ln(\Psi+1)\right]' = \frac{2}{\Psi+1} - 2$$

We can rewrite our integrand as:

$$ \Psi + 1 - \frac{2}{\ln 2}\left[\ln(\Psi+1)\right]'$$

So we need to find a solution of the form $a\Psi + \bar{b}$ ($\bar{b}$
can include additional logarithmic elements) that satisfy the Risch
equations:

$$a'\Psi + a\Psi' = a'\Psi + a(\ln 2)\Psi = \Psi \qquad a' + a(\ln 2) = 1$$
$$\bar{b}' = 1$$

Both equations have fairly obvious solutions:

$$a = \frac{1}{\ln 2} \qquad \bar{b}=x$$

So our solution is

$$\int {{4^x+1}\over{2^x+1}} {\rm d}x = \frac{1}{\ln 2}\Psi + x  - \frac{2}{\ln 2}\left[\ln(\Psi+1)\right] =
\frac{2^x}{\ln 2} + x - \frac{2}{\ln 2}\ln(2^x+1) $$

\endexample
