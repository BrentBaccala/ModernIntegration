
\documentclass{article}

\usepackage{nopageno}

\usepackage{vmargin}
\usepackage{times}
\usepackage{graphics}
\usepackage{amsmath, amscd, amsxtra, amsthm}
\usepackage{amssymb}
\usepackage{framed}
\usepackage{mdframed}
\usepackage[dvipsnames]{xcolor}
\usepackage[retainorgcmds]{IEEEtrantools}

\newtheorem*{lemma}{Lemma}

%% Suggested by http://tex.stackexchange.com/questions/349580
\usepackage{array}

%% Define a new column declaration
\newcolumntype{L}[1]{>{\raggedright\arraybackslash}p{#1}}

% These packages are used when embedding Maxima code

\usepackage{breqn}      % auto-break long equations - has to come after pythontex
\usepackage{listings}
\usepackage{fancyvrb}

\usepackage{comment}
%% \usepackage{minted}

% This package makes cut-and-paste of carets work right, so
% we can cut-and-paste from the PDF document into Maxima.

\usepackage[T1]{fontenc}



\newcommand{\Bold}[1]{\mathbf{#1}}
\newcommand{\FF}{\Bold{F}}
\newcommand{\CC}{\Bold{C}}
\newcommand{\QQ}{\Bold{Q}}
\newcommand{\ZZ}{{\mathbb Z}}
\newcommand{\PP}{{\mathbb P}}
\newcommand{\Norm}{{\rm Norm}}
\newcommand{\QQbar}{\overline{\QQ}}

\parindent 0pt
\parskip 12pt

\begin{document}




\title{Decomposition of ideals in function fields}
\author{Brent Baccala}
\date{October 9, 2019}
\maketitle

%\begin{abstract}
%\end{abstract}

The {\tt decomposition} method in Sage's {\tt
  FunctionFieldMaximalOrder\_global} class uses the ``Buchman-Lenstra''
algorithm described in \cite{cohen} section 6.2.  Buchman-Lenstra,
however, depends on operations in prime characteristic, and is thus
only suitable for number fields and function fields over $\FF_p$.

To perform decomposition in characteristic zero, I've developed
the alternate algorithm described in this paper, and implemented
in the {\tt decomposition} method of Sage's {\tt FunctionFieldMaximalOrder\_rational} class.

Given a function field $F$ with a maximal order $O$ and an
algebraic field extension $F'/F$ with maximal order $O' \supset O$.
We consider a prime ideal $P$ of $O$, and we wish to find the prime
ideals $P'_1,...,P'_k$ of $O'$ that lie over $P$, in the sense that
$P^e \subset P'_i$, in fact, $P^e = \cap_i P'_i$, i.e, we seek the
primary decomposition of $P^e$.

$P^e$ is the extension of $P$ in $O'$, i.e, the ideal of $O'$
generated by the elements of $P$.  It is not necessarily prime, so the
quotient ring $O' \bmod P^e$ is not necessarily either a field or an
integral domain.  It is, however, an Artinian ring and a
finite-dimensional algebra over the field $O \bmod P$.  Sage's
finite-dimensional algebra subsystem implements\footnote{Johan Bosman,
  personal email, June 19, 2019} the algorithm from Section 7 of
\cite{khuri} to find all of the algebra's maximal ideals (all
prime ideals of an Artinian ring are also maximal).

$$O \stackrel{\phi}{\rightarrow} O' \stackrel{\psi}{\rightarrow} O'\bmod P^e$$

Thus, we can easily find all maximal ideals of the ring $O' \bmod P^e$.  Since
the contraction of a maximal ideal is maximal, the maximal ideals of
$O'\bmod P^e$ are maximal in $O'$ (via contraction along $\psi$).  Since
$\psi$ is surjective, any maximal ideal in $O'$ that contains $P^e$ maps
to a maximal ideal in $O'\bmod P^e$ (wikipedia Ideal), so there is a one-to-one
relationship between the maximal ideals in $O'\bmod P^e$ and the maximal ideals in $O'$
that contain $P^e$ -- exactly what we're looking for.

So, given a maximal ideal $P_1$ of $O'\bmod P^e$, how can we extract the
pertinent information (generators, ramification index, relative
degree, $\beta$) for its corresponding maximal ideal in $O'$?

\begin{enumerate}
\item {\bf Generators}

The contraction of $P_1$ is its preimage
under $\psi$, so a set of generators of the contraction can be formed
just by lifting $P_1$'s generators from $O'\bmod P^e$ to $O'$ and
appending the generators of $P^e$.

\item {\bf Ramification Index}

To see how to compute the ramification
index, let's begin by studying how to characterize the ideals
of $O'\bmod P^e$

\begin{lemma}
$P^e$ consists of all elements in $O'$ with valuation greater than or
equal to the ramification index at {\it all} places lying over $P$.
\end{lemma}
\begin{proof}
$O'$ consists of all elements with valuation greater than or equal to
zero at every finite place.
$P$ contains at least one element $u$ (any uniformizing variable will do) with valuation equal
to the ramification index at {\it all} places lying over $P$.

Therefore, given {\it any} element $e \in O'$ with valuation greater than or
equal to the ramification index at {\it all} places lying over $P$,
we can use the Strong Approximation Theorem to find an element in $f \in O'$
such that $fu$ has the same valuation as $e$ at all places lying over $P$.
Then $\frac{e}{fu} \in O'$ and $e = \frac{e}{fu} f u$ shows that $e \in P^e$.
\end{proof}

\begin{mdframed}
{\bf\cite{stich} Theorem 1.6.5 (Strong Approximation Theorem).}
Let $S \varsubsetneqq \PP_F$ be
a proper subset of $\PP_F$ and $P_1 , ... , P_r \in S$. Suppose there are given elements
$x_1 , ... , x_r \in F$ and integers $n_1 , ... , n_r \in \ZZ$. Then there exists an element
$x \in F$ such that
$$\nu_{P_i} (x - x_i ) = n_i \qquad (i=1,...,r), \quad{\rm and}$$
$$\nu_P (x) \ge 0 \qquad \forall P \in S \backslash \{P_1 , ... , P_r\}.$$
\end{mdframed}

If a function $f$ has valuation greater than or equal to the
ramification index at a place $P_1$, it can be reduced $\bmod P^e$ by
using the Strong Approximation Theorem to construct a function in
$P^e$ with valuation equal to the ramification index at $P_1$ and
valuation greater than $f$'s valuation at all other places over $P$
and adding it to $f$.  Thus $O' \bmod P^e$ contains no functions with
valuations greater than the ramification indices.

So, $O' \bmod P^e$ consists of equivalence classes of functions
characterized by a tuple of valuations at each place over $P$, with
each valuation no larger than the ramification index at that place.
The functions with valuation equal to the ramification index
at all places over $P$ are in $P^e$, and therefore correspond
to the zero ideal.

Each prime ideal in $O' \bmod P^e$ is characterized by a tuple like
$(1,0,...,0)$ (i.e, a single one and the rest of its elements zero).
Squaring it will produce an ideal characterized by $(2,0,...,0)$.
Continue raising the prime ideal to higher and higher powers until
we've obtained an ideal characterized by $(r,0,...,0)$ ($r$ being the
ramification index).  Raising the ideal to higher powers will continue
producing this same ideal, a manifestation of the Artinian condition
guaranteeing that the descending chain of power ideals will stablize.

Thus, we can find the ramification index by raising a prime ideal
in $O'\bmod P^e$ to successively higher powers until it stabilizes.

{\bf Ex:} $y^2=x$; $P=(x-1)$.

$P^e = (y^2-x, x-1)$ decomposes into $P_1=(x-1,y-1)$
and $P_2=(x-1,y+1)$ with ramification one at both places.
Ideals in $O'\bmod P^e$ are characterized by tuples $(0,0)$, $(1,0)$, $(0,1)$, and $(1,1)$.
$(0,0)$ corresponds to the unit ideal $(1)$, $(1,0)$ is $(y-1)$,
$(0,1)$ is $(y+1)$, and $(1,1)$ is the zero ideal $(0)$.  Our theory
suggests that squaring $(y-1)$ will stabilize it, and we verify
that $(y-1)^2 \equiv -2(y-1) \bmod P^e$.

{\bf Ex:} $y^2=x$; $P=(x)$.

$P^e=(y^2-x,x)$ decomposes into a single ideal $P_1=(y)$
with ramification two.
Ideals in $O'\bmod P^e$ are characterized by tuples $(0)$, $(1)$, and $(2)$,
with $(0)$ corresponding to the unit ideal $(1)$, $(1)$ corresponding to
the ideal $(y)$, and $(2)$ corresponding to the zero ideal $(0)$.  Our
theory leads us to believe that squaring $(y)$ will produce the zero
ideal and, indeed, $y^2 \equiv 0 \bmod P^e$.

\item {\bf Relative Degree}

\cite{stich} Definition 3.1.5 defines the relative degree
of $P'$ over $P$ as $[F'_{P'}:F_P]$, where $F_P$ (the {\it residue
  class field} of P) is defined (\cite{stich} Definition 1.1.14) as
$O_P/P$ where $O_P$ is the valuation ring associated with $P$.

In our notation, $F_P$ is $O \bmod P$ and $F'_{P'}$ is $O' \bmod P_1$.
Remember that $O' \bmod P^e$ is a finite-dimensional algebra over
$F_P$.  Since

$$F'_{P'} = O' \bmod P_1^c \cong O' \bmod P_e \bmod P_1$$

and we have an $F_P$-basis for $P_1$ in $O' \bmod P_e$, we see that the
dimension of $F'_{P'}$ over $F_P$ is simply the $F_P$-dimension of $O' \bmod P_e$
minus the $F_P$-dimension of $P_1$.  Our finite dimensional algebra
code gives us an $F_P$-basis for $P_1$, so its dimension is just
the length of that basis.

\item $\boldsymbol{\beta}$

Finally, for computing valuations using \cite{cohen} Algorithm 4.8.17,
we wish to compute $\beta$, an element in $O'$ but not in $P^e$, and
with $\beta P_1 \subseteq P^e$.

Another way of characterizing $\beta$ is that its valuation is
non-negative at all finite places, $r-1$ at $P_1$, and $r$ at all of
$P_1$'s conjugates, where $r$ is the (local) ramification index.

Proof: Use the Strong Approximation Theorem to construct an element in
$P_1$ with valuation one at $P_1$ and valuation zero at all of its
conjugates.  Multiplying this element by $\beta$ must produce an
element in $P^e$, so $\beta$ must have the stated properties.

Working again in $O' \bmod P^e$,
we see that $\beta$'s image is not zero, but multiplying it
by each of $P_1$'s generators produces zero.
Since $\beta$ is in $O'$ (an $O$-module), we regard $\beta$ as
a vector in k[x] w.r.t. the basis of $O'$.  As long as
at least one element in this vector is not zero,
$\beta$ will not be in $P^e$.  To ensure that
$\beta P_1 \subseteq P^e$, multiplying $\beta$ by each of $P_1$'s
generators must produce a vector whose elements are all zero.
We can ensure that all this
occurs by constructing a matrix in $O \bmod P^e$,
and finding a non-zero vector in
the matrix's kernel.

\end{enumerate}

\begin{comment}

\vfill\eject
\section*{Field Extensions}

Constructing field extensions is currently somewhat problematic in Sage,
which creates issues when the function field code builds residue fields.

For example, consider the following sequence:

\begin{sageblock}
A.<a> = GF(5)[]
B.<b> = GF(5).extension(modulus=a^2-2)
b^2
C.<c> = B[]
D.<d> = B.extension(modulus=c^2-b)
d^2
d^4
\end{sageblock}

{\tt d} behaves the way we would expect, but {\tt D} isn't constructed
as $\FF_{5^4}$, as we might expect.  Yet we can get $\FF_{5^4}$ in
a more roundabout way, as follows:

\begin{sageblock}
E.<e> = B.extension(2)
F.<f> = E[]
r = (f^2-b).roots()
d = r[0][0]
d^2
d^2 == b
d^4
\end{sageblock}

This is the method used in {\tt FunctionFieldMaximalOrder\_rational}'s {\tt \_residue\_field\_global}.

We can attempt to do something similar with number fields:

\begin{sageblock}
A.<a> = QQ[]
B.<b> = QQ.extension(a^3 - 2)
C.<c> = B[]
D.<d> = B.extension(c^2-b)
d^2
E = D.absolute_field('e')
e = E.0
hom = E.structure()[1]
hom(d) == e
e^2
d^2 == e^2
e^2 == b
\end{sageblock}

\section*{}

Why are a lot of the test cases failing?

The basis calculation in {\tt FunctionFieldIdeal\_global}.

When we ask for {\tt gens()}, the code checks for a cached result from
{\tt \_gens\_two} and returns it if it exists, others it gives us
(more or less) the generators it was created with.

A problem with this is that the print representation of the ideal can
change after you call {\tt gens\_two()} on it.

\end{comment}

\begin{thebibliography}{9}

\bibitem[Coh1993]{cohen} Henri Cohen, {\it A Course in Computational
  Algebraic Number Theory}. Graduate Texts in Mathematics
  138. Springer, 1993.

\bibitem[Stich2009]{stich} Stichtenoth, Henning, {\it Algebraic
  function fields and codes}.  Vol. 254. Springer Science \& Business
  Media, 2009.

\bibitem[Khuri2004]{khuri} Khuri-Makdisi,
{\it Asymptotically Fast Group Operations on
Jacobians of General Curves}, 2004.

{\tt https://arxiv.org/pdf/math/0409209v2.pdf}

\end{thebibliography}

\end{document}
